%% Anhang.

%%SuppressWarnings("InputStyle")
%%SuppressWarnings("ContentCheck")

\kapitel{Zu diesem Band}

Der edierte Text folgt grundsätzlich zeichengetreu der Buchausgabe\dopp{}

Max Mohr\dopp{} Die Heidin. München, Georg Müller Verlag Aktiengesellschaft
1929.

Die in dieser Vorlage in Antiqua gesetzte Abschnitte kennzeichnen nicht nur
fremdsprachliche, sondern auch fremdsprachlich motivierte Passagen
(\fremdsprachlich{diamants}\,, S.~\pageref{lS47-1}). Erzählerische
Hervorhebungen sind ebenfalls in Antiqua gesetzt (\fremdsprachlich{Dr.}\,,
S.~\pageref{lS48-1}).
Um diese Informationen zu wahren, entscheidet sich diese Buchausgabe,
diese Stellen \begin{it}kursiv\end{it} wiederzugeben.

Doppelte Anführungszeichen \frqq{} und \flqq{} werden hier durch
\flqq{} und \frqq{}\,, einfache Anführungszeichen \glq{} und \grq{}
durch \flq{} und \frq{} wiedergegeben. Doppelte Anführungszeichen
innherhalb wörtlicher Rede werden durch einfache ersetzt.

Die Großschreibung der Pronomina und der Reflexivpronomina in der wörtlichen Rede
bei der Adressierung einer Person wird in dieser Ausgabe konsquent
umgesetzt. 

An folgenden Stellen (Seite, Zeile) wurde konkret in den Text der
Vorlage eingegriffen\dopp{}

\theendnotes
