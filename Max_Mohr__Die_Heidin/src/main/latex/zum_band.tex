%% Anhang.

%%SuppressWarnings("InputStyle")
%%SuppressWarnings("ContentCheck")

\kapitel{Zu diesem Band}

Der edierte Text folgt grundsätzlich zeichengetreu der Buchausgabe\dopp{}

Max Mohr\dopp{} Die Heidin. München, Georg Müller Verlag Aktiengesellschaft
1929.

Die in dieser Vorlage in Antiqua gesetzte Abschnitte kennzeichnen nicht nur
fremdsprachliche, sondern auch fremdsprachlich motivierte Passagen
(\fremdsprachlich{diamants}\,, S.~\pageref{lS47-1}). Erzählerische
Hervorhebungen sind ebenfalls in Antiqua gesetzt (\fremdsprachlich{Dr.}\,,
S.~\pageref{lS48-1}).
Um diese Informationen zu wahren, entscheidet sich diese Buchausgabe,
diese Stellen \begin{it}kursiv\end{it} wiederzugeben.

An folgenden Stellen (Seite, Zeile) wurde in den Text der
Vorlage eingegriffen\dopp{}

\theendnotes
