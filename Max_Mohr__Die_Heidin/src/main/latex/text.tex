\documentclass[fontsize=11pt,headings=small]{scrbook}

\setlength{\parskip}{0pt}

\usepackage[utf8]{inputenc}

\usepackage[T1]{fontenc}

\usepackage[german]{babel}

\usepackage[paperwidth=135mm,paperheight=215mm]{geometry}

\usepackage[activate={true,nocompatibility},final,protrusion=true,tracking=true,kerning=true,spacing=true,factor=1100,stretch=10,shrink=10]{microtype}

%\usepackage{hfoldsty}

%\linespread{1.00}

\typearea[5mm]{11}

\usepackage{lmodern}

\usepackage{endnotes}

\usepackage[pagewise]{lineno}

\usepackage{lipsum}

%\usepackage{poemscol}

\begin{document}




% --------------------------------------------------------------
% **************** Definitionen, Makros Anfang *****************

% Keine Kopfzeile, aber Seitennummer.
\pagestyle{plain}

% Marker für Endnote im Text entfernen.
\def\makeenmark{}

% Abstand zwischen Absätzen.
%\def\abstand{\\\parbox[t][3ex]{1cm}{}\\}
\def\abstand{\\\\\\}

% Eingriff in Text der Vorlage.
\def\eingriff#1{\linelabel{\theenmark}\endnote{\makebox[0em][r]
{\pageref{\theenmark}}, \makebox[1em][r]{\ref{\theenmark}}\makebox[1em]{}#1}}

% Kapitelüberschrift.
\def\kapitel#1{
\cleardoublepage
\parbox[c][6ex]{1cm}{}\addsec{#1}}

% Fremdsprachliche Textstellen.
\def\fremdsprachlich#1{\textit{#1}}

% Partseiten ohne Seitenzahl.
\renewcommand*{\partpagestyle}{empty}

% Chapterseiten ohne Seitenzahl.
\renewcommand*{\chapterpagestyle}{empty}

% Anführungszeichen links.
\def\aa{\flqq\kern .05em}
\def\aanah{\flqq}

% Anführungszeichen rechts.
\def\ae{\kern .05em\frqq}

% Doppelpunkt.
%\def\dopp{\thinspace{\kern .08em} :}
\def\dopp{\kern .03em :}

% Fragezeichen.
\def\frag{\kern .05em ?}

% Ausrufezeichen.
\def\ausr{\kern .05em !}

\def\punkte{\kern .08em \ldots}

% ***************** Definitionen, Makros Ende ******************
% --------------------------------------------------------------




% --------------------------------------------------------------
% ********************** Vorspann Anfang ***********************

\author{Max Mohr}
\title{Die Heidin}
\date{}
\lowertitleback{

\begin{center}
Der Text \aa Die Heidin\ae{} in diesem Buch folgt
grundsätzlich zeichengetreu der Ausgabe\dopp{} \mbox{Max Mohr.} \mbox{Die Heidin.}
\mbox{München, Verlag Georg Müller. 1929.}
Abweichungen sind im Anhang vermerkt.
\end{center}

\begin{center}
@version@\\
@timestamp@
\end{center}

\begin{center}
Dieses Buch wurde erstellt mit \LaTeX und \KOMAScript.
\end{center}
}

\maketitle

% *********************** Vorspann Ende ************************
% --------------------------------------------------------------




% --------------------------------------------------------------
% *********************** Roman Anfang *************************

\addpart{Die Heidin}

% Ab hier Zeilennummer pro Seite erfassen.
\linenumbers
\setpagewiselinenumbers
\modulolinenumbers[5] % Zum Ausblenden z. B. Wert 100 einsetzen.

\addchap{Erster Teil}
% Erster Teil.

\kapitel{Erstes Kapitel}

Ostwind, Wind aus dem Osten, ein mächtiger Hochdruck über
dem ganzen Kontinent. Schon zwei Juni-Wochen hielt es an.
Die alte Schlechtwetterhexe Europa strahlte in strammer
Wolkenlosigkeit. Sie tat, als wäre sie jung und ge\-sund wie
am fünften Schöpfungstag, als wäre sie unberührt von dem
Geschöpf des sechsten Schöpfungstages, unvergiftet von den
Dünsten seiner Großhirnrinde, unverseucht von den Ba\-%
zillen seiner Gier. Als wäre nichts geschehn, so tat die Dame
Europa, und rollte frisch und naiv dahin, mit reinem Atem,
auf ihrem kleinen Platz auf der großen Kugel.

Aber was für eine Figur sie hat, die gute Alte, auch beim
besten Wetter\ausr{} Ohne Grazie spreizt sie ihre drei Beine ins
Mittelmeer, doch sie sind schmutzig geblieben trotz des klassi\-%
schen Dauerbades. Plump ist ihr Leib und man kann nicht
sehn, was Bauch ist und was Brust ist. Einst war Frankreich
ihr süßer Bauch, Deutschland ihr bitteres Herz, Rußland ihr
geheimnisvoller großer Arsch, zusammengewachsen mit Asien.
Jetzt sind die Linien verwischt, viele politische Fettfalten
laufen wirr über das welke Fleisch, überallhin und nirgend\-%
wohin, du kannst nicht mehr erkennen, wo diese Dame atmet
und wo sie verdaut. Mit ihrem skandinavischen Fangarm
klammert sie sich ans Eis des Nordpols, mit ihrem englischen
Fangarm verkrallt sie sich im Ozean\dopp{} wo aber ist ihr Kopf,
ihre Augen, zu schauen, und ihre Ohren, zu hören\frag{} Wer
weiß,\eingriff{weiß, vielleicht ] weiß vielleicht}
vielleicht ins Meer gestürzt, seit Äonen, untergegangen,
dahin.

Gut erhalten ist ihr Schamhügel, die Alpen, wenigstens dort,
wo noch keine Fremdenzentralen angesiedelt sind. Dort gibt
es noch Landschaften, darin das Fleisch der alten europäischen
Natur noch lebt. Dort lagert noch in versteckten Seitentälern
jene körperliche Urhülle der Natur, darin der Mensch noch
atmen kann ohne das Asthma des Verstandes, atmen kann
one das Asthma der Ideen und Gegenideen. Es ist der Rest
jener lebendigen Urhülle der Natur, die einst den ganzen
Ball umspannte und in bewahrte vor der Nacktheit im All.
In versteckten Seitentälern, über dem Boden der Erde,
hoffnungsvoll dahinkeimend, ein geheimnisvoll-heidnischer
Rest jener Urhülle der Natur, lebendiger Raum, dort ist es
noch zu spüren. So auch im Tal der Glonn.

Die Glonn liegt am Nordhang der bayrischen Alpen, in einem
kleinen südlichen Seitental des Tegernseer Tals, zwei Weg\-%
stunden vom See, zwei Wegstunden von der nächsten Post\-station
und Fremdenstation. Die Glonn besteht aus drei
Höfen\dopp{} Zum Krallen, Zum Sennen, Zum Lori. Beim Krallen
und beim Sennen wohnen eingesessene Kleinbauern, sechs
bis zehn Milchkühe stark, der Lori ist seit fünfundzwanzig
Jahren in den Besitz norddeutscher Städter übergegangen.
Er hat seinen alten Hausnamen allmählich verloren und ist
jetzt fast nur noch unter dem Namen seiner städtischen Be\-%
wohner bekannt, als Hersehof. Er liegt ein wenig abseits vom
Krallen und vom Sennen, am tiefsten bergwärts in der
Glonn. Nach Süden ist dann diese kleine Landschaft zu Ende,
ein paar steile Bergwiesen noch, danach Wald und Latschen
und Geröll und Fels, meist im Schnee oder im Reif oder im
Nebel, wenn nicht gerade ein kontinentaler Hochdruck auch bis
hierher gedrungen ist.

Dies ist Lea Herses Heimat. Oder, wie die Schulkinder ihre
ersten Liebesbriefe adressieren\dopp{} Lea Herse, Hersehof, Glonn,
Alpen, Europa, Erde, Welt.
\abstand
Sie trat aus dem Haus und pfiff ihrem Hund. In ihrer
Jackentasche kollerte ein dickes Stück Schinken, frisch herunter\-%
gesäbelt, ohne Einwickelpapier, gegen einen alten Militär\-%
revolver, der mit sechs Schüssen geladen war.

Der Wolfhund Maffa, ein Greis von vierzehn Hundejahren,
lag auf der anderen Seite des Hersehofs in der Mittagsonne.
Er überhörte mürrisch den Pfiff, der mit einem kleinen Echo
aus dem Berg zu ihm drang. Um diese Stunde pfiff man ihm
nicht, es war die Stunde der Reste-Fresserei, Maffas großer
Leidenschaft, vollzogen an den Katzenschüsseln, am Pferde\-%
bottich, am Hühnertrog. Auf die eigene Schale mit Reis und
Kalbsknochen konnte ein alter Herse-Hund verzichten, aber
die Reste der anderen Herse-Tiere mußte er haben, was es
auch war, rohe Kleie, alte Fischköpfe, verfaulte Rübenstrunke,
er hätte sich ohne dieses Zeug den ganzen Tag über so be\-%
fangen gefühlt wie ein Städter, der seine Tageszeitung nicht
gefressen hat.

Erst nach dem fünften Pfiff gab er es auf. Er stellte sich gäh\-%
nend hoch und warf einen verzweifelten Blick auf den Trog,
der noch immer von dreißig Hühnern umstellt war. Dann
schlenkerte er verdrossen zu seiner jungen Herrin und ver\-%
drossen hinter ihr her\dopp{} die Herse-Wiese entlang zum oberen
Herse-Gatter, durch das obere Gatter zur Mooswiese, über
die Mooswiese hinauf zum Herse-Wald, der außerhalb der
Umzäunung gelegen war.

Kurz vor dem Waldbrand, unter dem alten Kruzifiz auf der
Mooswiese, lag Terese Nüll, die frisch importierte Kusine
aus der Stadt. Lea sah den farbenprächtigen Fleck schon vom
Gatter aus. Sie ärgerte sich schon von weitem über die
knallige Hingegossenheit, mit der sich ihre Kusine in die Land\-%
schaft drapiert hatte. Es war die übertriebene Natürlichkeit,
mit der sich alle Städter der natur entgegenstreckten, ohne zu
fühlen, daß diese Aufdringlichkeit der Natur unangenehm
war\dopp{} aber daß auch Terese Nüll den Takt der Landschaft
verfehlte, war besonders schlimm.

War es nicht zum Heulen, daß diese Schwärmerin im laden\-%
blauen Leinen von morgen ab den Hersehof verwalten sollte\frag{}
Das also war die Herrin über Leas Haus und Land und
Getier für Monate, für Jahre vielleicht\frag{} Es war zum Heulen,
aber es gab kein Zurück mehr, die große Expedition stand
startbereit, nur keine Schwachheiten am Start. Es war ja
ein Glück, daß Terese Nüll gerade frei war und die Verwaltung
des Hofs auf unbestimmte Zeit übernehmen konnte.

Lea schob sich freundlich an die Hingegossene heran.

\aa Schöner Tag, was\frag{}\ae

\aa Märchenhaft.\ae

Terese wältze sich auf die Seite und blinzelte verzückt zu ihrer
Kusine empor. Sie war einunddreißig, fünf Jahre älter als
Lea. An Lebenserfahrung dünkte sie sich tausend Jahre älter
als Lea. Die wußte überhaupt nichts vom Leben außerhalb
der Glonn. Sie war durch die Zurückgezogenheit ihrer Mutter
fast nie in die Stadt oder unter die Menschen gekommen. Tante
Daniela hatte sich an ihrem Kind versündigt, das stand bei
der jüngeren Generation der Familie fest. Lea galt als ein
wenig zurückgeblieben bei ihren städtischen Vettern und Basen.

Trotzdem fühlten sie alle, wenn sie zu kurzem Besuch in die
Glonn kamen, eine seltsame Neugier nach Leas Dingen,
während Lea nicht im gerinsten neugierig war nach ihren
städtischen Dingen. So auch jetzt. Während Lea neben Terese
Halt machte und ruhig über sie hinweg ins Tal sah, musterte
Terese interessiert die Toilette zu ihren Häuptern.

Warum hatte sich Lea nach dem Mittagessen umgezogen,
obwohl kein Besuch zu erwarten war\frag{} Das hatte sie von ihrer
Mutter geerbt, das kannte man. Tante Danielas fünfmaliger
Umzug pro Tag, auch wenn jahrelang kein Besuch im Herse\-%
hof empfangen wurde, war in der Familie sprichwörtlich
geworden\dopp{} Stallhose, Reitkleid, Stallhose, Teekleid, Stall\-%
hose, Abendkleid. Und warum gerade dieses blendende Weiß,
zehn Tage nach dem Begräbnis der Mutter\frag{} Extra\frag{} Echt
\aa hersisch\ae\ausr{} Mißtrauisch blickte Terese auf die champagner\-%
farbenen Strümpfe und sah sie hoch oben in einer champag\-%
nerfarbenen Schlupfhose verschwinden.

\aa Warum so elegant, Leachen\frag{}\ae

\aa Wieso\frag{} Mein Tenniskleid vom vorigen Jahr.\ae

\aa Wohin damit\frag{}\ae

\aa In den Wald, Maffa totschießen.\ae

\aa Im Ernst\frag{}\ae

Terese setzte sich mit einem Ruck hoch.

\aa Hab ichs Dir noch nicht erzählt\frag{}\ae, fragte Lea kühl.

Sie wußte ganz genau, daß sie es Terese noch nicht erzählt
hatte. Alle Dinge, die mit ihrer großen Expedition zusammen\-%
hingen, behielt sie bis zur letzten Minute für sich.

\aa Der Krallenpeter hat doch schon gestern das Grab gegraben,
ganz hoch oben, bei der Grenztanne, es ist alles bereit.\ae

Terese starrte mit entsetzten, blauen Glotzaugen auf Maffa,
der sich neben seiner Herrin auf die Hinterhand gesetzt hatte.

\aa Und Du mußt ihn selbst erschießen\frag{} Steht das im Testament\frag{}\ae{}
Sie witterte überall eine Apartität Tante Danielas.

\aa Du bist ja verrückt\ae, sagte Lea. \aa Im Testament stehen andere Dinge,
keine solchen Kinkerlitzchen, die Du glaubst.\ae

\aa Bitte, laß es, Leachen, laß ihn mir, ich will ihn gut pflegen,
den armen Kerl.\ae

\aa Ach was\ausr{} Daß Du ihn gut pflegen willst, glaub ich Dir. Er
würde trotzdem den ganzen Tag am Gatter sitzen und nach
mir heulen. Er ist seit seiner Geburt bei mir\punkte{} Besser so,
Maffa, was\frag{}\ae

Maffa wedelte freudig mit dem Schwanz, dann lugte er
wieder gespannt ins Tal. Schon verschwammen vor seinen
Augen die Linien der Landschaft, doch er konnte noch jene
Stelle dort unten ahnen, wo sich das Hühnervolk um den Trog
drängte. Er sog nervös die Luft auf, aber er bekam nichts
in die Nase. In jungen Jahren hätte er bis hierher gewittert,
ob es dort unten um Fleischreste oder um Fischreste ging, jetzt
bekam er nichts mehr in die Nase. Nur ganz schwach den Dunst
seiner nahen Herrin, aber das war keine Neuigkeit für ihn,
das war sein Lebenselement wie Luft, Wasser, Erde.

\aa Ja, ja\ae, sagte Lea, \aa das Grab ist ein wenig zu klein geraten,
weil so dicke Wurzeln kamen, das ist dumm, ich muß ihn zu\-%
sammenrollen.\ae\eingriff{zusammenrollen.\ae{} ] zusammenrollen\ae.}

Terese hielt sich die Ohren zu, entsetzt über soviel Roheit, und
kreischte auf, als hörte sie schon den Schuß. \aa Schrecklich,
schrecklich\ausr\ae

\aa Na für Dich doch nicht, Du blödes Huhn\ausr\ae

\aa Bitte nicht in meiner Nähe\ausr\ae

\aa Keine Angst\ausr{} Du merkst nichts davon. Zum Tee bin ich
zurück\punkte{} Komm, Maffa, komm\ausr\ae

Sie schritt weiter, empor zum Wald, Maffa hinterher.

Das Loch, das der Krallenpeter ausgeworfen hatte, war
wirklich zu klein. Es enttäuschte, wie jedes offene Grab ent\-%
täuscht. Ein Grab sollte keine gewöhnliche Grube sein, dachte
sie, ein Grab sollte tief hinunter reichen, tief bis zum Mittel\-%
punkt der Erde hinunter, auch wenn es nur ein ordinäres
kleines Hundegrab war. Aber das Begräbnis kam später.
Vorerst mußte sie sich ganz darauf einstellen, Maffa auf die
beste Weise zu töten. Sie hatte ihren festen Plan.

Zuerst gab sie ihm das dicke Stück Schinken aus der Jacken\-%
tasche. Sie warf es ihm in hohem Bogen zu, fang, Maffa,
fang. Aber Maffa war kein gewandter Fänger mehr, er ließ
das kostbare Stück auf den Boden klatschen, ehe er sich langsam
und erstaunt darüber hermachte. Dann wollte sie ihn ermüden,
indem sie ihm kleine Äste schleuderte, bring, Maffa, bring.
Aber Maffa war ohne Lust beim Spiel, er schleppte die in\-%
teressantesten Schleudergeschosse nur aus altem Anstand
zurück. Schließlich legte sie sich neben die frische Grube aufs
Laub und simulierte Müdigkeit und Mittagsschlaf. Vielleicht
half das, vielleicht schlief er so am schnellsten ein. Denn sie
hatte sich vorgenommen, ihn im Schlaf zu erschießen. Wenn
er ganz in sich selbst hineingerollt war, wenn die Welt bereits
von ihm gegangen war, ein kurzer Knall zwischen Schlaf und
Tod, das war das beste.

Tatsächlich wirkte ihr Trick. Er legte sich friedlich neben sie
und schien einschlafen zu wollen. Sie kratzte ihn sanft an seiner
Lieblings-Kratzstelle im Kreuz. Doch es dauerte noch eine
Ewigkeit, bis er nicht mehr wedelte und nicht mehr blinzelte,
bis er ganz in sich selbst hineingerollt war, bis der Revolver
gezogen werden konnte.
\abstand
Das Vorspiel zu dieser kleinen Hunde-Tragödie im Herse-%
Wald lag weit zurück. Es war gespielt worden am 29. Sep\-%
tember 1901, die Szene war die Gruttenhütte im Kaiser\-%
gebirg gewesen, die Spieler Leas Mutter und Leas Vater.

Frau Daniela Herse hatte damals noch Daniela Oldenkott
gehießen und die Durchquerung des Kaisergebirgs über die
Steinerne Rinne und das Ellmauer Tor hatte damals noch
als schwierige alpine Tour gegolten. Tatsächlich war es für
ein fünfundzwanzigjähriges Ding von der Wasserkante eine
Leistung gewesen, diese Tour auf Daniela Oldenkotts Art
durchzuführen\dopp{} ohne Training, ohne Führer, allein. Aber der
Hauptspaß war gewesen, daß die Eltern Oldenkott der Schlag
getroffen hätte, hätten sie eine Ahnung gehabt, was damals
von ihrer Tochter als geheimer Abschluß einer Italienreise
selbständig unternommen worden war. Die Eltern hätte der
Schlag getroffen und den Onkel und die Tante, von denen
Daniela Oldenkott in den Schnellzug Bozen-Hamburg ver\-%
frachtet worden war, hätte ebenfalls der Schlag getroffen\dopp{}
das war der Hauptspaß jener Tour gewesen, denn es war die
Zeit der ersten Korsett-Revolution der jungen Mädchen.

Stolz und erhitzt ruhte Daniela Oldenkott um die Mittagszeit
jenes Septembertags in dem kleinen Geröllfeld unterhalb
des Ellmauertors, nachdem sie die Hauptschwierigkeit der
Tour, die plattigen Wandstufen der Steinernen Rinne,
glücklich hinter sich hatte. Hier war der Blick nach Süden,
nach den Gletschern und nach den Keesen der Zentrale,
noch von den letzten Stufen des felsigen Tors versperrt. Doch
gewaltig genug stürzten rechts und links von der romantischen
Ausreißerin die Wände des Predigtstuhls und der Fleisch\-%
bank in die Tiefe. Daniela hatte das Gefühl, allein auf der
Welt zu sein, ine einsame Heldin auf totem Mond, siegreich
über die Götter, abgesondert von den Menschen -- bis plötzlich
aus dem senkrechten Gestein das Predigtstuhls ein mensch\-%
licher Ruf zu ihr drang. Das gab eine kleine Ernüchterung.
Vor den Augen eines Menschen, der in jenen gewaltigen
Abstürzen steckte, zerschmolz ihr stolzer Höhenweg zu einer
Alten-Leute-Promenade. Dennoch war sie begeistert von der
Kühnheit des Rufers, sie nahm ihr Glas und tastete das
Gemäuer zu ihrer Rechten ab.

Lange Zeit bekam sie nur Fels ins Glas, glatten Fels und
zerklüfteten Feld, nassen Fels und ausgedörrten Fels, weißen
und roten und schwarzen und gelben Fels, dann eine tiefe
Schlucht, einen senkrechten Kamin, ein Seil-Ende, das zu
einer kleinen Geröllstufe in diesem Kamin herabpendelte,
endlich einen Menschen, der neben dem Seil-Ende an der
Kaminwand lehnte. Ein junges Männergesicht, das sie plötzlich
zum Greifen nah ins Glas bekam.

Sie hätte dieses Gesicht mit dem Glas abtasten können, aber
sie scheute sich, in die übermenschliche Einsamkeit jenes Men\-
schen einzudringen, während er ahnungslos vor sich hin döste
im heiligen Glotzen des müden Pioniers. Sie setzte schnell das
Glas ab und machte sich wieder auf den Weg.

Als nach einigen Minuten der Ruf wieder kam, fühlte sie, daß
der Mensch in dem schwarzen Riesenkamin nach ihr rief. Sie
stoppte und erschrak. War es ein Hilferuf\frag{} Wie hätte sie hier
Hilfe bringen können\frag{} Sie rief ein schrilles Wie zurück, Wie --
Wie -- Wie, aber sie verstand die Antwort nicht, das Echo über\-%
schlug die Worte von allen Seiten her.

% Erster Teil.

\kapitel{Zweites Kapitel}

\lipsum

\addchap{Zweiter Teil}
% Zweiter Teil.

\kapitel{Erstes Kapitel}

Zicke Zacke Hühnerkacke.
% Zweiter Teil.

\kapitel{Zweites Kapitel}

\lipsum

% Ab hier Zeilennummern wieder aus.
\endlinenumbers

% ************************ Roman Ende **************************
% --------------------------------------------------------------




% --------------------------------------------------------------
% *********************** Anhang Anfang ************************

\addpart{Anhang}

\addsec{Eingriffe in den Text}
\def\enotesize{\normalsize}
\theendnotes

\tableofcontents

% ************************ Anhang Ende *************************
% --------------------------------------------------------------


\end{document}
