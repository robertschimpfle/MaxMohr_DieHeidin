%%SuppressWarnings("InputStyle")
%%SuppressWarnings("ContentCheck")
\documentclass[fontsize=11pt,headings=small]{scrbook}

\setlength{\parskip}{0pt}

\usepackage[utf8]{inputenc}

\usepackage[T1]{fontenc}

\usepackage[german]{babel}

\usepackage[paperwidth=135mm,paperheight=215mm]{geometry}
%\usepackage[a4,frame,center]{crop}

\usepackage[activate={true,nocompatibility},final,tracking,kerning,spacing,stretch=25,shrink=20]{microtype}

\typearea[4mm]{11}

%\linespread{1.05}

\usepackage{lmodern}

\usepackage{endnotes}

\usepackage[pagewise]{lineno}

\usepackage{lipsum}

%\usepackage{titlesec}

\usepackage{color}

% Counters als Nummer verwenden.
\usepackage{refcount}

\usepackage{ifthen}

% Überschriften zentrieren.
\renewcommand*{\raggedsection}{\centering}

\begin{document}




% ******************* Definitionen, Makros *********************

% Abstand zwischen Absätzen.
\def\abstand{\vspace{\baselineskip}\vspace{\baselineskip}\\}

\def\abstanddrei{\vspace{\baselineskip}\vspace{\baselineskip}\vspace{\baselineskip}\\}

% Endnoten größer machen.
\def\enotesize{\small}

% Marker für Endnote im Text entfernen.
\def\makeenmark{}

% Eingriff in Text der Vorlage.
\newcommand{\eingriff}[2]{\linelabel{#1}\endnotetext{\makebox[0em][r]
{\pageref{#1}},
\makebox[1em][r]{\ref{#1}}\makebox[1em]{}#2}}

% Anmerkung.
\newcounter{anmseite}
\newcounter{anmrefseite}

% TODO Nicht mit Einzug arbeiten, sondern zwei Boxen nebeneinander:
% Eine für die Seitenzahl und eine gleich hohe für den Text
\newcommand{\anmerkung}[3]{%
\begin{addmargin*}[2.5em]{0em}%
\ifthenelse{\(\pageref{#1}=\arabic{anmrefseite}\)\AND\(\thepage=\value{anmseite}\)}%
{\hspace{-2.5em}\makebox[2em][c]{}\hspace{0.5em}}%
{\hspace{-2.5em}\makebox[2em][c]{\pageref{#1}}\hspace{0.5em}}%
\setcounter{anmrefseite}{\getpagerefnumber{#1}}%
\setcounter{anmseite}{\thepage}%
\begin{it}#2\end{it} -- #3%
\end{addmargin*}%
}

% Überschrift der Endnoten verschwinden lassen.
\renewcommand{\notesname}{\vspace{-1cm}}

% Kapitelüberschrift.
\def\kapitel#1{
\clearpage{}
\null
\vspace{2.5ex}
\addsec{#1}
\vspace{\baselineskip}
}

% Fremdsprachliche Textstellen.
\def\fremdsprachlich#1{\textit{#1}}

% Partseiten ohne Seitenzahl.
\renewcommand*{\partpagestyle}{empty}

% Chapterseiten ohne Seitenzahl.
\renewcommand*{\chapterpagestyle}{empty}

% Anführungszeichen links.
\def\aa{\flqq{}\kern .05em}
\def\aanah{\flqq{}}

% Anführungszeichen rechts.
\def\ae{\kern .05em\frqq{}}

% Doppelpunkt.
\def\dopp{\kern .03em :}

% Semikolon.
\def\semi{\kern .03em ;}

% Fragezeichen.
\def\frag{\kern .05em ?}

% Ausrufezeichen.
\def\ausr{\kern .05em !}

\def\punkte{\kern .08em \ldots{}}

\setkomafont{partentrypagenumber}{\color{white}}
\setkomafont{chapterentry}{\normalfont}
\setkomafont{chapterentrypagenumber}{\color{white}}

% ************************ Vorspann ****************************
%% Titel.

%%SuppressWarnings("InputStyle")
%%SuppressWarnings("ContentCheck")

\begin{titlepage}
\pagestyle{empty}

\begin{center}

Max Mohr / Die Heidin

\cleardoublepage

\null

\vspace{4ex}

\usekomafont{title}\huge
Max Mohr

\vspace{3ex}
*
\vspace{2ex}

\usekomafont{title}\Huge
Die Heidin

\vspace{4ex}


\clearpage

\null

\vfill

\normalfont\small

\begin{center}
Der Text \aa{}Die Heidin\ae{} in diesem Buch folgt
grundsätzlich zeichengetreu der Ausgabe\dopp{} \mbox{Max Mohr.} \mbox{Die Heidin.}
\mbox{München, Verlag Georg Müller. 1929.}
Abweichungen sind im Anhang vermerkt.
\end{center}

\begin{center}
@version@\\
@timestamp@
\end{center}

Dieses Buch wurde erstellt mit \LaTeX{} und \KOMAScript{}.

\end{center}

\end{titlepage}





% ************************** Roman *****************************
% Keine Kopfzeile, aber Seitennummer.
\pagestyle{plain}

\addpart{Die Heidin}

\linenumbers % Ab hier Zeilennummer pro Seite erfassen.
\setpagewiselinenumbers
\modulolinenumbers[100] % Zum Ausblenden z. B. Wert 100 einsetzen.

\addchap{Erster Teil}
\cleardoublepage
%% Erster Teil.

% Seite 7
\kapitel{Erstes Kapitel}

Ostwind, Wind aus dem Osten, ein mächtiger Hochdruck über
dem ganzen Kontinent. Schon zwei Juni-Wochen hielt es an.
Die alte Schlechtwetterhexe Europa strahlte in strammer
Wolkenlosigkeit. Sie tat, als wäre sie jung und ge\-sund wie
am fünften Schöpfungstag, als wäre sie unberührt von dem
Geschöpf des sechsten Schöpfungstages, unvergiftet von den
Dünsten seiner Großhirnrinde, unverseucht von den Ba\-%
zillen seiner Gier. Als wäre nichts geschehn, so tat die Dame
Europa, und rollte frisch und naiv dahin, mit reinem Atem,
auf ihrem kleinen Platz auf der großen Kugel.

Aber was für eine Figur sie hat, die gute Alte, auch beim
besten Wetter\ausr{} Ohne Grazie spreizt sie ihre drei Beine ins
Mittelmeer, doch sie sind schmutzig geblieben trotz des klassi\-%
schen Dauerbades. Plump ist ihr Leib und man kann nicht
sehn, was Bauch ist und was Brust ist. Einst war Frankreich
ihr süßer Bauch, Deutschland ihr bitteres Herz, Rußland ihr
geheimnisvoller großer Arsch, zusammengewachsen mit Asien.
Jetzt sind die Linien verwischt, viele politische Fettfalten
laufen wirr über das welke Fleisch, überallhin und nirgend\-%
wohin, du kannst nicht mehr erkennen, wo diese Dame atmet
und wo sie verdaut. Mit ihrem skandinavischen Fangarm
klammert sie sich ans Eis des Nordpols, mit ihrem englischen
Fangarm verkrallt sie sich im Ozean\dopp{} wo aber ist ihr Kopf,
ihre Augen, zu schauen, und ihre Ohren, zu hören\frag{} Wer weiß%
\eingriff{S7-1}{weiß, vielleicht ] weiß vielleicht},
% Seite 8
vielleicht ins Meer gestürzt, seit Äonen, untergegangen,
dahin.

Gut erhalten ist ihr Schamhügel, die Alpen, wenigstens dort,
wo noch keine Fremdenzentralen angesiedelt sind. Dort gibt
es noch Landschaften, darin das Fleisch der alten europäischen
Natur noch lebt. Dort lagert noch in versteckten Seitentälern
jene körperliche Urhülle der Natur, darin der Mensch noch
atmen kann ohne das Asthma des Verstandes, atmen kann
ohne das Asthma der Ideen und Gegenideen. Es ist der Rest
jener lebendigen Urhülle der Natur, die einst den ganzen
Ball umspannte und in bewahrte vor der Nacktheit im All.
In versteckten Seitentälern, über dem Boden der Erde,
hoffnungsvoll dahinkeimend, ein geheimnisvoll-heidnischer
Rest jener Urhülle der Natur, lebendiger Raum, dort ist es
noch zu spüren. So auch im Tal der Glonn.

Die Glonn liegt am Nordhang der bayrischen Alpen, in einem
kleinen südlichen Seitental des Tegernseer Tals, zwei Weg\-%
stunden vom See, zwei Wegstunden von der nächsten Post\-station
und Fremdenstation. Die Glonn besteht aus drei
Höfen\dopp{} Zum Krallen, Zum Sennen, Zum Lori. Beim Krallen
und beim Sennen wohnen eingesessene Kleinbauern, sechs
bis zehn Milchkühe stark, der Lori ist seit fünfundzwanzig
Jahren in den Besitz norddeutscher Städter übergegangen.
Er hat seinen alten Hausnamen allmählich verloren und ist
jetzt fast nur noch unter dem Namen seiner städtischen Be\-%
wohner bekannt, als Hersehof. Er liegt ein wenig abseits vom
Krallen und vom Sennen, am tiefsten bergwärts in der
Glonn. Nach Süden ist dann diese kleine Landschaft zu Ende,
ein paar steile Bergwiesen noch, danach Wald und Latschen
und Geröll und Fels, meist im Schnee oder im Reif oder im
% Seite 9
Nebel, wenn nicht gerade ein kontinentaler Hochdruck auch bis
hierher gedrungen ist.

Dies ist Lea Herses Heimat. Oder, wie die Schulkinder ihre
ersten Liebesbriefe adressieren\dopp{} Lea Herse, Hersehof, Glonn,
Alpen, Europa, Erde, Welt.
\abstand{}
Sie trat aus dem Haus und pfiff ihrem Hund. In ihrer
Jackentasche kollerte ein dickes Stück Schinken, frisch herunter\-%
gesäbelt, ohne Einwickelpapier, gegen einen alten Militär\-%
revolver, der mit sechs Schüssen geladen war.

Der Wolfhund Maffa, ein Greis von vierzehn Hundejahren,
lag auf der anderen Seite des Hersehofs in der Mittagsonne.
Er überhörte mürrisch den Pfiff, der mit einem kleinen Echo
aus dem Berg zu ihm drang. Um diese Stunde pfiff man ihm
nicht, es war die Stunde der Reste-Fresserei, Maffas großer
Leidenschaft, vollzogen an den Katzenschüsseln, am Pferde\-%
bottich, am Hühnertrog. Auf die eigene Schale mit Reis und
Kalbsknochen konnte ein alter Herse-Hund verzichten, aber
die Reste der anderen Herse-Tiere mußte er haben, was es
auch war, rohe Kleie, alte Fischköpfe, verfaulte Rübenstrunke,
er hätte sich ohne dieses Zeug den ganzen Tag über so be\-%
fangen gefühlt wie ein Städter, der seine Tageszeitung nicht
gefressen hat.

Erst nach dem fünften Pfiff gab er es auf. Er stellte sich gäh\-%
nend hoch und warf einen verzweifelten Blick auf den Trog,
der noch immer von dreißig Hühnern umstellt war. Dann
schlenkerte er verdrossen zu seiner jungen Herrin und ver\-%
drossen hinter ihr her\dopp{} die Herse-Wiese entlang zum oberen
% Seite 10
Herse-Gatter, durch das obere Gatter zur Mooswiese, über
die Mooswiese hinauf zum Herse-Wald, der außerhalb der
Umzäunung gelegen war.

Kurz vor dem Waldbrand, unter dem alten Kruzifix auf der
Mooswiese, lag Terese Nüll, die frisch importierte Kusine
aus der Stadt. Lea sah den farbenprächtigen Fleck schon vom
Gatter aus. Sie ärgerte sich schon von weitem über die
knallige Hingegossenheit, mit der sich ihre Kusine in die Land\-%
schaft drapiert hatte. Es war die übertriebene Natürlichkeit,
mit der sich alle Städter der natur entgegenstreckten, ohne zu
fühlen, daß diese Aufdringlichkeit der Natur unangenehm
war\dopp{} aber daß auch Terese Nüll den Takt der Landschaft
verfehlte, war besonders schlimm.

War es nicht zum Heulen, daß diese Schwärmerin im laden\-%
blauen Leinen von morgen ab den Hersehof verwalten sollte\frag{}
Das also war die Herrin über Leas Haus und Land und
Getier für Monate, für Jahre vielleicht\frag{} Es war zum Heulen,
aber es gab kein Zurück mehr, die große Expedition stand
startbereit, nur keine Schwachheiten am Start. Es war ja
ein Glück, daß Terese Nüll gerade frei war und die Verwaltung
des Hofs auf unbestimmte Zeit übernehmen konnte.

Lea schob sich freundlich an die Hingegossene heran.

\aa{}Schöner Tag, was\frag{}\ae{}

\aa{}Märchenhaft.\ae{}

Terese wältze sich auf die Seite und blinzelte verzückt zu ihrer
Kusine empor. Sie war einunddreißig, fünf Jahre älter als
Lea. An Lebenserfahrung dünkte sie sich tausend Jahre älter
als Lea. Die wußte überhaupt nichts vom Leben außerhalb
der Glonn. Sie war durch die Zurückgezogenheit ihrer Mutter
fast nie in die Stadt oder unter die Menschen gekommen. Tante
% Seite 11
Daniela hatte sich an ihrem Kind versündigt, das stand bei
der jüngeren Generation der Familie fest. Lea galt als ein
wenig zurückgeblieben bei ihren städtischen Vettern und Basen.

Trotzdem fühlten sie alle, wenn sie zu kurzem Besuch in die
Glonn kamen, eine seltsame Neugier nach Leas Dingen,
während Lea nicht im geringsten neugierig war nach ihren
städtischen Dingen. So auch jetzt. Während Lea neben Terese
Halt machte und ruhig über sie hinweg ins Tal sah, musterte
Terese interessiert die Toilette zu ihren Häuptern.

Warum hatte sich Lea nach dem Mittagessen umgezogen,
obwohl kein Besuch zu erwarten war\frag{} Das hatte sie von ihrer
Mutter geerbt, das kannte man. Tante Danielas fünfmaliger
Umzug pro Tag, auch wenn jahrelang kein Besuch im Herse\-%
hof empfangen wurde, war in der Familie sprichwörtlich
geworden\dopp{} Stallhose, Reitkleid, Stallhose, Teekleid, Stall\-%
hose, Abendkleid. Und warum gerade dieses blendende Weiß,
zehn Tage nach dem Begräbnis der Mutter\frag{} Extra\frag{} Echt
\aa{}hersisch\ae{}\ausr{} Mißtrauisch blickte Terese auf die champagner\-%
farbenen Strümpfe und sah sie hoch oben in einer champag\-%
nerfarbenen Schlupfhose verschwinden.

\aanah{}Warum so elegant, Leachen\frag{}\ae{}

\aanah{}Wieso\frag{} Mein Tenniskleid vom vorigen Jahr.\ae{}

\aanah{}Wohin damit\frag{}\ae{}

\aa{}In den Wald, Maffa totschießen.\ae{}

\aa{}Im Ernst\frag{}\ae{}

Terese setzte sich mit einem Ruck hoch.

\aa{}Hab ichs Dir noch nicht erzählt\frag{}\ae{}, fragte Lea kühl.

Sie wußte ganz genau, daß sie es Terese noch nicht erzählt
hatte. Alle Dinge, die mit ihrer großen Expedition zusammen\-%
hingen, behielt sie bis zur letzten Minute für sich.

% Seite 12
\aa{}Der Krallenpeter hat doch schon gestern das Grab gegraben,
ganz hoch oben, bei der Grenztanne, es ist alles bereit.\ae{}

Terese starrte mit entsetzten, blauen Glotzaugen auf Maffa,
der sich neben seiner Herrin auf die Hinterhand gesetzt hatte.

\aa{}Und Du mußt ihn selbst erschießen\frag{} Steht das im Testament\frag{}\ae{}
Sie witterte überall eine Apartität Tante Danielas.

\aa{}Du bist ja verrückt\ae{}, sagte Lea. \aa{}Im Testament stehen andere Dinge,
keine solchen Kinkerlitzchen, die Du glaubst.\ae{}

\aa{}Bitte, laß es, Leachen, laß ihn mir, ich will ihn gut pflegen,
den armen Kerl.\ae{}

\aanah{}Ach was\ausr{} Daß Du ihn gut pflegen willst, glaub ich Dir. Er
würde trotzdem den ganzen Tag am Gatter sitzen und nach
mir heulen. Er ist seit seiner Geburt bei mir\punkte{} Besser so,
Maffa, was\frag{}\ae{}

Maffa wedelte freudig mit dem Schwanz, dann lugte er
wieder gespannt ins Tal. Schon verschwammen vor seinen
Augen die Linien der Landschaft, doch er konnte noch jene
Stelle dort unten ahnen, wo sich das Hühnervolk um den Trog
drängte. Er sog nervös die Luft auf, aber er bekam nichts
in die Nase. In jungen Jahren hätte er bis hierher gewittert,
ob es dort unten um Fleischreste oder um Fischreste ging, jetzt
bekam er nichts mehr in die Nase. Nur ganz schwach den Dunst
seiner nahen Herrin, aber das war keine Neuigkeit für ihn,
das war sein Lebenselement wie Luft, Wasser, Erde.

\aa{}Ja, ja\ae{}, sagte Lea, \aa{}das Grab ist ein wenig zu klein geraten,
weil so dicke Wurzeln kamen, das ist dumm, ich muß ihn zu\-%
sammenrollen\eingriff{S12-1}{zusammenrollen.\ae{} ]
zusammenrollen\ae{}.}.\ae{}

Terese hielt sich die Ohren zu, entsetzt über soviel Roheit, und
kreischte auf, als hörte sie schon den Schuß. \aa{}Schrecklich,
schrecklich\ausr{}\ae{}

% Seite 13
\aa{}Na für Dich doch nicht, Du blödes Huhn\ausr{}\ae{}

\aa{}Bitte nicht in meiner Nähe\ausr{}\ae{}

\aa{}Keine Angst\ausr{} Du merkst nichts davon. Zum Tee bin ich
zurück\punkte{} Komm, Maffa, komm\ausr{}\ae{}

Sie schritt weiter, empor zum Wald, Maffa hinterher.

Das Loch, das der Krallenpeter ausgeworfen hatte, war
wirklich zu klein. Es enttäuschte, wie jedes offene Grab ent\-%
täuscht. Ein Grab sollte keine gewöhnliche Grube sein, dachte
sie, ein Grab sollte tief hinunter reichen, tief bis zum Mittel\-%
punkt der Erde hinunter, auch wenn es nur ein ordinäres
kleines Hundegrab war. Aber das Begräbnis kam später.
Vorerst mußte sie sich ganz darauf einstellen, Maffa auf die
beste Weise zu töten. Sie hatte ihren festen Plan.

Zuerst gab sie ihm das dicke Stück Schinken aus der Jacken\-%
tasche. Sie warf es ihm in hohem Bogen zu, fang, Maffa,
fang. Aber Maffa war kein gewandter Fänger mehr, er ließ
das kostbare Stück auf den Boden klatschen, ehe er sich langsam
und erstaunt darüber hermachte. Dann wollte sie ihn ermüden,
indem sie ihm kleine Äste schleuderte, bring, Maffa, bring.
Aber Maffa war ohne Lust beim Spiel, er schleppte die in\-%
teressantesten Schleudergeschosse nur aus altem Anstand
zurück. Schließlich legte sie sich neben die frische Grube aufs
Laub und simulierte Müdigkeit und Mittagsschlaf. Vielleicht
half das, vielleicht schlief er so am schnellsten ein. Denn sie
hatte sich vorgenommen, ihn im Schlaf zu erschießen. Wenn
er ganz in sich selbst hineingerollt war, wenn die Welt bereits
von ihm gegangen war, ein kurzer Knall zwischen Schlaf und
Tod, das war das beste.

Tatsächlich wirkte ihr Trick. Er legte sich friedlich neben sie
und schien einschlafen zu wollen. Sie kratzte ihn sanft an seiner
% Seite 14
Lieblings-Kratzstelle im Kreuz. Doch es dauerte noch eine
Ewigkeit, bis er nicht mehr wedelte und nicht mehr blinzelte,
bis er ganz in sich selbst hineingerollt war, bis der Revolver
gezogen werden konnte.
\abstand{}
Das Vorspiel zu dieser kleinen Hunde-Tragödie im Herse-%
Wald lag weit zurück. Es war gespielt worden am 29. Sep\-%
tember 1901, die Szene war die Gruttenhütte im Kaiser\-%
gebirg gewesen, die Spieler Leas Mutter und Leas Vater.

Frau Daniela Herse hatte damals noch Daniela Oldenkott
gehießen und die Durchquerung des Kaisergebirgs über die
Steinerne Rinne und das Ellmauer Tor hatte damals noch
als schwierige alpine Tour gegolten. Tatsächlich war es für
ein fünfundzwanzigjähriges Ding von der Wasserkante eine
Leistung gewesen, diese Tour auf Daniela Oldenkotts Art
durchzuführen\dopp{} ohne Training, ohne Führer, allein. Aber der
Hauptspaß war gewesen, daß die Eltern Oldenkott der Schlag
getroffen hätte, hätten sie eine Ahnung gehabt, was damals
von ihrer Tochter als geheimer Abschluß einer Italienreise
selbständig unternommen worden war. Die Eltern hätte der
Schlag getroffen und den Onkel und die Tante, von denen
Daniela Oldenkott in den Schnellzug Bozen-Hamburg ver\-%
frachtet worden war, hätte ebenfalls der Schlag getroffen\dopp{}
das war der Hauptspaß jener Tour gewesen, denn es war die
Zeit der ersten Korsett-Revolution der jungen Mädchen.

Stolz und erhitzt ruhte Daniela Oldenkott um die Mittagszeit
jenes Septembertags in dem kleinen Geröllfeld unterhalb
des Ellmauertors, nachdem sie die Hauptschwierigkeit der
% Seite 15
Tour, die plattigen Wandstufen der Steinernen Rinne,
glücklich hinter sich hatte. Hier war der Blick nach Süden,
nach den Gletschern und nach den Keesen der Zentrale,
noch von den letzten Stufen des felsigen Tors versperrt. Doch
gewaltig genug stürzten rechts und links von der romantischen
Ausreißerin die Wände des Predigtstuhls und der Fleisch\-%
bank in die Tiefe. Daniela hatte das Gefühl, allein auf der
Welt zu sein, eine einsame Heldin auf totem Mond, siegreich
über die Götter, abgesondert von den Menschen -- bis plötzlich
aus dem senkrechten Gestein das Predigtstuhls ein mensch\-%
licher Ruf zu ihr drang. Das gab eine kleine Ernüchterung.
Vor den Augen eines Menschen, der in jenen gewaltigen
Abstürzen steckte, zerschmolz ihr stolzer Höhenweg zu einer
Alten-Leute-Promenade. Dennoch war sie begeistert von der
Kühnheit des Rufers, sie nahm ihr Glas und tastete das
Gemäuer zu ihrer Rechten ab.

Lange Zeit bekam sie nur Fels ins Glas, glatten Fels und
zerklüfteten Feld, nassen Fels und ausgedörrten Fels, weißen
und roten und schwarzen und gelben Fels, dann eine tiefe
Schlucht, einen senkrechten Kamin, ein Seil-Ende, das zu
einer kleinen Geröllstufe in diesem Kamin herabpendelte,
endlich einen Menschen, der neben dem Seil-Ende an der
Kaminwand lehnte. Ein junges Männergesicht, das sie plötzlich
zum Greifen nah ins Glas bekam.

Sie hätte dieses Gesicht mit dem Glas abtasten können, aber
sie scheute sich, in die übermenschliche Einsamkeit jenes Men\-%
schen einzudringen, während er ahnungslos vor sich hin döste
im heiligen Glotzen des müden Pioniers. Sie setzte schnell das
Glas ab und machte sich wieder auf den Weg.

% Seite 16
Als nach einigen Minuten der Ruf wieder kam, fühlte sie, daß
der Mensch in dem schwarzen Riesenkamin nach ihr rief. Sie
stoppte und erschrak. War es ein Hilferuf\frag{} Wie hätte sie hier
Hilfe bringen können\frag{} Sie rief ein schrilles Wie zurück, Wie --
Wie -- Wie, aber sie verstand die Antwort nicht, das Echo über\-%
schlug die Worte von allen Seiten her.

Erst als der Mensch im Kamin hinter jedes Wort eine lange
Pause setzte, um das Echo abklingen zu lassen, konnte sie ver\-%
stehn. Ob sie den Schlüssel zur Gruttenhütte hätte\frag{} Natürlich,
die Gruttenhütte war ihr Ziel für diese Nacht, sie hatte sich den
Schlüssel in der Talstation verschafft, sie schrie ein verwirrtes Ja
zurück, Ja -- Ja -- Ja hallte es von der Fleischbank wider,
und floh.

Es dämmerte schon, als sie vor der Gruttenhütte Schritte
hörte. Sie hatte die Hütte längst instand gesetzt, alle Türen
und Läden waren geöffnet, der Schmutz der letzten Besucher
war weggekehrt, sie hatte sich gewaschen, sie hatte das Hütten\-%
buch studiert, sie hatte sich schon zur Genüge an ihrer Ein\-%
samkeit erbaut, Holz war gespalten und Feuer war gemacht
und Tee war gekocht, als endlich jener Mensch anmarschiert
kam.

\aa{}Großartig\ae{}, sagte er, während er in der kleinen Küche seinen
Rucksack abwarf und sich mühsam das riesige Seil, das er in
losen Schlingen umgehängt hatte, über den Kopf zog. \aa{}Ich
hätte sonst ins Tal tippeln müssen -- ich hab nämlich keinen
Schlüssel für diese Hütte -- großartig -- Himmelherrgott\-%
sakrament, geh runter, altes Biest~--\ae{}

\aa{}Darf ich Ihnen eine Tasse Tee anbieten\ae{}, sagte Daniela und
stierte ihm, ohne es sich bewußt zu werden, ununterbrochen
ins Gesicht.

% Seite 17
\aa{}Nur her damit!\ae{} Er plumpste am Tisch nieder und trank drei
Tassen von ihrem Tee.

Er dankte ihr nicht, er beachtete sie nicht, er war noch völlig
benommen und ausgepumpt von seiner Fahrt. Er mochte
ungefähr in ihrem Alter sein. Die kastanienroten Bartstoppeln
zeigten an, daß er mindestens schon eine Woche lang nicht
mehr unter Menschen gekommen war.

Erst als es dunkelte und sie die Lampe richtete, riß er sich zu\-%
sammen und torkelte mit seinem Rucksack aus der Küche. Sie
hörte ihn lange Zeit im oberen Stockwerk herumpoltern.
Aber zum gemeinsamen Abendessen trug er ein frisches Hemd
zu seiner weiten langen abgescheuerten Manchesterhose, er
war wieder ganz bei sich, seine Ausgepumptheit war wie
weggeblasen.

Auch Danielas Müdigkeit war weggeblasen, sie horchte hinge\-%
rissen auf seine Erzählungen, während sie mit ihm am Küchen\-%
tisch saß. Die Vorräte aus den beiden Rucksäcken wurden
zusammengetan: Sardinen, Speck, Reis, gedörrte Pflaumen,
Pumpernickel, Kognak, Tee.

Der dunkle Kamin, den sie gesehn hatte, war der Bot"-zong-%
Kamin gewesen. Schon vor sechs Jahren zum erstenmal bis
zum nördlichen Vorgipfel des Predigtstuhls durchstiegen,
aber erst heute zum erstenmal im freien Aufstieg bis zum
südlichen Hauptgipfel bezwungen. Das Seil, das sie dort in
der Ecke sah, war vierzig Meter lang. Menschen, die mit einem
bezahlten Führer gingen, waren Schweine. Auch Italien\-%
reisende waren Schweine. Mit Ausnahmen natürlich, mit
großartig blonden Ausnahmen. Kletterschuhe waren eine
Selbstverständlichkeit. Es gab Alltagsmenschen und Höhen\-%
menschen. Speck und Sardinen waren der beste Touren\-%
% Seite 18
proviant. Chemie war ein sehr interessantes Studium, ein
weites Feld, die Professoren waren Schweine. Der größte
deutsche Dichter hieß Heinrich von Kleist. Älter als dreißig
Jahre sollte ein anständiger Mensch überhaupt nicht werden.
Die Hamburger waren Spießer mit Krämergeist. Mit Aus\-%
nahmen natürlich, mit großartig blonden Ausnahmen. Er
selbst hieß Anton Pasternak und stammte aus Franken. Die
Franken waren Schweine. Mit Ausnahmen natürlich, mit
kastanienroten Ausnahmen. Der größte lebende Mensch war
Fritjof Nansen. Wenn im freien alpinen Gelände Einbruch
von Nebel zu erwarten war, dann legte man sich grellrote
Markierungsblätter. Man legte sie an möglichst markanten
Stellen unter Steine, alle Zehn Meter etwa, um sich den
Rückzug zu sichern. Die schönste Oper war Aida. Die Pfote,
die er ihr zum Gutenacht reichen mußte, war wund, zerfetzt
vom Gestein, doch das war bei allen anständigen Menschen
so, denn nur Huren, männliche oder weibliche Huren, trieben
Maniküre, Gutenacht.

Gutenacht. Sie lag begeistert in ihrem Bett im kleinen Damen\-%
raum der Gruttenhütte und dachte über alle diese Dinge nach.
Eine neue Welt tat sich wunderbar vor ihr auf, aber was
männliche oder weibliche Huren war, wußte sie nicht. Durch
die heuchlerischen Formen der Jahrhundertwende, durch die
strengen Formen ihres puritanischen Elternhauses, vor allem
durch eine fast krankhafte Scheu vor dem Gekicher ihrer Freun\-%
dinnen, wenn von Liebesdingen die Rede war, war sie noch
völlig naiv geblieben. Wäre sie Katholikin gewesen, sie hätte
keine unkeuschen Gedanken beichten können. Wie ein verirrtes
Enzian im Gemüsegarten war sie das seltene und ein wenig
komische Exemplar geistiger Unbeflecktheit inmitten ihrer
% Seite 19
tuschelnden Freundinnen. Wohl ahnte sie geheime Dinge
zwischen Weib und Mann, aber sie schob ihre Neugier von sich,
wo immer sie auf das purpurne Geheimnis stieß. Manchmal
fühlte sie, daß ihre Freundinnen hinter ihrem Rücken über
sie lachten und sie für zurückgeblieben hielten in irgend einem
wichtigen Punkt des Lebens, aber sie fühlte sich bei allen
Sports und bei allen Tanzereien, durch ihren natürlichen
Witz und durch die vielen Anträge, die sie zurückweisen konnte,
allen diesen Kicher-Freundinnen und Tuschel-Freundinnen
so weit überlegen, daß sie das seltsame Geflüster hinter ihrem
Rücken schnell verdrängte, so oft sie es zu spüren bekam. So
kam es, daß sie nicht verstand, was manikürte männliche oder
weibliche Huren waren. So kam es auch, daß sie keine große
Angst empfand, als nach einer halben Stunde die Tür neben
ihrer Bettstatt leise geöffnet wurde und der herrliche fremde
Felskletterer in das dunkle Zimmer trat.

\aa{}Ich wollte nur nochmal sehn, ob Sie gut untergebracht sind
in diesem geheimnisvollen Damenraum\ae{}, sagte der Student
der Chemie Anton Pasternak.

\aa{}O ja, danke\ae{}, erwiderte Fräulein Daniela Oldenkott, die
Ausreißerin aus dem Schnellzug Bozen-Hamburg.

\aa{}Haben Sie genug Decken\frag{}\ae{}

\aa{}Danke, ja.\ae{}

\aa{}Es wird sehr kalt werden.\ae{}

\aa{}Besten Dank.\ae{}

\aa{}Ich hab nämlich noch eine Masse Wolldecken gefunden\ae{}, kam
es durch die Dunkelheit.

\aanah{}Ach was\frag{}\ae{}

\aa{}Im Führerraum. Dort liegt noch ein ganzes Lager voll. Soll
ich welche bringen\frag{}\ae{}

% Seite 20
\aa{}Nein, danke.\ae{}

Ihre Stimme versagte nun doch ein bißchen. Er sprach in so
zartem Ton, ein wenig bebend, ganz anders wie zuvor in der
Küche. Sie hörte, daß er verliebt war, und sie war entzückt da\-%
von. Aber sie erschrak, als sie ihn näher auf ihr Lager zutreten
hörte. Sie spürte plötzlich durch das Dunkel, daß er nackt war.

\aa{}Noch ein wenig plaudern schadet doch nicht\frag{} Oder\frag{}\ae{}

\aa{}Nein, nein~--\ae{}

\aa{}Ich verschwinde auf der Stelle, wenn Sie es wünschen\frag{}\ae{}

\aa{}Nein, nein, warum denn~--\ae{}

Er stand über sie gebeugt, ohne sie zu berühren, und schwieg.
Da fühlte sie, daß jetzt die Reihe an ihr war, und sie schlang
mutig ihre Arme um ihn.
\abstand{}
Erst seit dem Tod ihrer Mutter wußte Lea Herse, wer ihr
Vater war. Sie verübelte es ihrer Mutter ein wenig, daß
dieses Lügengewebe des Hersehofs nicht schon längst zerrissen
war. Sie erinnerte sich jetzt ganz deutlich, daß die Mutter in
den letzten Jahren oftmals angesetzt hatte, das Geheimnis zu
lüften. Wenn der Schnee hoch gelegen war in der Glonn und
der Gang um das Haus wie zwischen Festungswällen lief,
wenn die Geburt eines neuen Füllen oder die Abreise eines
lästigen Besuchs den Abend zwischen den zwei Frauen intimer
gemacht hatte, oftmals hatte Frau Daniela Herse zu irgend
einer Beichte angesetzt. Aber immer wieder hatte sie abge\-%
brochen. Was mochte sie abgehalten haben\frag{}

% Seite 21
Ach, sie hatte zu den Müttern gehört, die das weiche Gewebe
ihrer Babies nicht vergessen können und die noch an ihren
erwachsenen Kindern das weiche Baby-Gewebe zu verspüren
glauben, auch wenn die Babies längst härter gewebt sind als
sie selbst. Sie hatte der Unverwundbarkeit ihrer Tochter nicht
getraut. Sie hatte gewartet und gewartet, bis es zu spät
gewesen war. Genau wie sie mit ihrer Blinddarmoperation
gewartet hatte, bis es zu spät gewesen war\dopp{} schon im April
und Mai hatte sie ständig über Schmerzen geklagt, aber noch
am Tag vor Durchbruch und Tod hatte sie ihren Hengst Carlo
geritten statt zum Arzt zu gehn.

Jetzt mußte Lea alle diese wichtigen Dinge und brennenden
Dinge und doch auch für ihre Generation ein wenig lächer\-%
lichen Dinge aus dem dicken Brief zusammenlesen, den Frau
Daniela Herse in einer bedrängten Stunde aufgeschrieben
hatte, um ihn der Tochter außer dem Hersehof und einem
kleinen Betriebskapital in Pfandbriefen zu hinterlassen.

Nach ihren zwei Hochzeitstagen auf der Gruttenhütte hatte
Daniela Oldenkott weder ihren Vater noch ihre Mutter ins
Vertrauen gezogen, sondern ihre Patentante, die ältere
Schwester ihrer Mutter, Fräulein Lily Gezelle. Die war
damals in der ganzen Familie Oldenkott-Gezelle-Nüll ge\-%
fürchtet und belächelt gewesen als überspannte alte Jungfrau
und als eine der ersten Sozialistinnen inmitten einer bomben\-%
sicheren Krämerzeit. Zu ihr war Daniela Oldenkott gelaufen,
als sie sich nach ihrer Heimfahrt nach Hamburg geschwängert
fühlte. Fräulein Lily Gezelle hatte zuerst ein krasses Donner\-%
wetter losgelassen, nicht gegen Daniela, sondern gegen die Heu\-%
chelei der ganzen Familie, dann hatte sie sich mit ihrer Nichte
auf die Bahn gesetzt und war mit ihr nach München zurückge\-%
% Seite 22
dampft. Den Eltern war vorgeschwindelt worden, Daniela
müßte auf Einladung der Outsider-Tante und zu deren Gesell\-%
schaft ein paar wichtige Wagner-Opern in München hören.

Aber die Expedition zu Lohengrin und dem Fliegenden Hol\-%
länder war gescheitert. Man hatte zwar sofort Anton Paster\-%
naks Adresse im Studentenverzeichnis der Universität gefun\-%
den, aber als er beim ersten Stelldichein ziemlich fremd getan
hatte, war es um Daniela geschehn gewesen. Keine Macht der
Welt hatte sie zu einem zweiten Zusammentreffen und zu
einer klaren Aussprache mit ihrem Helden bewegen können.
Vielleicht war es nur seine Ahnungslosigkeit gewesen und seine
Scheu in der fremden Hotelhalle, vielleicht war er in dem
steifen Kragen noch der gleiche Mensch gewesen wie in der
verschabten Manchesterhose, ihr immer noch zugetan wie
zwischen Fels und Kar\dopp{} aber ihr Stolz hatte sich durch seine
zurückhaltende Art zu einer Mauer aufgeworfen, die auch
Fräulein Gezelles tagelanges Gekeif nicht mehr hatte stürmen
können. Nein, sie wollte ihn nicht mehr sehn, sie wollte sich
ihm nicht aufdrängen, sie wollte sich nicht an ihn hängen,
das war ihr letztes Wort gewesen.

Und sie hatte ihn nicht wieder gesehn. Erst als Fräulein
Gezelle der verkrampften Stolz ihrer Nichte als Tatsache
hingenommen hatte, hatte sich ihr glänzendes Outsidertum
bewähren können und ihr glänzendes Organisationstalent,
darauf Daniela vertraut hatte. Noch in München war ein
Windhund von verabschiedetem Kavallerie-Offizier auf"-ge\-%
gabelt worden, Herr Karl Herse, der kurz darauf Daniela
Oldenkott in Brüssel geheiratet hatte, um sich sofort nach der
Trauung wieder von ihr scheiden zu lassen. Die dicke Ab\-%
findungssumme hatte Fräulein Gezelle durch telegraphischen
% Seite 23
Verkauf einiger ihrer Seefahrts-Aktien beschafft. Sämtliche
Mitglieder der Familie Oldenkott-Gezelle-Nüll hatten aufs
tiefste Danielas Ehe-Skandal beklagt, doch waren ahnungs\-%
los geblieben. Auch Anton Pasternak war ahnungslos ge\-%
blieben und hatte nie mehr ein Wort von seiner großartig
blonden Dame aus dem Damenraum der Gruttenhütte
gehört.

Das Kind, das dann im Hersehof von der geschiedenen Frau
Daniela Herse geboren wurde, hatte einen Vater und einen
Vaternamen gehabt. Fräulein Gezelle war zur Geburt dieses
neuesten Patenbabys angefahren gekommen, um ihm noch
en paar Monate lang mit dieser Freude das Ärschchen einzu\-%
pudern, kurz danach war sie gestorben. Herr Herse war im
Weltkrieg gefallen, ohne Frau Herse noch einmal gestört zu
haben. So hatte sie ihre Lüge durchgehalten und ihren Stolz
gerettet. Tausend miserable Tricks hatte sie dazu gebraucht,
aber tausend bunte Märchen hatte sie der Tochter von dem
toten Vater erzählen können.

Auch der Militärrevolver, den Lea in der Jackentasche trug,
um Maffa zu erschießen, war einer von diesen vielen Tricks.
Die Mutter hatte ihn stets als Andenken an den toten Vater
ausgegeben, vom Regiment der Witwe zugesandt. Es war
alles Lüge gewesen, Lüge und Stolz, ein bißchen Melan\-%
cholie und ein bißchen Lächerlichkeit und viel Lüge und viel
Stolz. Leas Vater lebte, er leitete eine große chemische Gesell\-%
schaft in Berlin, sie hatte sich schon ein Bild von ihm verschafft
und seine Züge studiert, in einer illustrierten Zeitung hatte sie
ihn gesehen\dopp{} Herrn Professor Doktor Anton Pasternak, einen
der leitenden Köpfe der chemischen Industrie, im Kreis seiner
Familie\punkte{}

% Seite 24
Maffa war eingeschlafen, es war Zeit. Der Hersehof war in
Verwaltung gegeben, die geerbten Pfandbriefe waren flüssig
gemacht und reichten aus, die Expedition in die Stadt des
Vaters auf lange Sicht zu finanzieren -- auch Maffa mußte
daran glauben, was war dabei\ausr{} Sie zog leise den Revolver
und entsicherte ihn mit einem dünnen Knack. War es nicht die
einfachste Sache der Welt\frag{} War es nicht lächerlich, schon bei
den ersten Vorbereitungen für die große Fahrt schwach zu
werden\frag{} Sie drückte ab.

Sie drückte ab, doch es war ein grausamer Fehler gewesen,
daß sie nicht zuvor den Krallenpeter um Rat gefragt hatte,
wohin der Schuß gehn mußte. Nachdem sie es verworfen
hatte, Maffa mit Blausäure zu vergiften, hätte sie wenigstens
wissen müssen, daß ein blitzschneller Tod nur durch einen guten
Schuß hinters Ohr zu erwarten war. Aber sie schoß nach eigener
Berechnung, sie schoß von oben her durch die Stirn. Der alte
Hund sprang hoch und stand vor ihr, er spreizte zitternd die
Vorderhand, aus seiner Nase floß Blut, er starrte mit großen
Augen auf seine Herrin. \aa{}Maffa, Maffa, mein Maffa, mein
einziger Maffa\ae{}, schrie sie entsetzt und war die Waffe weit
von sich, aber die heilige Sekunde zwischen Schlaf und Tod
war schon zerschossen. Maffa blieb zitternd stehn, er begann
zu röcheln, er sah sie mit vollem Bewußtsein an. Sie sank
dicht vor ihm auf den Waldboden und wimmerte ihm hilf\/los
entgegen. \aa{}Maffa, Maffa, Maffa, Maffa\ae{}. Bis er endlich mit
der Hinterhand einknickte und sich langsam umlegte und die
vier alten Beine zu strecken begann.
\abstand{}
% Seite 25
Terese Nüll wartete seit dem Schuß, der durch die stille Glonn
gedröhnt war, mit Ungeduld auf die Rückkehr ihrer Kusine.
Sie hatte sich bereits eine Menge Trostworte zurechtgelegt.
Da sie schon zwei auseinandergegangene Verlobungen er\-%
litten hatte, hielt sie sich für eine Autorität in tragischen
Dingen. Sie freute sich schon darauf, ihrer süßen kleinen
Kusine mit literarischer Überlegenheit zeigen zu können,
wie man solch ein trauriges Ding dreht.

Nach zwei Stunden wurde es ihr zu dumm. Sie lief in den
Hof hinab, Tee zu trinken. Nana servierte den Tee mit bösem
Gesicht. Nana war schon seit Leas Geburt im Hersehof
bedienstet, man konnte nicht unterscheiden, ob ihr böses Ge\-%
sicht noch zur allgemeinen Trauer über Leas Abreise gehörte
oder ob es schon auf den speziellen Fall, daß der Tee ohne
Lea genommen wurde, zugeschnitten war. Jedenfalls fühlte
Terese Nüll Unbehaben, als sie allein vor dem Tee saß, sie
bereute ihre Treulosigkeit und packte Tee und Toast auf Tante
Danielas japanisches Lackbrett, um zur Mooswiese zurück\-%
zupilgern. Kaum war das Picknick auf einer hübschen bunten
Decke unter dem Christus aufgestellt, da kam Lea aus dem
Wald herabgeschlenkert.

Als wäre nichts geschehn, so schlenkerte sie daher. Nach Maffas
Begräbnis war sie bergwärts gelaufen, bis zu der kleinen
Quelle in dem kühlen Graben, wo die Glonn entsprang.
Dort hatte sie sich ausgeheult und dann die Tränen abge\-%
waschen. Jetzt war nichts mehr zu sehn. Die Lider waren
abgeschwollen. Nur in den blauen Augensternen zündelte ein
kleinen Flämmchen, ein feindseliges schlimmes kleines Flämm\-%
chen, das Terese Nüll warnte vor allzu eifrigem Zuspruch.

\aa{}Na\frag{}\ae{} Sie plumpste munter ins Moos. \aa{}Hast Du Dich unter\-%
% Seite 26
dessen gut mit Jesus Christus unterhalten\frag{}\ae{} Sie zog einen
weißen Kamm aus der Jackentasche und kämmte sich. Ihr
Haar war ziemlich in Unordnung geraten. Als ein"-ziges Mit\-%
glied der Familie Oldenkott-Gezelle-Nüll hatte sie kastanien\-%
braunes Haar mit rötlichem Schimmer, die eisblauen Augen
der Oldenkotts, doch fremdes kastanienbraunes Haar.

Terese goß den Tee ein, ohne aufzublicken.

\aa{}Hast Du alles so schön drapiert\frag{}\ae{}, sagte Lea und strich sich
durchs Haar, daß es nur so knisterte.

\aa{}Noch ganz heiß%
\eingriff{S26-1}{heiß\ae{}, sagte ] heiß,\ae{} sagte}\ae{},
sagte Terese, \aa{}glänzend erraten.\ae{}

\aa{}Sehr malerisch drapiert\ausr{} Das ist die Hauptsache.\ae{}

\aa{}Zucker\frag{}\ae{}

\aa{}Nein, danke.\ae{}

\aa{}Zitrone\frag{}\ae{}

\aa{}Danke.\ae{}

\aanah{}Trinkst Du keinen Tee\frag{}\ae{}

\aa{}Später vielleicht.\ae{}

\aanah{}War es schlimm\frag{}\ae{}

\aanah{}Vorbei ist vorbei.\ae{}

\aanah{}Aber ein schönes Brot darf ich meinem Leachen streichen\frag{}\ae{}

\aa{}Sag doch nicht immer Leachen\ausr{} Sowas Blödes\ausr{} Leachen\ausr{}
Du sagst Leachen und Deine Mutter sagt Lealein und Onkel
Gezelle sagt Leamaus, das ist ja zum Verzweifeln\ausr{}\ae{}

\aa{}Das ist alles nur Liebe.\ae{}

\aa{}Liebe\ausr{} Hühnerdreck, die ganze Familie\ausr{}\ae{}

\aa{}Na Lea, benimm Dich doch\ae{}, schrie Terese Nüll.

Lea lachte.

\aanah{}Alle Welt ist außer sich über die Sprache, die Du von Deinen
Stallmägden und Holzknechten annimmst\ausr{}\ae{}

\aa{}Ja\frag{}\ae{}

% Seite 27
\aanah{}Wirklich, Du mußt Dir diese Sprache abgewöhnen\ausr{} In der
Stadt machst Du Dich unmöglich damit\ausr{} Und wie Du von
unsrer Familie sprichst ist direkt zum Heulen, wirklich zum
Heulen.\ae{}

\aanah{}\fremdsprachlich{Ne pleure pas, ma pauvre Térèse}\ae{}, sagte Lea mit
dem klingenden Ouchy-Akzent, den sie von ihrer Mutter über\-%
nommen hatte. \aanah{}\fremdsprachlich{Ne pleure pas, si je parle comme ça de
ma famille. Moi, je parle au figuré, dans d'absolu, c'est la lan\-%
gue de Glonn, chacun suit sa ligne.}\ae{}

Terese wußte keine Antwort, da sie nicht mitgekommen war.
Außerdem war doch alles verkehrt, war man diesem Bock
entgegensetzte. Sie legte sich zurück und grub ihren semmel\-%
blonden Nüllschen Schopf ins Moos. Traurig starrte sie in
den Himmel und wünschte, Lea möchte auf sie schaun und ihre
Melancholie bemerken.

\aanah{}Was ich noch sagen wollte, Du\ae{}, kam es nach einiger Zeit aus
der gefährlichen Ecke, \aa{}ich glaube, Du hast doch recht mit dem
künstlichen Dünger. Warum soll man es nicht doch einmal mit
diesem chemischen Zeug probieren\frag{}\ae{}

Terese schnellte hoch und griff freudig nach dem friedlichen
Thema und nach dem ausgekühlten Tee.

\aa{}Unser eigener Mist\ae{}, sagte Lea, \aa{}reicht höchstens für unseren
halben Wiesengrund. Es war immer eine Riesengeschichte,
fremden Naturmist aufzutreiben. Und teuer, sag ich Dir\ausr{}
Und die schwere Zufuhr\ausr{} Auf hundert Kilometer im Umkreis
sind wir die einzigen Leute, die noch niemals Kunstdünger
auf die Wiesen gelassen haben.\ae{}

\aa{}Das habe ich doch immer gesagt, mein Kind\ausr{} So was Un\-%
rationelles\ausr{} Es war eine fixe Idee von Deiner Mutter.\ae{}

\aa{}Fixe Idee\ausr{} Mutter wußte ganz genau, warum sie gegen
% Seite 28
den chemischen Dünger war\ausr{} Soviel wie die Professoren auf
Deiner landwirtschaftlichen Hochschule verstehn wir hier
auch noch vom Wiesenbau und von der Viehzucht, kannst Dich
drauf verlassen. Mußt Dir nicht einbilden, daß Du auf Deiner
Schule irgendwas gelernt hast\ausr{} Wirst ja sehn, wie das dann
in der Praxis aussieht. Wenn meine Hühner nur nicht Bauch\-%
weh kriegen bei Deiner neuen Futtermethode\ausr{} Vielleicht
legen sie Herings-Eier, bis ich zurückkomme\frag{}\ae{}

Terese lachte. Klar, daß sie nach ihrem zweijährigen Studium
besseren Bescheid wußte. Tante Daniela hatte den Hersehof
ganz stur nach den uralten Bauernregeln der Glonn weiter\-%
bewirtschaftet, Pferdezucht und Hühnerzucht, Milchwirtschaft
und Wiesenbau. Was war hier nach den neusten wissenschaft\-%
lichen Methoden herauszuholen\ausr{} Aber sie wollte das abzie\-%
hende Gewitter nicht wieder beschwören und ging freundlich
auf die Laune ihrer Kusine ein. \aanah{}Thomasmehl und Kainit
nimmt doch selbst der alte Krallenbauer. Und es gibt doch
jetzt so wunderbare neue Salze und künstliche Dungmittel.
Du glaubst wirklich Deine Mutter hatte einen bestimmten
Grund dagegen\frag{}\ae{}

\aa{}Kann ich Dir genau erklären\ae{}, sagte Lea und trank jetzt doch
eine Tasse Tee und aß mit Appetit ein paar Toasts. Sie
vergaß für eine kleine Weile Maffas Augen und setzte Terese
Nüll die Kunstdünger-Theorie ihrer Mutter auseinander.

Frau Herses Idee war gewesen, daß alle diese neuen chemi\-%
schen und maschinellen Dinge für die Menschheit und für die
Erde noch nicht praktisch erprobt waren. Zehn, dreißig, fünfzig
Jahre Erfahrung, was war das viel\frag{} Wußte man denn, ob
nicht alle Großstädter mit der Zeit geisteskrank wurden oder
sonstwie krank, ohne die Krankheit aufziehn zu fühlen\frag{}
% Seite 29
Warum zum Beispiel sollte nicht in allen Automibilisten oder
in deren Samen irgendetwas Lebenswichtiges absterben,
absterben durch die gestohlene Geschwindigkeit, die nicht von
ihnen kam und die ihnen nicht zustand\frag{} Klappten nicht schon
jetzt alle Männer in den amerikanisierten Betrieben mit
vierzig Jahren zusammen\frag{} Viele Jahrtausende hatte die
menschliche Natur gebraucht, um sich die Technik des Feuers
zu erobern, und jetzt sollte sie sich von einem Tag zum andern
Tag an ähnlich eingreifende Dinge gewöhnen\frag{} Wußte man
denn, ob nicht im Kuhmist oder im Pferdemist trotz der besten
chemischen Analyse irgendein geheimnisvolles Etwas steckte,
wohlverborgen vor der Wissenschaft, jenseits aller Chemie,
und der Boden der Erde liebte gerade dieses geheime Etwas
und wollte es schluc"-ken\frag{} Die Wiese dort unten, Jahrtausende
lang hatte sie dicken, warmen, tierischen Mist geschluckt, um
blühen zu können, und jetzt plötzlich sollte sie sich mit chemischem
Ersatz abspeisen lassen\frag{} Vielleicht ließ sie sichs fünfzig Jahre
lang gefallen, vielleicht hundert Jahre lang, weiterblühend
aus der gesammelten alten Kraft, so wie die Großstädter noch
immer Kinder zeugten aus der geborgten Kraft der Väter --
aber dann war es eines Tages aus, Streik gegen die Chemie,
Hohn gegen die Wissenschaft, war denn nicht stets im Triumph\-%
zug der Wissenschaft der Hohn auf das Gestern mitgeschritten\frag{}
Dann war es zu spät. Dann stand das chemische Gelände der
rollenden Kugel kahl. Dann konnten die letzten Großstädter
mit ihren Wochenendkarren auf einer Mondkrater-Landschaft
herumgondeln.

Das war Frau Herses Idee gewesen. Daher ihre Feindschaft
gegen den künstlichen Dünger. Lea selbst war nicht mehr so
% Seite 30
ganz dieser Ansicht. Ihre Lebensanschauung hatte in den
letzten Tagen ein wenig umgeschlagen.

\aa{}Interessant\ae{}, sagte Terese Nüll. \aa{}Sehr interessant, Sek"-tierer\-%
tum, \fremdsprachlich{retournons à la nature, }aufgewärmter Rousseau, Deine
Mutter war eine großartige Frau, bei ihr wirkten diese alt\-%
väterlichen Spleens ganz reizend. Aber wir Jungen sollten
doch ein wenig fortschrittlicher denken, Leachen, was\frag{}\ae{}

\aa{}Schon wieder Leachen\ausr{}\ae{}

\aa{}Le--ha\ausr{} Freut mich wenigstens, daß Du erstmal zum
chemischen Dünger bekehrt bist, Le--ha\ausr{}\ae{}

\aa{}Ich bin gar nicht bekehrt, Du Affe\ausr{}\ae{} Lea erhob sich und
dehnte sich mit Luft und streckte sich mit Luft. \aa{}Aber nimm
nur Kunstdünger dieses Jahr, meinetwegen, es ist wahr, ich
bin auch für die Chemie.\ae{}
\abstand{}
Auf der kleinen Landstraße im Tal war ein Radler zu sehn.
Im Tempo eines Rennfahrers kam er auf die Glonn zu. In
vollem Schwung stieß er mit seinem Vorderrad das ange\-%
lehnte untere Gatter des Hersehofs zurück, daß es nur so
knallte. Vor dem Haus sprang er ab, ohne zu bremsen. Sämt\-%
liche Herse-Hühner flohen mit Gegacker bergwärts.

\aanah{}Wer ist das\ae{}, frug Terese Nüll und starrte neugierig auf den
Bravour-Fahrer im Tal. Man sah Nana aus dem Haus treten
und mit ihm sprechen.

\aa{}Quirin Linsinger\ae{}, sagte Lea. Es klang nicht besonders erfreut.

\aa{}Hast Du gesehn, wie er das Gatter aufgestoßen hat\frag{}\ae{}

\aa{}Er macht immer solche Geschichten. Wenn zugeriegelt ge\-%
wesen wär, hätt es ihn geschmissen.\ae{}

% Seite 31
\aa{}Großartig war das.\ae{}

\aa{}Er weiß ganz genau, daß der Riegel kaputt ist.\ae{}

\aa{}Das ist Quirin Linsinger\frag{}\ae{}

\aa{}Ja, das ist er.\ae{}

Der junge Mann hatte sein Rad an die Hauswand gelehnt
und einen dichten roten Blumenstrauß, der an die Lenkstange
geschnürt war, losgebunden. Nana brachte ihm ein großes
Glas Wasser aus dem Haus. Er stürzte es mit einem Zug
hinunter. Dann marschierte er mit seinem Blumenstrauß los\dopp{}
die Herse-Wiese entlang, zum oberen Herse-Gatter, zur Moos\-%
wiese.

Er trug weite graue Flanellhosen, scharf gebügelt, am Knöchel
mit Radfahrer-Klammern zugesteckt, dazu eine enge Woll\-%
weste in hübschem Beige. Die Mütze hatte er beim Rad
zurückgelassen. Man sah schon von weitem, daß sein glattes
schwarzes Haar sehr windsicher zurückpräpariert war. Auch
daß sein Gesicht außergewöhnlich stark eingebräunt war, sah
man schon von weitem.

Terese Nüll kannte ihn dem Namen nach. Er galt als der einzige
Freund des Hersehofes, als Leas Tennispartner, als der beste
Schispringer im Gau. Wenn die alten Tanten in Hamburg
über Leas weltfremde Erziehung klagten, hieß es stets\dopp{} \aa{}Ihr
einziger Verkehr sind diese wilden Linsingers, wenig anzie\-%
hende Leute.\ae{} Man hatte sich schon oft den Kopf zerbrochen,
ob nicht zwischen Lea und einem der Linsinger-Söhne eine
Liebschaft bestände. Aber man war noch zu keinem festen
Resultat gekommen. Jedenfalls wurde Terese Nüll von einem
Anfall von Diskretion überkommen, als sie den jungen Mann
so selbstsicher näherstapfen sah\dopp{} sie sprang auf und trat zu Lea,
die an dem morschen Kruzifix-Balken lehnte und ihrem
% Seite 32
Freund ein kleines kühles Halloh entgegenrief, als er vom
oberen Gatter aus winkte.

\aa{}Du, ich gehe, Lea, ich drücke mich.\ae{}

\aa{}Unsinn, Du bleibst\ausr{} Weiß der Teufel, woher er meine Abreise
erfahren hat\ausr{}\ae{}

\aa{}Er will sich verabschieden, er will Dich allein sprechen, ich
verschwinde.\ae{}

\aa{}Du bist ja verrückt, Du bleibst, Du mußt ihn kennenlernen,
ich hab wirklich kein Geheimnis mit ihm.\ae{}

Die Linsingers waren Bayern. Sie betrieben eine kleine
rentable Silberfuchszucht, eine halbe Stunde westlich von
der Glonn. Der Verkehr zwischen den alten Linsingers und
Frau Daniela Herse war ursprünglich durch den Austausch
von Bruteiern entstanden. Doch Frau Herse war stets sehr
zurückhaltend geblieben, sie haßte jeden gesellschaftlichen
Verkehr wie die Pest. Noch kurz vor ihrem Tod konnte sie
sechs Stunden im Sattel durchhalten und acht Stunden auf
Schiern, aber an einem runden Teetisch mit durcheinander\-%
quatschenden Damen und Herren klappte sie prompt nach
zehn Minuten zusammen. Außerdem vertrug sie nicht den
starken Fuchsgeruch, der die ganze Linsinger-Farm durch\-%
schwelte. Erst als die beiden alten Linsingers schnell hinter\-%
einander gestorben waren und die Geschwister Linsinger,
zwei Söhne und zwei Töchter, Leas ausgesprochene Sports\-%
kameraden geworden waren, hatte der Verkehr zwischen den
zwei Höfen freundschaftliche Formen angenommen.

Daß Lea kein Geheimnis mit Quirin Linsinger hätte, war eine
krasse Lüge. Er hatte sie gehabt. Aber vielleicht war es auch
keine Lüge, sie fühlte kein Geheimnis mehr mit ihm, sie war
ihm längst entglitten, sie hatte sein Manntum längst wieder
% Seite 33
vergessen. Sie sah jetzt nur noch den guten lieben Sportsbubi
in ihm, den sie ein wenig bemuttern mußte, obwohl er fünf
Jahre älter war als sie.

Sie hatte sich in ihn verliebt, als er vor vier Jahren nach
seinem Londoner Winter ins Tal zurückgekehrt war, um seine
Erbfarm zu übernehmen und seinen jüngeren Bruder ins
Ausland zu schicken. Sie hatte sich heimlich mit ihm verlobt,
als im März jenes Jahres die Füchse ranzten und als sie ihn
eines Tages inmitten seiner ranzenden Füchse besucht hatte,
auf dem kleinen Holzturm über dem Auslauf der Fehen und
Rüden, wo er während der Ranzzeit tagelang ausharren
mußte, um zu kontrollieren, welche Fehen belegt wurden
und welche Rüden taugten. Und er hatte sie gehabt im darauf\-%
folgenden Sommer, als der junge Wurf jener Ranzzeit be\-%
reits entwurmt war und als er an einem heißen Heu-Tag
angefahren gekommen war, um ihr beim Einbringen ihrer
entlegensten Bergwiese zu helfen. Ein paar Heu-Wochen
lang hatte er sie gehabt, dann war sie ihm wieder entglitten.
Sie gehörte ihm längst nicht mehr, sie hatte ihm nie gehört.
Wie zuvor trainierte sie auf dem Linsingerschen Platz Tennis
mit ihm, wie zuvor half sie ihm im Winter die Sprungschanze
einzustampfen, aber nie mehr hatte er gewagt, sich ihr mit
einem einzigen intimen Wort oder Blick zu nähern, seitdem
sie damals ganz ohne äußeren Grund plötzlich Schluß gemacht
hatte. Sie spielte nicht mit ihm, sie hatte sein Manntum
wirklich vergessen, das fühlte er und das nahm ihm jede
Hoffnung auf ein Wiedergewinnen des verlornen Spiels. So
war er ihr erst richtig verfallen, seitdem er wußte, daß es
ihm nicht gelungen war, sie wahrhaftig zu besitzen und zu
erwecken.

% Seite 34
\aanah{}Almrausch\ausr{}\ae{} Sie war ehrlich begeistert von dem frischen
Strauß, den er ihr entgegenhielt. \aa{}Gibt's schon Almrausch\frag{}
Wo haben Sie ihn gefunden\frag{}\ae{}

\aanah{}Am Plankenstein. Dort blüht er immer zuerst.\ae{}

\aa{}Das ist Herr Linsinger, das ist meine Kusine aus Hamburg --
na so was, gibt's schon Almrausch, besten Dank~--\ae{}

\aa{}Entzückend\ae{}, sagte Terese Nüll. \aa{}Ich dachte, Alpenrosen
blühn erst im Spätsommer\frag{}\ae{}

\aa{}Das sind doch keine Alpenrosen, Du Kalb, das ist Almrausch,
ganz was anderes.\ae{}

\aa{}Das wird immer verwechselt, gnädiges Fräulein\ae{}, erklärte
Quirin Linsinger verbindlich, \aa{}der Almrausch blüht zwei
Monate vor der Alpenrose und ist kleiner und buschiger~--\ae{}

\aa{}Und tausendmal würziger, ganz was anderes -- trinken Sie
eine Tasse Tee, Quirin, er ist zwar schon kalt~--\ae{}

Sie lagerten sich wieder um das Teegeschirr und sprachen
vom Wetter, vom Heu, von den Füchsen. Leas Abreise wurde
weder von ihm noch von ihr mit einem Wort erwähnt. So
sprach auch Terese Nüll nicht davon. Aber sie wurde in ihrem
Verdacht bestärkt, daß die beiden Eingeborenen allein sein
wollten. Nach zehn Minuten lief sie zum Hersehof hinunter,
um Nana und dem Krallenpeter beim Abendstall zu helfen.

\aanah{}Was ist, Lea\frag{} Sie verreisen\frag{}\ae{}

\aanah{}Woher wissen Sie\frag{}\ae{}

\aanah{}Auf der Post haben die Leute gesagt, Sie haben den Herse\-%
hof verkauft und Sie treten eine Weltreise an, weil Sie eine
Million geerbt haben.\ae{}

\aa{}So ein Zimt\ausr{} Neunzehntausend Mark hab ich geerbt, wenn
Sie es ganz genau wissen wollen. So eine Affenbande\ausr{} Auf
% Seite 35
ein paar Monate nach Berlin fahre ich und meine Kusine
verwaltet inzwischen den Hof, das ist alles.\ae{}

\aanah{}Was wollen Sie denn in Berlin, Lea\frag{}\ae{}

\aa{}Luftveränderung. Der Arzt sagt, ich muß nach diesen letzten
Woche meine Nerven wieder aufpäppeln.\ae{}

\aa{}Sie und Nerven\frag{} Und ausgerechnet in Berlin\frag{}\ae{}

\aa{}Selbstverständlich\ausr{} Das Ei des Columbus, daß der Arzt mich
nach Berlin schickt\ausr{} Die Großstädter gehn aufs Land, wenn
sie zusammenknaxen, wir müssen in die Stadt, wenn wir
zusammenknaxen. Das ist doch klar\ausr{}\ae{}

Gegen diese Lüge vom ärztlichen Befehl gab es keinen Ein\-%
wand, das wußte sie. Von ihrer Mutter hatte sie beizeiten
gelernt, daß man den Menschen stets mit dem Ausspruch
einer fremden Autorität kommen mußte, wollte man nicht,
daß sie einem die eigene Willensfreiheit wie die Ratten be\-%
knabberten.

\aanah{}Außerdem kann ich in Berlin doch endlich mal was für
meine arme Bildung tun, Quirinchen. Stellen Sie \eingriff{S35-1}
{Sie Sich ] Sie sich}Sich vor,
wie fein ich dort aufgebügelt werde. Alle meine Hamburger
Tanten und Kusinen stehn sowieso schon auf dem Kopf und
wackeln mit den X-Beinen, weil ich so ungebildet bin. Bei
Tante Nüll sind schon beide Strumpfhalter abgerissen, so
stark wackelt sie über mich.\ae{}

Aber Quirin Linsinger war heute nicht für schlechte Witze
empfänglich. Als er von ihrer Abreise gehört hatte, hatte er
sich vorgenommen, noch einmal aufs Ganze zu gehn, und ihr
einen neuen Antrag zu machen. Wie alle Dreißigjährigen
seiner Generation pendelte er hilflos zwischen Minderwertig\-%
keitsgefühlen hin und her, zwischen Kleinheitswahn und
Größenwahn, zwischen Melancholie und Frechheit. Wie alle
% Seite 36
jungen Männer seiner Zeit konnte er sein wahres Manntum
und seine männliche Form nicht finden. So griff er, nachdem
er eine Zeitlang über die Großstadt geschimpft hatte, ver\-%
zweifelt nach dem alten Spruch aus der Großväterzeit und
sagte plötzlich\dopp{} \aa{}Darf ich um Ihre Hand bitten, Lea\frag{}\ae{}

\aa{}Bitte\ae{}, sagte Lea und streckte ihm über das Teegeschirr hinüber
die Hand hin.

\aa{}Machen Sie keinen Unsinn, Lea. Sie wissen, was ich
meine.\ae{}

\aa{}Sind Sie doch nicht kindisch, Quirin.\ae{}

\aanah{}Also nein\frag{}\ae{}

\aanah{}Was für eine Frage\ausr{} Nein\ausr{} Das wissen Sie selbst ganz
genau.\ae{}

\aa{}Sie sind ein verdammtes Luder, Lea.\ae{}

\aanah{}Warum denn\frag{} Wieso denn\frag{}\ae{}

Er schwieg.

\aanah{}Warum bin ich denn ein Luder\frag{}\ae{}

Er verweigerte die Auskunft.

\aanah{}Wissen Sie, was ich am meisten in meinem Leben bereue\ae{},
frug er nach einiger Zeit. Er hatte den Ton gewechselt. Man
konnte deutlich hören, daß die männliche Unterwürfigkeit jetzt
der männlichen Frechheit Platz machte.

Sie war auf dem qui-vive. \aa{}Na, was denn\frag{} Was bereuen
Sie denn am meisten, Quirin\frag{}\ae{}

\aa{}Daß ich Ihnen damals kein außereheliches Kind gemacht
habe, meine kleine Maus.\ae{}

\aa{}Dummes Schwein\ausr{}\ae{}

Sie dachte an das Schicksal ihrer Mutter\dopp{} süß und zart hatte
die stolze Daniela Oldenkott die Blume ihres Leibes über\-%
wintert, bis sie im Hersehof aufgeblüht war. Er klammerte
% Seite 37
sich an den Gedanken, daß es auf der Welt noch andere Frauen
gab\dopp{} die Welt war voll von Weibern und alle trugen die gleiche
Unterwäsche und waren gleich gebaut. So schwiegen sie eine
Zeitlang, Weib und Mann, kleine Maus und dummes
Schwein.

\aanah{}Also Verzeihung, gnädiges Fräulein, und gute Reise\ausr{}\ae{}

\aa{}Gut, Schluß\ausr{}\ae{}

\aanah{}Vielleicht gibt es in Berlin nettere junge Männer als hier\frag{}\ae{}

\aa{}Leicht möglich. Ein Mann ist gewiß dort, der mir gefällt,
doch das ist ein alter Mann~--\ae{}

Da sie aber mit diesem Spruch schon an ihr Geheimnis rührte,
brach sie schnell ab und schwätzte darüber hinweg. Daß sie in
Berlin Theater sehn wollte, Tanzen lernen, sich die Lippen
anschmieren, daß es nur so knallte. Und wenn gerade die Mode
aufkäme, daß man Halsketten aus getrocknetem Kuhmist trüge,
dann müßte sie eben auf Befehl ihrer alten Wackeltanten eine
Halskette aus getrocknetem Kuhmist umlegen. Sie kälberte,
bis das Böse vergessen war. Nein, sie wollte sich nicht im Bösen
von Quirin verabschieden, sie wollte keine Rückwärts-Ge\-%
danken mit sich schleppen, keine guten und keine bösen Ge\-%
danken, keine guten Gedanken an Nana und an Maffa, keine
bösen Gedanken an Quirin und an Terese, Gut und Bös war
die gleiche Fessel, sie aber mußte frei sein für die Fahrt zu
ihrem Vater, sie brauchte freien Raum um sich herum, Raum,
Raum, Raum. Zum Schluß bat sie Quirin, hie und da nach
Nana zu sehn, nach dem Krallenpeter, vor allem nach den Tie\-%
ren im Stall, sie hatte ein wenig Angst vor dem landwirt\-%
schaftlichen Eifer ihrer Kusine.

\aanah{}Wird besorgt\ae{}, sagte Quirin und stand auf. \aa{}Sonst kann ich
nichts für Sie tun\frag{}\ae{}

% Seite 38
\aa{}Doch\ae{}, sagte sie und lachte. \aa{}Schaun Sie mal, wie morsch
das Kruzifix ist\ausr{} Schmeißen Sie es zum Abschied um\ausr{}\ae{}

\aa{}Ich werd mich hüten.\ae{}

\aanah{}Warum nicht\frag{} Es ist ganz morsch.\ae{}

\aa{}Es hält noch hundert Jahre.\ae{}

\aa{}Mit einem richtigen Fußtritt werfen Sie es um, Quirin.\ae{}

\aa{}Kann sein. Ich will aber nicht.\ae{}

Sie versuchte selbst, das alte Kreuz zu stürzen. Es wackelte
und ächzte nur ein wenig, ohne einzufallen.

Quirin Linsinger lachte.

\aa{}Los, Quirin, Sie schaffen es mit Ihrer Bärenkraft\ausr{}\ae{}

\aanah{}Weshalb denn eigentlich\frag{} Es steht doch sehr schön hier\frag{}\ae{}

\aa{}Ja, finden Sie\frag{}\ae{}

Sie blickte prüfend auf den Gekreuzigten. Er hing unter einem
kleinen Regendach aus Schindelbrettern, das Kreuz war
zweieinhalb Meter hoch, ein primitiver Bauernchristus. Viele
Menschenalter hing er schon so, wie er hing. Die Farben
waren trotz des Regendachs stark verwischt.

\aa{}Überall hängt dieser Mann und verkündet, daß unser schönes
Diesseits nichts wert ist. Überall hängt dieser Mann und
verkündet den Tod. Werfen Sie ihn um, Quirin, werfen Sie
ihn um\ausr{}\ae{}

\aa{}Fällt mir nicht im Traum ein\ausr{}\ae{}

\aa{}Halten Sie es für Sünde\frag{}\ae{}

\aa{}Ach was\ausr{}\ae{}

\aa{}Sie glauben noch an ihn\frag{}\ae{}

\aa{}Nicht im geringsten.\ae{}

\aanah{}Warum wird er dann nicht umgeschmissen\frag{}\ae{}

\aanah{}Weil ich nicht will\ausr{} Weil es nur eine blöde Laune von
Ihnen ist\ausr{}\ae{}

% Seite 39
\aa{}Das ist keine Laune. Es ist schlimm, wenn man nicht mehr
an ihn glaubt und doch nicht Schluß mit ihm macht.\ae{}

\aa{}Halb so schlimm.\ae{}

\aa{}nein, sehr schlimm, das allerschlimmste. Das würde Christus
selber sagen, wenn er wiederkäme.\ae{}

\aa{}Quatsch\ausr{}\ae{}

\aa{}Ein großer Schwindel ist auf der Welt durch diese halbe Ge\-%
schichte. Meine Mutter hat Recht gehabt, daß sie zur Heidin
geworden ist, nachdem sie nicht mehr an ihn glauben konnte.
Ich will eine richtige Heidin wie meine Mutter werden.\ae{}

\aanah{}Warum hat dann Ihre Mutter ihn stehn lassen\frag{} Wa"-rum
hat sie ihn nicht schon längst weggeschafft, wenn sie eine rich\-%
tige Heidin war\frag{}\ae{}

\aanah{}Ach, sie war eine zarte Seele -- aber ich bin keine zarte Seele,
ich werfe ihn um~--\ae{}

Sie versuchte noch einmal, den Christus zu stürzen. Qui"-rin
Linsinger stand dabei und amüsierte sich über die erfolglosen
Fußtritte, die sie dem alten Kreuz versetzte. Als sie sah, daß
es umsonst war, stampfte sie mit dem Fuße auf den moosigen
Boden. Quirin Linsinger brach in Gelächter aus.

\aa{}Sie Idiot\eingriff{S39-1}{Idiot\ae{}, ] Idiot,\ae{}}%
\ae{}, schrie sie wütend, \aa{}Sie werfen ihn nicht um\frag{}\ae{}

\aa{}Nein, sag ich.\ae{}

\aa{}Dann lassen Sie ihn stehn, Idiot\ausr{}\ae{}

Sie war wenigstens noch das Regendach herunter, bevor
sie ins Tal stieg. Mit ein paar hohen Luftsprüngen stieß sie
die alten Bretter über den Gekreuzigten herunter. Was
brauchte der Gott des Todes ein Regendach\ausr{}
\abstand{}
% Seite 40
Am nächsten Morgen fuhr sie. Sie wollte allein loskutschie\-%
ren, so mußte der Krallenpeter schon drei Stunden vor ihrer
Abfahrt zur Bahnstation tippeln, um dort den kleinen Land\-%
wagen mit der dicken Stute wieder in Empfang zu nehmen.
Es gab ein großes Gewinke aus allen Häusern der Glonn, als
sie dann endlich abfuhr. Nana hatte ein zuckendes Heul-%
Mäulchen, Terese Nüll mache innige Familien-Glotzaugen,
der alte Krallenpeter humpelte aus dem Krallenhof und
krächzte immerzu\dopp{} \aa{}Bald wiedakumma, bald wiedakumma,
bald wiedakumma\ausr{}\ae{}, und die drei kleinen Buben vom Sen\-%
nenhof rannten noch hundert Meter neben dem Wagen her.
Dann sank die Glonn zurück, die Menschen zuerst, danach die
Häuser, danach der lichtgrüne Wald, wo auf Maffas kleinem
Hügel zum erstenmal der Tau lag. Die alte Landschaft glitt
zurück, die dicke Stute fiel in vollen Trab und begann mit
Gefühl zu furzen und zu misten, die neue Landschaft tat sich
auf.

In München langweilte sie sich durch einen toten Nach\-%
mittag, weil vor Nacht kein Zug nach Berlin ging. Sie be\-%
sah die Läden und die Menschen in den Straßen, sie trank
viermal Kaffee zwischen grauen alten Damen und grünen
jungen Zigarettenrauchern, graugrün zog sich ihr erster Reise\-%
tag zu Ende, bis endlich der Schlafwagenzug abgelassen
wurde, der Stadt ihres Vaters entgegen, der Stadt ihres
Vaters entgegen, der Stadt ihres Vaters entgegen.

Zum erstenmal in ihrem Leben fuhr sie Schlafwagen. Da sie
mit einer jener alten Damen aus der graugrünen Welt zu\-%
sammenschlafen sollte, löste sie beim Schaffner einen Zuschlag
für ein freies Abteil erster Klasse. Das ging wider das Pro\-%
gramm, aber als sie dann in ihrem eigenen Abteil lag und
% Seite 41
die Kirschkerne auf die Reiselektüre spuckte, konnte sie wenig\-%
stens wieder aus der graugrünen städternen Welt in die licht\-%
grüne Glonner Welt zurückfinden.

Es war sehr wohlig in dem kleinen Messingkäfig. Nirgendwo
stand angeschrieben, daß man die Bettdecke nicht bis ans
Fußende zurückstoßen durfte, wenn es zu heiß war. Es störte
auch keinen Menschen, wenn man zum allgemeinen Getöse
der Fahrt den bekannten gregorianischen Kirchenchor für
Eisenbahnfahrten anstimmte, Korinther zwo dreiviertel ein\-%
halb\dopp{}

\begin{verse}Ein Mädchen klein und weiß wie Schnee,\\*
So einst spazieren ging am Tegernsee,\\*
Da fand es einen Kieselstein,\\*
Da fiel ihm der Gedanke ein\dopp{}\\*
Ach -- wärst -- Du -- doch -- Herr Jenkerlein,\\*
Herr Ging -- Gang -- Genkerlein\\*
Vom -- Te -- gern -- see --
\end{verse}

\noindent{}Hundertmal konnte man das in den Takt der Räder hinein\-%
plärren, ohne verhaftet zu werden. Dagegen war es verboten,
im Nachthemd auf den Gang zu laufen und an alle Türen
zu pumpern. Auch die geheimnisvolle Nachbartür durfte
man nicht aufreißen, um zu rufen\dopp{} \aa{}Her mit den Perlen, Frau
Bankier, oder ich schieße\ausr{}\ae{} Aber warum auch\frag{} Es waren
lauter reizende Menschen in den Messingkäfigen ringsum.
Nur die zwei Herren in den übertrieben schnittigen Sports\-%
hosen hatten ziemlich frech gegrinst. Vielleicht waren es zwei
Chemiker, angestellt im Werk Pasternak\frag{} So konnte sie morgen
früh zu ihnen treten und sprechen\dopp{} \aa{}Meine Herren, grinsen
können Sie, aber daß ich die Tochter Ihres Chefs bin, ist
% Seite 42
Ihnen offenbar unbekannt. Sie sind ja sehr nette Chemiker,
meine Herren, aber Ihre Hosen sind ordinär, wenn Sie das
noch nicht wissen sollten. Ich will Sie nicht noch einmal in
diesen Hosen sehn, meine Herren, ich danke Ihnen, meine
Herren.\ae{} Was taten die schneidigen Chemiker dann\frag{} Sie
zogen natürlich sofort die Hosen aus. Lieber nicht, gluckste
sie lachend und war schon halb hinüber. Lieber nicht, meine
Herren, die Hosen sollen dran bleiben, alle Herren in Europa
behalten bis auf weiteres die Hosen dran, Befehl von Lea
Pasternak, wer in den nächsten drei Jahren noch einmal seine
Hose abtut, wird schwer bestraft, rattapatta-rattapatta, aber
richtig bestraft, rattapatta-rattapatta-rattapatta.

Sie erwachte mit einem kleinen Alp. Es war nicht der zentner\-%
schwere Druck des erzenen Nichts, mit dem sie in den ersten
Tagen nach dem Tod der Mutter aufgewacht war\semi{} es war
nur der kleine Morgenalp, zwanzig Pfund schwer auf der
Brust. Sie sprang aus dem Bett und stieß den Rollvorhang
am Fenster hoch.

Bösartig und schmutzig sah der Messingkäfig im ersten Mor\-%
gengrauen aus. Durch eine unerbittliche Ebene jagte der Zug.
Die vorbeisausenden Häuser verkündeten die ungelüftete
Melancholie einer hoffnungslosen Generation.

Aber nach dem Gepantsch am Wasserpatent wurde es besser.
Und nach dem dicken Frühstück wurde es noch besser. Sie ließ
die Fensterscheiben herab und begann mit Luft de fremde
Luft der Ebene zu atmen. Auch in Berlin war Ostwind, Wind
aus dem Osten, ein mächtiger Hochdruck auch hier.

Sie war in der Stadt ihres Vaters. Sie nahm einen char\-%
manten alten Verlaß-Dich-drauf-Chauffeur und fuhr ins
Hotel.
%% Erster Teil.

% Seite 43
\kapitel{Zweites Kapitel}

Der liebe Gott, nähme er wieder einmal Menschengestalt an,
um unter seinen Erdenkindern zu wandeln, und versetzte er
sich zu diesem Zweck plötzlich in ein Hotelzimmer im Westen
irgend einer Großstadt, auch der liebe Gott hätte hier nur
zwischen zwei Dingen die Wahl\dopp{} entweder auf die Straße
rennen, um sich mit irgendeiner Predigt lächerlich zu machen,
oder telefonieren. Lea Herse telefonierte.

Nach dem Auspacken der Koffer war sie allmählich in den
kleinen Champagnerschwips der Ankunft hineingeschliddert.
Der Chauffeur hatte vor Freude über ihr Trinkgeld fast geheult
und der Portier hatte sie begrüßt wie ein Ehrenmitglied des
Geheimbundes \aa{}Ging-Gang-Genkerlein and Company\ae{}. Die
Liftjungen waren wie die Ziegenböckchen vom letzten April"-%
wurf um sie herumgesprungen und die Zimmerzofe, als sie
versicherte\dopp{} \aa{}Die Toilette ist im Gang schräg vis-a-vis ganz
hinten links um die Ecke\ae{}, hatte deutlich gelächelt\dopp{} \aa{}Das
kommt für Dich Reh aus der Glonn natürlich gar nicht in
Frage\ausr{}\ae{} Jetzt trug sie ihr Lichtgrünes und telefonierte. Des
Jahrhunderts dicke Bibel lag aufgeschlagn von Par bis Paw
vor ihr, die Nummer war schnell gefunden.

Zentrale Hotel, Amt, Zentrale Pasternak, Herr Professor
ist noch nicht im Haus, tüt -- tüt -- tüt.

Gottseidank. Das war ein Fingerzeig des Himmels. War hätte
sie sagen sollen, wenn sie jetzt plötzlich seine Stimme gehört
hätte\frag{} Erst in der Sekunde, da die kühle Muschel an ihrem
% Seite 44
Ohr gelegen war, hatte sie die Gefahr erkannt. Sie war eine
Träumerin, jawohl, noch immer viel zu tief im Traum der
Glonn versunken. Viele Träume der letzten zwei Wochen
waren ausgeträumt und abgestoßen, doch auch die kühlsten
Pläne der Glonn erwiesen sich jetzt plötzlich immer noch als
trübe Träumerei -- wo steckte die kalte Wirklichkeit, nach der
diese kalte Muschel verlangte\frag{}

Ein gewaltiger Garten, Pflanzen und Tiere und ein paar
Menschen, so rollte die große Kugel um die Sonne. Aus dem
Wald trat ein Mann und schaute über die Ebene. Wer bist
Du, sprach er zu dem Kind am geschlängelten Bach, bist Du
nicht von meinem Stoff, damit wir Hand in Hand marschieren
können über dieses herrliche sonnenbestrahlte Reich\punkte{}%
\eingriff{eS44-1}{Reich\punkte{} ] Reich~\punkte{}}
Oder eine Landstraße, weithin zwischen den Dörfern und
Märkten, die Handwerker zu Fuß, auf Schimmeln und Rappen
die Grandseigneurs und ihre Damen. Wohin so allein, mein
Kind\frag{} Meinen Vater suchen, den Herrn vom Pommeranzen"-%
land\ausr{} Ich bin der Herr vom Pommeranzenland, mein Kind,
doch hast du auch ein kleines Muttermal unter der rechten
Brust\frag{} Hier ist es, mein Vater, in der Form einer kauernden
Taube\punkte{} Oder eine Weltstadt, jeder Tag eine dröhnende
Schlacht, jede Nacht ein Siegesfest mit Feuerwerk, gesegnet,
wer dem stolzen Wachstum dient. Dort sitzt der Mann, der
Brot aus Luft zu machen versteht, die Kellner am Bufett
flüstern sich den Namen zu und die Gäste ringsum starren
mit blanken Augen auf ihn, aber einsam sitzt der große Che"-%
miker vor seinem Schweinsbraten mit Kartoffelsalat, bis
endlich ein Ding in Lichtgrün zu ihm tritt. Erinnern Sie Sich%
\eingriff{eS44-2}{Sich ] sich}
noch an Botzong-Kamin und Gruttenhütte und Daniela
Oldenkott, Professor Pasternak\frag{} Jeden Tag meines Lebens
% Seite 45
denke ich daran, mein Fräulein. So lassen Sie Sich%
\eingriff{eS45-1}{Sich ] sich}%
, aber bitte
nicht sentimental zu werden, ein kleine Geschichte erzählen,
Professor. Kein Wort mehr, meine einzige kleine Tochter,
ich wartete auf diese Stunde\punkte{} Pfui über dies Geträum,
gut für die Glonn, gut für Dichter und Dienstmädchen, wo
steckte die kalte Wirklichkeit, nach der diese kalte Muschel ver"-%
langte\frag{} \aa{}Mein Name ist Lea Herse, ich muß Sie dringend
sprechen, Professor.\ae{} \aanah{}Worum handelt es sich\frag{}\ae{} \aa{}Nicht
um Geld und nicht um Chemie, sondern um eine seltsame familiäre
Sache, ich kann es am Telefon nicht sagen.\ae{} Dann kam er an
die Reihe und gab ihr eine Stunde. Dann traf sie ihn und
überließ sich dem Schicksal. Dann rollte das Schicksal im Zwie"-%
gespräch dahin\punkte{}%
\eingriff{eS45-2}{dahin\punkte{} ] dahin \punkte{}}
Und hier
steckte der Rechenfehler der Glonn\dopp{} im Zwiegespräch. In der Glonn vertraute man noch
auf Zwiegespräche, in der Großstadt fühlte man plötzlich,
daß es nur noch Geschäfte und Monologe gab. Würde sie
nicht einen höchst blamablen Monolog in diese kalte Muschel
sprechen, wenn der Mensch am andern Ende der Welt nicht
die richtige Antwort gab\frag{}

Ach was\ausr{} Selbstverständlich mußte sie ihren Vater noch heute
Aug in Aug sehn\ausr{} Selbstverständlich mußte sich alles andere
von selbst ergeben\ausr{} Wer seine Fahrt vor Beginn zu Ende
denken will, versagt schon beim ersten Schritt. Wie gehst Du
so fröhlich mit tausend Füßen dahin, frug die Heuschrecke den
Tausendfüßler, wie machst Du das\frag{} Und der Tausendfüßler
stoppte und überlegte und kam bis an sein Ende keinen
Schritt mehr voran.

Zentrale Hotel, Amt, Zentrale Pasternak, jawohl, Herr Pro"-%
fessor ist im Haus, er hat eine dringende Sitzung, wird aber
nicht mehr lange dauern, tüt -- tüt -- tüt.

% Seite 46
Eine Sitzung hatte Papi, ei ei\frag{} Vermutlich im Zimmer neben
dem freundlichen Telefonisten\frag{} War man sich also doch schon
bis auf zehn Schritt nahgekommen im unendlichen All\frag{}
Selbstverständlich war es die Einfalt der Glonn, diese plötz"-%
liche Angst, daß es in der Stadt nur noch Monologe und
Geschäfte gäbe. Selbstverständlich gab es auch noch Zwiege"-%
spräche. Reizende Zwiegespräche gab es noch, hin und her und
her und hin, der eine Mensch sagte Bims, da sagte der andere
Mensch Bams, mit Bims-Bams aus einem einzigen Munde
kam keine Menschenseele vom Fleck. Immer hübsch hin und
her im Zwiegespräch, so würde sich auch dieser große Fall
ganz von selbst entwickeln, was konnte viel passieren\ausr{}

Zentrale Hotel, Amt, Zentrale Pasternak, jawohl, eine Mi"-%
nute, sofort --

\aa{}Ja\frag{}\ae{}

Das war eine weibliche Stimme. Der böse Wolf hatte Kreide
gefressen, um die sieben Zicklein zu täuschen.

\aa{}Sie wünschen Herrn Professor selbst\frag{}\ae{}

\aanah{}Wer ist am Apparat\frag{}\ae{}

\aa{}Lea Herse\punkte{}\ae{}

\aa{}Einen Augenblick.\ae{}

Dies also war ein Augenblick. Wer war doch der Mann, vor
dessen Sinn tausend Jahre waren wie ein Augenblick\frag{}

\aa{}Sind Sie noch da\frag{}\ae{} Es war immer noch der Wolf aus dem
Vorzimmer. \aa{}Um was handelt es sich\frag{}\ae{}

\aa{}Ich will Herrn Professor Pasternak sprechen\punkte{}\ae{}

\aa{}Ja in welcher Sache denn\frag{}\ae{}

\aa{}Eine ganz persönliche Sache\punkte{}\ae{}

\aa{}Moment.\ae{}

% Seite 47
Unsinn. Persönliche Sache war Unsinn gewesen. Geschäftliche
Sache hätte es%
\eingriff{eS47-1}{es ] er}
heißen müssen. Es handelte sich um etwas
Chemisches, um irgendein Patent, um die Erfindung, Kaviar
aus Kuhmist zu fabrizieren --

\aa{}Pasternak.\ae{}

\aa{}Professor Pasternak selbst\frag{}\ae{}

\aa{}Ja, wer ist dort\frag{}\ae{}

Sie schwieg. Kein Wort fiel ihr durch den Sinn.

\aa{}Bist Du es, Zilly\frag{}\ae{}

Zilly\frag{} Wer war Zilly\frag{}

\aa{}Halloh\frag{} Was los\frag{}\ae{}

\aa{}Hier ist Miß Dsching-Dschang-Dschenkerlein aus New York\ae{},
sagte sie mit starkem amerikanischem Mauschel-Akzent. \aa{}Ich
bin von die New York-Times befohlen, Herrn Professor
Pasternak um eine kleine Interview zu bitten~--\ae{}

\aanah{}Wer ist das\frag{}\ae{}

\aa{}Dschenkerlein, New York, Times, \fremdsprachlich{photographer and inter"-%
viewer} --\ae{}

\aa{}Ja und\frag{}\ae{}

\aa{}Ich interessieren mich for Ihnen, Professor, weil Sie so
chemisch sind --\ae{}

\aa{}Das ist irgend eine Mystifikation, wer ist denn dort\frag{}\ae{}

\aa{}Dschenkerlein, New York, ich haben eine ähnliche Patent
wie der Herr Professor, ich machen \fremdsprachlich{diamants}\label{lS47-1} aus
Kuhmist --\ae{}

\aa{}Ich verbitte mir Ihre dummen Witze\ausr{}\ae{}

Tüt -- tüt -- tüt.
\abstand{}
% Seite 48
Der junge Mann hieß Bechterew, \fremdsprachlich{Dr.}\label{lS48-1} Fritz
Bechterew, wissenschaftlicher Mitarbeiter einer großen Tageszeitung.
Das Abendessen in dem kleinen russischen Restaurant be"-%
stand aus Borschtsch, Beefsteak, Obstsalat, Wodka. Die Ka"-%
pelle spielte argentinische Tangos, Wiener Walzer, Urwald-%
Krach, Beethoven, \fremdsprachlich{I am lonely without you,} Lieder vom
Rhein.

\aa{}Bitte, sehn Sie mal möglichst unauf"|fällig nach dem dritten
Tisch hinter Ihrem Rücken\ae{}, sagte Bechterew zu Lea. Er
sprach eine Spur Sächsisch, denn er stammte aus Chemnitz,
doch das war nur für geographisch geschulte Ohren zu hören.
Lea Herse nahm ihn für einen amüsanten und gerissenen
Berliner Juden, obwohl er als dumpfer Antisimist und als
englisches Halbblut gelten wollte. Seit einer halben Stunde
analysierte er ihr die Gäste ringsum, ihre Berufe und ihre
Laster, ihre Lächerlichkeit und ihre Tricks, die meisten waren
Freunde oder Bekannte von ihm.

\aa{}Der dritte Tisch hinter Ihnen, der fette kleine Kerl mit der
hübschen blonden Leiche in Schwarz. Was, glauben Sie, ist
der Mann\frag{}\ae{}

\aanah{}Ach, ich will es gar nicht wissen\ae{}, sagte Lea, ohne sich zu
wenden. \aa{}Ich hab schon genug. Halte ich ihn für einen Dichter,
dann ist er gewiß ein Lustmörder, und umgekehrt\ausr{}\ae{}

Bechterew lachte. \aa{}Erraten. Er ist nämlich beides, Dichter
und Lustmörder in einer Person\dopp{} berühmter Kritiker. Martin
Frogge, guter Freund von mir, an den gleichen Zeitungstrust
versklavt wie ich. Wollen Sie ihn kennenlernen\frag{}\ae{}

\aa{}Nein, danke.\ae{}

\aanah{}Verzeihn Sie, wenn ich ihn begrüße, eine Minute nur\frag{}\ae{}

\aa{}Bittebittebitte --\ae{}

% Seite 49
Sie war froh, eine kleine Weile allein zu sein. Der erste Tag
der Expedition ging zu Ende, der erste Ansturm war abge"-%
schlagen, doch der Tag war nicht verloren, morgen würde ihr
das Leben in der Großstadt sicherer von der Hand gehn als
heut. Sehr gut, daß sie unter Führung des Herrn Bechterew
ein paar Stunden auf dem Asphalt herumgerutscht war, ehe
sie den Kampf mit ihrem Vater antrat. Sehr gut, daß sie sich
von einem Fremden auf der Straße hatte ansprechen lassen,
zu einem kleinen Bummel durch die große Welt, die sie nur
aus Büchern und Zeitungen kannte, Straße, Kaufhaus
Kino, Restaurant. Bei dem Blinden mit dem Hund war sie
von Herrn Bechterew angesprochen worden. Er hatte be"-%
hauptet, schon eine ganze Stunde hinter ihr hergetrabt zu
sein, verliebt in ihren Gang, und verlegen, weil er sich noch
niemals einer Dame auf der Straße angehängt hätte, erst die
gemeinsame Ergriffenheit vor dem blinden Bettler mit dem
Hund hätte den Bann gebrochen\dopp{} doch das war sicher Lüge
gewesen. Alles war Lüge ringsum. Aber man mußte diese
Lüge fest ins Auge fassen, um die Wahrheit, die dahinter steckte,
zu packen.

Die Musik spielte einen amerikanischen Marsch. Der Refrain
wurde von allen Nichtbläsern der Kapelle mitgesungen. Lea
geriet in eine heroische Stimmung bei diesen ungewohnten
schrillen Klängen. Wunderbar war das Leben.

Wunderbar war das Leben, wenn man Augen hatte zu
schauen und Ohren zu hören und ein Herz sich zu entscheiden.
Am Rand der steilsten Glonner Wiese lag Frau Herse be"-%
graben, nach einem blutigen Kampf mit den Behörden war
es gelungen, ein Grab außerhalb des Kirchhofs graben zu
dürfen, ein Grab auf eigenem Boden. Fast hätte Lea in letzter
% Seite 50
Stunde den Kampf aufgegeben und den Leichnam an Staat
und Kirche ausgeliefert, schließlich war dieser heidnische
Wunsch ihrer Mutter nur eine Sentimentalität gewesen und
im Tod war alles so gleichgültig. Jetzt kam eine tiefe Freude
über sie, daß sie den Kampf gegen die Beamten siegreich
durchgefochten und die Mutter in der Erde der Glonn einge"-%
bettet hatte.

Ihr verdankte sie es, daß sie ein Herz hatte, sich zu entscheiden.
Nicht aus Weltangst hatte sich Frau Daniela in die Glonn
zurückgezogen, sondern in blanker Entschiedenheit. Sie hatte
sich entschieden, entschieden, entschieden, das war es. Leicht
möglich, daß sie sich für die falsche Seite des Lebens entschie"-%
den hatte, vielleicht mußte man sich in das Chaos stürzen,
mittenhinein, um eingetragen zu werden in das Gästebuch
des Weltgeists\dopp{} aber sie hatte sich wenigstens entschieden, sie
hatte nicht geflunkert wie diese Zeitgenossen ringsum, wie
diese Hin-und-her-Pendler zwischen Gut und Bös. Frau
Daniela hatte sich entschieden, das war die große Erinnerung,
die ihre Tochter ihr an diesem Abend weihte. So oder so,
Glonn oder Weltstadt, tierischer Dünger oder Chemie, es gab
keinen Mittelweg, das war die Erkenntnis ihre rsten Groß"-%
stadt-Tags, das war das heroische Gefühl bei diesen schrillen
amerikanischen Synkopen. Man mußte sich zwischen zwei
Welten entscheiden und es war heroisch, sich entscheiden zu
müssen. Wunderbar war das Leben, wie auch die Entscheidung
in den nächsten Tagen fiel, wenn sie nur fiel.

\aa{}Bitte, Frogge, gehn Sie doch mal möglichst unauf"|fällig an
meiner Lady vorbei\ae{}, sagte unterdessen Bechterew zu dem klei"-%
nen fetten Herrn am dritten Tisch hinter Leas Tisch. \aa{}Ein toller
Fall\ausr{} Möchte furchtbar gern wissen, was Sie von ihr halten.\ae{}

% Seite 51
\aa{}Mensch, löchern Sie mich nicht mit Ihren ewigen Weiber"-%
geschichten -- na, was ist denn, Ober, wo sind denn meine
Muscheln, eine halbe Stunde warte ich schon, bei Euch geht
das Geschäft wieder mal viel zu gut --\ae{}

\aa{}Nein wirklich, Frogge, ein toller Fall. Das Mädchen ist heute
zum ersten mal in ihrem Leben in der Großstadt, die richtige
Unschuld vom Lande, aber klüger als wir alle zusammen.\ae{}

\aa{}Das muß ich mir begucken\ae{}, sagte Martin Frogges Freundin.
Sie hieß Lizzy von Save und war aus Düsseldorf importiert,
um rhythmische Tanzkunst zu studieren.

\aa{}Guck sie Dir an, die Unschuld aus der Dragonerstraße\ae{},
sagte Frogge und ballte wütend die Faust gegen den Kellner,
der am Nachbartisch servierte.

\aa{}Kostet zehn Pfennig\ae{}, sagte die rhythmische Tänzerin und
hielt Bechterew die offene Hand hin.

Bechterew gab ihr ein Markstück. Sie schritt mit einem
scharfen Seitenblick auf Lea rhythmisch zur Toilette.

\aa{}Hören Sie mal, Bechterew\ae{}, sagte Martin Frogge, \aa{}Ihr
gestriger Artikel über die Lehre vom Yoga war aber richtiger
Schmus. Glauben Sie wirklich an dieses Zeug\frag{}\ae{}

\aa{}Ich schwöre Ihnen, Frogge, es ist die einzige Religion, an
die ich noch glauben kann.\ae{}

Martin Frogge bekam seine Muscheln und stürzte sich wie ein
trojanischer Held darüber her.

\aa{}Tatsächlich, Frogge\ausr{} Der Mensch lebt zwischen dem Zentrum
der Erdkugel und dem Zentrum der Sonnenkugel. Wenn der
Mensch in dem Gefühl lebt, die Verbindung zwischen dem
Erdmittelpunkt und dem Sonnenball zu sein, dann hat er
sein wahres Lebensgefühl gefunden.\ae{}

% Seite 52
\aa{}Und wenn er es nicht hat\frag{}\ae{}

\aa{}Dann ist er tot, geistig und seelisch tot, der körperliche Tod
spielt dabei gar keine Rolle.\ae{}

\aa{}Und was bekommen Sie für so einen Aufsatz gezahlt\frag{}\ae{}

\aa{}Das ist Geschäftsgeheimnis.\ae{}

\aanah{}Auf jeden Fall viel zu viel.\ae{}

\aanah{}Auf jeden Fall viel zu wenig, Sie kennen ja unsere Blut"-%
sauger. Aber das Tolle ist, daß ich gerade heute dieses
Mädchen dort treffe, sie hat nämlich wirklich jenes uralte und
doch so moderne Lebensgefühl, von dem ich in meinem
Aufsatz schrieb. Ehrenwort, Frogge, ich spüre es bei ihr ganz
genau.\ae{}

\aanah{}Toller Fall\ae{}, sagte Martin Frogge und legte sich lachend eine
neue Portion Muscheln vor. \aa{}Na was ist, Lizzy\frag{} Ist sie Yoga
oder ist sie Yogurt\frag{}\ae{}

Die rhythmische Tänzerin zählte schweigend neun Zehn"-%
pfennigstücke auf den Tisch.

\aa{}Na, was ist\frag{}\ae{},\eingriff{eS52-1}{ist\frag{}\ae{}, ] ist\frag{}\ae{}} frug Bechterew gespannt und schob das
Kleingeld in die Hosentasche.

\aa{}Schlagen Sie Sich%
\eingriff{eS52-2}{Sich ] sich}
diese Dame aus dem Kopf, Bechterew"-%
chen\ausr{}\ae{}

\aa{}Ist sie nicht hübsch\frag{}\ae{}

\aa{}Hübsch ist sie, sehr hübsch, aber --\ae{} Sie winkte Bechterews
gekräuseltes Haupt zu sich und flüsterte ihm mit gespitztem,
aristokratischem Mäulchen ihre Diagnose ins Ohr.

Martin Frogge brach in Gelächter aus, ohne hinzuhören.
\aa{}Na eben, na was denn, na selbstverständlich\ausr{} Das seh
ich doch von hier aus an ihrem knabenhaften Schulter"-%
geblätter\ausr{}\ae{}

% Seite 53
Er trank strahlend dem zusammengeknickten, wissenschaft"-%
lichen Mitarbeiter seiner Zeitung zu. \aa{}Hoch Yoga, Bechterew\ausr{}
Hoch Yoga, Berlin, Chemnitz und Lesbos\ausr{}\ae{}
\abstand{}
Die Uhr auf dem Nachttisch war stehn geblieben. In der
Glonn wußte man beim Erwachen, wie spät es war vor
allem im Juni. Dort brauchte man nicht erst die Augen zu
öffnen, um nach dem Licht des Tags zu sehn. Drei Uhr
und es piepsten die ersten Vögel, vier Uhr und der Krallen"-%
peter begann die Sense zu dengeln, fünf Uhr und die alte
Nana wälzte sich mir Krach aus ihrer geblümten Kiste, sechs
Uhr und das Konzert des Tags stand auf fortissimo. Hier
wußte man nichts. Totenstille ringsum. Besaß man Geld,
so konnte man sich mitten in der Weltstadt eine Portion
Totenstille kaufen. Totenstille mit Luxussteuer. Und prunk"-%
voll sanken die samtenen Vorhänge des Fensters zum Par"-%
kett herab und garantierten das Dunkel, das Totendunkel
mit Luxussteuer.

Um elf Uhr wollte Herr Bechterew anrufen. Es handelte
sich um eine chinesische Ausstellung, die man unbedingt gesehn
haben mußte. Aber Herr Bechterew war zum Schluß etwas
abgekühlt gewesen, das liebe Halbblut. War er eifersüchtig
gewesen, weil sie sich von Herrn Martin Frogge Theater"-%
karten für heute Abend besorgen lassen wollte\frag{} Nun hatte
sie sich doch noch in Herrn Martin Frogges Gesellschaft ziehn
lassen\ausr{} Um zwei Uhr sollte sie Fräulein Lizzy von Save
treffen, der Name des Lunch-rooms stand auf dem kleinen
Notizblock, viele interessante und rhythmische Menschen und
% Seite 54
sehr billige Knödel mit Kraut gab es in jenem Lunch"-%
room. Dann mußte endlich das Telegramm mit der Adresse
und der guten Ankunft an Terese Nüll und Nana abgesandt
werden. Dann mußte anstandshalber eine alte Tante
Oldenkott besucht werden, die in Berlin wohnte. Bechterew
schätzte die Fahrt bis zur Wohnung dieser alten Tante auf
mindestens eine Stunde, mit der besten Verbindung eine
geschlagene Stunde, alle alten Tanten wohnten an der
Peripherie.

Sie sprang aus dem Bett und beschloß, nach allen Seiten
wortbrüchig zu werden. Nach dem Stand der Sonne war es
schon elf Uhr, sie telefonierte in die Zentrale des Hotels, daß
sie nicht zu sprechen wäre. Herr Bechterew sollte sich bei dem
Blinden mit dem Hund eine andere Ergriffene aufgabeln,
Fräulein von Save sollte ihre Lunch-room-Knödel allein
verzehren, Herr Frogge sollte die Theaterkarten für das Nackt"-%
ballett an Tante Oldenkott schicken, die Arme war siebzig und
hatte gewiß schon lnge nichts Nacktes mehr gesehn. Neun
Telefon-Nummern standen schon auf dem kleinen Notizblock\dopp{}
morgen waren es neunzig, wenn sich pro Kopf zehn neue Be"-%
ziehungen auftaten. In einem Monat ging es ihr wie Mister
Dschenkerlein aus New York\dopp{} der hatte zehn Jahre lang
ununterbrochen telefoniert, und als man ihm dann den Kopf"-%
hörer mit Gewalt abriß, krochen ihm große bleiche Würmer
aus den Ohren, unzählige Würmer, immerzu Würmer, bis
er endlich den letzten Wurm von sich gab und starb.

Aber gesegnet sei das Klima dieser Stadt\ausr{} Eine herrliche Luft
drang durch das offene Fenster, als die Vorhänge zurück"-%
gewürgt waren. Wie ein Nichts wurde der schlechte Atem der
Millionen Lungen vom Himmel aufgeschluckt und in reine
% Seite 55
Ware umgetauscht, von Herzschlag zu Herzschlag ein ewiger
Vorrat an Reinheit und Macht.

Gesegnet diese Luft und gesegnet dieses Badezimmer\ausr{} Bei
Kalt allein konnte man die Brause nicht sehr lange ertragen,
aber bei einer winzigen Mischung mit Warm konnte man ver"-%
weilen. Fing man den Strahl im Nacken auf, so stob der
Wassergott in dunklem Zorn das harte Flußbett der Wirbel"-%
säule hinunter. Doch bog man sich zurück, so glitt er sanft und
hell über das kleine weiche Vordach der Brüste und umspielte
mit zarten Rinnsalen den Bauch und seine Säulen.

Schluß, Frottage, Wäsche, Kleid, Schuhlitzen-Abreißen, Hut,
Frühstück, Taxi, helle Straßen, schnelle Menschen, Portier,
Vorhalle, Lift, zweiter Stock, Anmeldung.

Auf dem Zettel, den der kleine uniformierte Bengel ihr
vorlegte, stand gedruckt\dopp{} \aa{}Frau oder Herr~\ldots{}\ldots{}%
\eingriff{eS55-1}{Herr~\ldots{}\ldots{} ] Herr\ldots{}\ldots{}}
wünscht
Frau oder Herrn~\ldots{}\ldots{}%
\eingriff{eS55-2}{Herrn~\ldots{}\ldots{} ] Herrn\ldots{}\ldots{}}
in Angelegenheit~\ldots{}\ldots{}%
\eingriff{eS55-3}{Angelegenheit~\ldots{}\ldots{} ] Angelegenheit \ldots{}\ldots{}}
zu sprechen.\ae{}
Natürlich wünschte Frau Herse Herrn Pasternak zu sprechen,
doch in welcher Angelegenheit\frag{} Angelegenheit Gruttenhütte\frag{}
Angelegenheit 29.~September 1901\frag{} Angelegenheit \aa{}Eine-%
Waise-im-Sturm-der-Zeit-sucht-ihren-väterlichen-Felsen\frag{}\ae{}

Warum hatte sie den schönen Brief zerrissen, den sie ihm
gestern geschrieben hatte\frag{} Warum wollte sie ihn durchaus
Aug in Aug sehn, bevor sie ihr Geheimnis preisgab\frag{} Wozu
all diese Schwierigkeiten\frag{}

Neben ihr stand ein blutarmer Familienvater und schrieb
ohne Besinnen seine Angelegenheit auf seinen Zettel\dopp{} \aa{}Herr
Meier~1 wünscht Herrn Maier~2 in Angelegenheit Habe-%
nichts-zu-fressen zu sprechen.\ae{} Der uniformierte Bengel be"-%
gann sie bereits mißtrauisch abzuschätzen, weil sie ihre An"-%
gelegenheit nicht wußte. Ja, mein Junge, Deine Angelegen"-%
% Seite 56
heit ist kar, Du bist ein kleiner Maulaufreißer und wirst
mal ein großer Maulaufreißer werden, bei mir liegt die
Angelegenheit etwas schwieriger, mein süßer Gockel mit den
engen Hosen.

Sie zerriß ihren Zettel und gab dem Anmelde-Gockel einen
Fünfziger. Sie ging wieder. Hinab das große Treppenhaus
\aanah{}Wir-sind-wir\ae{}, hindurch durch die große Vorhalle \aa{}Eine-%
feste-Burg-ist-unser-Geld\ae{}, vorbei an dem Portier \aanah{}Alle-%
Menschen-gehören-ins-Zuchthaus\ae{}, auf die Straße.
\abstand{}
Tags zuvor, beim Bummel über den Bummel, hatte Bech"-%
terew von einer schrecklichen Großstadt-Krankheit erzählt, die
er Buden-Angst nannte. Nach Bechterew war die Buden-%
Angst schleichend und unbemerkbar. Bei männlichen Patien"-%
ten konnte man am Anfang der Erkrankung leichte Zer"-%
streutheit und leichte Verstopfung feststellen, bei weiblichen
Patienten ziehende Schmerzen in den Hüften, verstärkte
Schönheitspflege, verstärktes Wippen beim Gang\dopp{} doch dieses
erste Stadium der Krankheit wurde meistens übersehn, weil
es als normal galt. Erst wenn die Buden-Angst in voller
Blüte stand, merkten die Patienten, daß sie vor ihrem eigenen
Selbst und vor ihrer leeren Bude entsetzliche Angst empfanden.
Dann griffen sie ohne Wahl nach den zahlreichen Medika"-%
menten, die von der Großstadt zur Linderung der Buden-%
Angst geboten wurden. Gesteigertes Arbeitstempo war bei
allen Buden-Ängstlern als Linderung ihrer Seuche sehr
beliebt. Für Kassenpatienten wurde außerdem empfohlen
Theater-Kino-Radio-Sport. Bessere Patienten nahmen Reli"-%
gion-Literatur-Psychoanalyse-Auto-Verjüngung. Allen sozi"-%
alen Klassen gemeinsam war der große Verbrauch an Zeitung,
der billigsten Droge gegen Buden-Angst. Niemals jedoch hatte
man gehört, daß ein richtiger Buden-Ängstler durch alle diese
Drogen der großstädtischen Buden-Angst-Industrie geheilt
worden war.

Schon am zweiten Tag ihres neuen Lebens lernte Lea Herse
diese Krankheit kennen. Nach einem sinnlosen Trottoirmarsch
vom Zentrum der Stadt zum Westen der Stadt, kehrte sie
mit müden Füßen und müden Augen in ihr Hotelzimmer
zurück. Aber fünf Minuten später trieb sie die Angst vor der
leeren Bude wieder auf den Asphalt. Nun mußte sie doch in den
neuen Lunch-room wallen, zu dem preiswerten Sauerkraut
der rhythmischen Tänzerin, sie wußte nichts anderes zu tun.

Zum zweitenmal in ihrem Leben ein Bummel über den Bum"-%
mel. Beim Passieren eines Blumenladens fiel ihr ein, daß
jetzt in der Glonn ihr \aa{}Delphinium Berghimmel\ae{} seine erste
Blüte trieb. Zwei Meter hoch, der Stolz ihres Stauden"-%
gartens, der hellblaue Rittersporn \aa{}Delphinium Berg"-%
himmel\ae{}, daneben die lichte \aa{}\fremdsprachlich{Sancy de Parabère}\ae{}, die
Kletterrose des Hersehofs, jetzt blühten sie. Hier blühte die
Buden-Angst.

\aa{}Ich dachte schon, Sie versetzen mich\ae{}, rief Fräulein Lizzy
von Save durch den überfüllten Kohl-room. \aa{}Da wären Sie
% Seite 58
aber schwer reingefallen, mein Kind. Erstens gibt es hier eine
Gemüseplatte, wie sie in der ganzen Stadt nicht mehr zu
finden ist -- Sie sind doch als Landmensch gewiß ein richtiger
Gemüse-Mensch\frag{} Und zweitens habe ich Ihnen hier die klügste
Frau von Berlin mitgebracht, sie ist schon ganz verzweifelt,
weil Sie nicht kommen. Fräulein Lea Herse aus den Alpen --
Frau Johanna Duske aus -- woher stammst Du eigentlich,
Johanna\frag{}\ae{}

\aanah{}Aus dem Nichts\ae{}, sagte die klügste Frau von Berlin und
reichte Lea die Hand. Sie trug eine Hornbrille und war stark
gepudert. Aber hinter den großen Gläsern steckten ein paar
wirklich kluge Augen, ein ähnliches Eisblau wie Leas Augen,
und unter dem Puder steckte eine frische junge Haut. Im
Gegensatz zu der Rhythmischen schien sie ruhig und selbstsicher
zu sein.

\aa{}Sagen Sie offen, Sie wollten mich versetzen\frag{}\ae{}, maulte
Fräulein von Save, nachdem sie für Lea eine Gemüseplatte
bestellt hatte. \aa{}Mein Kind, Sie wollten mich versetzen, ob"-%
wohl Sie mir gestern in die Hand versprochen haben, zu
kommen\frag{}\ae{}

\aa{}Offengestanden ja\ae{}, sagte Lea, \aa{}ich wollte nicht kommen.\ae{}

\aa{}Und dann hat eine geheimnisvolle innere Stimme Sie doch
noch hierhergetrieben\frag{}\ae{}

\aanah{}Ach nein\ae{}, sagte Lea, \aa{}es war nur Buden-Angst, gar nichts
anderes.\ae{}

\aa{}Na so was Entzückendes\ae{}, rief Fräulein von Save begeistert.

\aanah{}Was sagst Du zu dieser frechen Beleidigung, Johanna\frag{}\ae{}

\aa{}Ganz in Ordnung\ae{}, sagte Frau Duske mit ihrem wunder"-%
hübschen tiefen Alt und griff nach der Mittagszeitung. Sie
begann zu lesen und schien Leas Ankunft schon wieder ver"-%
% Seite 59
gessen zu haben. Das war ein wenig kränkend, dennoch fand
Lea dieses freiheitliche Gebaren sehr sympathisch.

Eine Viertelstunde lang war über dem Zeitungsrand nur eine
glatte Stirn und eine glate helle Hartafel zu sehn. Frau
Duske schie sich nicht im geringsten für ihre Umgebung zu
interessieren, weder für den dicken Herrn, der seinen Hut
über ihren Kopf hinweg vom Kleiderständer schwang, noch
für Fräulein Herse aus den Alpen, am wenigsten für Fräulein
von Saves soziale Ideen, die Lea mit der Gemüseplatte
vorgesetzt bekam.

\aa{}Mensch, ich muß gehn\ae{}, sagte die Düsseldorferin endlich und
brach mit dem amerikanischen Armbanduhr-Blick ihr Ge"-%
schwätz ab. \aanah{}Theoretischer Unterricht, Muskellehre, mag ich
gar nicht --\ae{}

Lea geriet in Verlegenheit, während Fräulein von Save
zahlte. Sie wartete noch auf ihre Mehlspeise und Frau Duske
saß hinter ihrer Zeitung, als gäbe es nichts andres auf der
Welt wie Eisenbahnkatastrophen, möblierte Zimmer, Völker"-%
bundssitzungen, Theaterklatsch, fast fabrikneue Markenwagen,
Schwergewichtler.

Sollte sie auf den Apfelstrudel nach Wiener Art verzichten
und mit Fräulein von Save zahlen und geht\frag{} Hatte sie auf
die klügste Frau Berlins einen so üblen Eindruck gemacht,
daß die sich ihretwegen hinter der Zeitung vergrub\frag{} Sie rief
den Kellner zu sich, zu zahlen und zu gehn.

\aanah{}Aber wieso denn\frag{}\ae{} Frau Duske lächelte freundlich über die
Zeitungswand hinüber. \aa{}Bleiben wir nicht noch ein wenig\frag{}\ae{}

\aa{}Ja\frag{}\ae{}, frug Lea verdutzt, \aa{}meinen Sie\frag{}\ae{}

\aa{}Natürlich bleiben Sie\ae{}, sagte Fräulein von Save schnell
und verabschiedete sich. \aa{}Sie sind doch ein freier Mensch, Sie
% Seite 60
stecken doch nicht in der Tretmühle wie ich -- djüs djüs djüs --\ae{}

Sie rannte zur Muskellehre und ließ Lea mit der klügsten
Frau Berlins allein.

\aa{}Ich kann nämlich diesen sozialen Quatsch nicht mehr hören\ae{},
sagte Frau Duske und legte die Zeitung beiseite, während
ihre Freundin noch in der Drehtür steckte. \aa{}Lizzy ist bürgerlich
bis in die Knochen, sie soll die Hände von diesen Dingen lassen
-- aber sie ist reizend, nicht wahr\frag{}\ae{} Dabei blinzelte sie Lea
zu, als wollte sie sagen\dopp{} \aa{}Die blödeste Kuh der Welt, nicht
wahr\frag{}\ae{}

Lea lachte.

\aanah{}Wir verstehn uns\ae{}, sagte Frau Duske mit bezauberndem
Lächeln. \aa{}Sind Sie Kommunistin oder sind Sie Nihilistin\frag{}
Eins von beiden muß man ja schließlich sein\frag{}\ae{}

\aa{}Ich bin gar nix\ae{}, sagte Lea trocken.

\aanah{}Wunderbar\ausr{} Ich bin auch nur beruflich an diesen Dingen
interessiert, nicht substantiell. Ich bin Photographin und
entwerfe nebenbei Dekorationen für die kommunistischen
Theater, ganz interessante Arbeit, jedenfalls besser als bürger"-%
liche Kunst, aber im Grunde der gleiche Schwindel. Inter"-%
essieren Sie Sich\eingriff{eS60-1}{Sich ] sich} für photographische Kunst\frag{}\ae{}

\aa{}O ja\ae{}, sagte Lea, \aa{}was photographieren Sie denn\frag{}\ae{}

\aanah{}Was mir über den Weg läuft\ausr{} Katzen, Menschen, Blumen,
alles. Darf ich Sie auch photographieren\frag{}\ae{}

Lea empfand diese Frage unangenehm und sagte schnell\dopp{}
\aa{}Haben Sie schon mal ein \aa{}Delphinium Berghimmel\ae{}
photographiert\frag{}\ae{}

\aanah{}Was ist das\frag{}\ae{}

\aa{}Eine hellblaue Sorte Rittersporn.\ae{}

\aa{}Möglich, ich kann mich nicht erinnern.\ae{}

% Seite 61
\aa{}Haben Sie schon mal einen Herrn Pasternak photographiert\frag{}\ae{}

\aanah{}Was ist das\frag{}\ae{}

\aa{}Professor Pasternak, ich kenn ihn nicht, aber ich sah mal ein
glänzendes Photo von ihm.\ae{}

\aa{}Pasternak\frag{} Cellist\frag{}\ae{}

\aa{}Nein, berühmter Chemiker.\ae{}

\aa{}Chemiker\frag{} Nein. Ich glaube, ich hab in meinem ganzen
Leben überhaupt noch keinen Chemiker gesehn. Aber trinken
Sie doch eine Tasse Kaffee bei mir, wenn Sie sich\eingriff{eS61-1}{Sich ] sich} für schöne
Aufnahmen interessieren, ich zeige Ihnen ein paar gute Stu"-%
dien. Oder sind Sie heute nachmittag besetzt\frag{}\ae{}

\aa{}Nein, ich bin frei\ae{}, sagte Lea.

Sie fuhren in Frau Duskes Atelier für moderne Bildkunst
und tranken in einer Ecke mit vielen bunten Kissen bitteren
Mokka und süßen Schnaps.

\aa{}Später mache ich ein paar Bilder von Ihnen\ae{}, sagte Frau
Duske, \aa{}aber zuvor will ich Ihnen ein paar interessante
Dinge aus meinem Archiv zeigen. Was wollen Sie sehn\frag{}
Kinder, Hunde, Männer, Weiber\frag{}\ae{} Sie nahm Leas Arm
und schlenkerte mit ihr in das kleine Archiv-Zimmer. \aa{}Männer
interessieren Sie wohl nicht\frag{}\ae{}

\aa{}Nicht im geringsten\ae{}, plapperte Lea ohne Nachdenken. Doch
das schien ein glänzender Spaß gewesen zu sein, denn Frau
Duske gluckste zur Antwort wie eine Lachtaube. \aanah{}Aber wirklich
nicht im geringsten\ae{}, wiederholte Lea, um den verborgenen
Witz dieses Ausspruchs für ihre liebenswürdige Wirtin noch
ein wenig auszutreten. \aa{}Ich will gar nichts von Männern
wissen, Katzen sind mir lieber.\ae{}

\aa{}Wir verstehn uns, mein süßes kleines Alpenveilchen\ae{}, sagte
Frau Duske und legte zuerst ein paar Katzenporträts vor, die
% Seite 62
Lea entzückend fand. Dann kamen ein paar farbige Blumen"-%
studien, Astern und Phlox und Dahlien, aber \aa{}Delphinium
Berghimmel\ae{} war nicht dabei. Ein paar Damen und Hunde
aus der Gesellschaft fand Lea langweilig, doch sie blieb bei
ihrem freundlichen \aa{}Entzückend\ae{}. Es kam eine Mappe mit
Dekorations-Entwürfen für ein modernes Theaterstück\semi{} zu
jedem Blatt gab es eine langatmige Erklärung über die beab"-%
sichtigte Wirkung\semi{} sie verspürte keinen Hauch von dieser Wir"-%
kung und dachte nur noch krampfhaft daran, nicht gähnen zu
müssen. Es kamen ein paar Aktstudien, Freilicht-Aufnahmen\semi{}
sie kannte dieses Zeug bereits aus den illustrierten Zeitungen
und Magazinen, die sich hie und da in die Glonn verirrt hatten\semi{}
sie kannte diese verzückten Stellungen irer berühmten Ge"-%
schlechtsgenossinnen und diese unappetitliche Manier, dem
begeisterten Beschauer den Steiß entgegenzustrecken\dopp{} ent"-%
zückend. Dann kamen doch noch ein paar Männerköpfe, be"-%
rühmte Künstler und berühmte Kaufleute\semi{} die Kaufleute
versuchten künstlerisch zu glotzen und die Künstler glotzten
möglichst kaufmännisch\semi{} eitle Tröpfe die und die\dopp{} entzückend.
Sie war froh, als bei ihrem hundertsten \aa{}Entzücken\ae{} eine
Assistentin im weißen Operationsmantel in das Archiv trat
und Besuch meldete.

\aa{}Na so was\ae{}, rief Frau Duske und schlug sich mit der Hand an
die Stirn, \aa{}wie konnte ich das vergessen\ausr{} Ists schon vier Uhr\frag{}
Für vier Uhr ist nämlich Major Ellen Wladden aus New
York, eine Heilige, eine führende Persönlichkeit in der Heils"-%
armee, setzen Sie Sich\eingriff{eS62-1}{Sich ] sich} während der Aufnahme ganz still in die
Ecke, ich werde Sie als meine Schülerin vorstellen, sonst
% Seite 63
wird mir dieser Engel scheu und verpatzt mir die Aufnahme,
kommen Sie --\ae{}

Lea wurde von trüben Gefühlen beschlichen, als sie wieder
in der Ecke mit den vielen bunten Kissen hockte und guckte,
wie der Engel von der Heilsarmee für ein literarisches Magazin
photographiert wurde. Zuerst war sie von der klügsten Frau
Berlins bezaubert gewesen. Aber die Atmosphäre im Archiv
war ihr aufs Herz gefallen wie Meltau\label{lS63-1} auf eine junge Staude.
Sie wußte selbst nicht, wieso es kam. Vielleicht war sie nur
matt von dem ewigen Bewundern-Müssen\frag{} War Frau Duske
nicht reizend\frag{} Hatte sie sich nicht wie eine Schwester an sie
gepreßt, als sie bei einem roten Phlox-Photo\label{lS63-2} in wirkliche
Begeisterung ausgebrochen war\frag{}

Major Ellen Wladden\label{lS63-3} trug den langen Rock der Heilsarmee, am
Kittel das bescheidene Abzeichen ihres Ranges\label{lS63-4}, dazu die große
Schute mit den zwei Bändern. Sie sprach wenig und ließ sich
geduldig von Frau Duske hin und her schieben, eine halbe
Stunde\label{lS63-5} lang, bis Stellung und Licht paßten\label{lS63-6}. Sie sah tatsächlich
wie eine Heilige aus, ein süßes rundes Gesichtchen, sanfte dunk"-%
le Augen\label{lS63-7}, frische rote Bäckchen, ein wenig säuerlich der Mund
vom vielen Hallelujah und Gebettel um die Gnade des Herrn.

Aber je länger Lea hinsah, umso nervöser wurde sie. Die Hei"-%
lige aus New York wurde ihr von Minute zu Minute peinlicher.
Warum stand sie vor dem Apparat und vor den grellen
Lampen mit dem gleichen Sanftmut wie in den Kaschemmen
vor den Sündern der Welt, ihren frommen Song zu plärren
und ihre Cents zu sammeln\frag{} Warum streckte sie nicht, während
ihr Frau Duske die blöden Schutenbänder wirkungsvoll
drapierte, ihre Zunge aus dem Engelsmäulchen raus\frag{} Wa"-%
rum sprach sie kein Wort\frag{}

% Seite 64
Immer kribbeliger wurde Lea, als die Prozedur nicht zu Ende
kam. Die nervösen Ameisen liefen ihr über die Arme und über
die Oberschenkel. Die Wut auf die Heilige wuchs von Minute
zu Minute in ganz sinnloser Weise. Ein richtiger Krach mit
dem sanften Geschöpf wäre eine wahre Erlösung gewesen.
Nur aus Rücksicht auf die liebe Frau Duske hielt sie an sich.
Und es war ja auch wahrhaftig kein Wort gegen Major Ellen
Wladden vorzubringen\frag{} Ein leibhaftiger Engel, was sollte
dieser schwere heidnische Zorn\frag{}

Wenigstens beantwortete sie den schwesterlichen Abschieds"-%
gruß, als dann der Engel endlich ging, mit einem starren und
gehässigen Blick und übersah die zarte Patschhand, die sich ihr
entgegenstreckte. Sollte der Engel doch fühlen, daß hier ein
Feind gesessen war\ausr{} Vermutlich aber war der Engel viel zu
dumm, um überhaupt etwas zu fühlen, was nicht nach Elend
oder Hallelujah roch\frag{} Hauptsache, es war vorüber und die
neurasthenischen Kribbel-Ameisen verzogen sich.

Frau Duske kam strahlend aus dem Entree zurück. \aa{}Ist sie
nicht himmlisch\frag{} Einer unserer up-to-datesten jungen Lyriker
hat eine brilliante Schauerballade auf sie geschrieben, guter
Freund von mir, wissen Sie, so eine Ballade in primitivem
Stil, dazu brauchen wir ihr Bild -- ist sie nicht wunder"-%
bar\frag{}\ae{}

\aa{}Die\frag{}\ae{}, sagte Lea. \aa{}In meinem Leben noch kein so blödes
Weib gesehn.\ae{}

\aanah{}Was sagen Sie da\frag{}\ae{}

\aa{}Na, fühlen Sie denn nicht, daß das alles Schwindel ist\frag{}\ae{}

\aanah{}Aber hören Sie, Kind\ausr{} Sie täuschen Sich\eingriff{eS64-1}{Sich ] sich} wirklich\ausr{} Ellen
Wladden opfert sich für ihre Idee.\ae{}

\aanah{}Was geht das mich an\ausr{}\ae{}

% Seite 65
\aa{}Man kann sich ja zu jeder Idee stellen, wie man will, aber man
muß doch den heiligen Willen achten\ausr{}\ae{}

\aa{}Das wäre ja noch schöner\ae{},\eingriff{eS65-1}{schöner\ae{}, ] schöner,\ae{}} rief Lea, \aa{}ich achte gar nichts.\ae{}

\aa{}Sie wissen nichts von dieser Frau, mein Kindchen. Die geht
in die finstersten Spelunken, um den Menschen Hilfe zu
bringen, aber wirkliche Hilfe. Die reist in der ganzen Welt
herum und organisiert ihre einzelnen Stationen. Die ist über
jede menschliche Eitelkeit erhaben. Wirklich ein Engel, glauben
Sie mir.\ae{}

\aa{}Glaube ich gern, daß sie ein Engel ist, glaube ich, glaube ich\ausr{}\ae{}

\aanah{}Aber gegen einen Engel läßt sich doch nichts Böses sagen\frag{}\ae{},
rief Frau Duske lachend.

\aa{}Nein\frag{} Läßt sich nichts Böses sagen\frag{}\ae{}, schrie Lea in heller
Wut. \aa{}Dieser Engel ist ein Schwein\ausr{} Jawohl\ausr{} Ein Engel und
ein Schwein in einer Person, jawohl\ausr{} Daß sie in die Hölle
kommt, wenn sie gestorben ist, will ich gar nicht behaupten --
wir andern kommen natürlich in die Hölle und diese Dame
schwebt zum Himmel auf, das glaube ich gern -- als Schwein
mit Flügeln nämlich, als Schwein mit Flügeln wird sie in den
Himmel aufsteigen.\ae{}

Frau Duske brach in ein Gelächter aus und war hingerissen
von diesem unbegründeten und ungerechten Ausbruch.

\aa{}So ist die Sache\ausr{}\ae{}, schrie Lea. \aa{}Jawohl\ausr{} Ganz einsam wird
dieser Engel als Flügelschwein auf seiner nassen Wolke hocken
und Halleluja plärren\ausr{} Ich möchte nicht mir ihr tauschen, tat"-%
sächlich nicht\ausr{} Lieber als das, was ich bin, in der tiefsten Hölle
rösten\ausr{} Immer noch besser als solch Flügelschwein im
Himmel\ausr{}\ae{}

Hingerissen war Frau Duske. Sie umarmte Lea. Die war
froh, daß ihr Groll entleert war und daß die klügste Frau
% Seite 66
Berlins sie verstand. Es schien nicht schwer zu sein, gute
Freunde in der Großstadt zu finden und verstanden zu
werden. Zwar verstand sie selbst nicht recht, warum diese
schlimme Wut auf die sanfte Heilige über sie gekommen war,
aber ihre neue Freundin schien es zu verstehn. Immer wieder
wurde sie von ihr umschlungen. Mit geheimnisvoller Be"-%
geisterung wurde jedes böse Wort aufgenommen, das sie der
Amerikanerin nachrief. Als sie merkte, welches Vergnügen
ihr Zorn auslöste, geriet sie ins Kälbern und kälberte immer
weiter hinter dem Engel her. \aa{}Überhaupt hasse ich alle
Amerikanerinnen\ausr{} Ob sie mit dem lieben Gott flirten oder
mit einem Filmstar, ist ganz egal\ausr{} Das ist zwar die erste
Amerikanerin, die ich in meinem Leben sah, aber das ge"-%
nügt für alle.\ae{} Und das schien nun wieder ein viel größerer
Spaß zu sein, als sie selber ahnte. Frau Duske wälzte sich vor
Lachen, riß sie keuchend an sich, gab ihr einen Kuß auf den
Mund.

Zuerst war es ein fader Lippenstift-Geschmack. Sie ließ sich
gutmütig noch ein paarmal küssen. Plötzlich fühlte sie, daß ihr
Mund nicht mehr freigegeben wurde. Eine feuchte fremde
Zunge versuchte zwischen ihren Lippen vorzudringen. Wie eine
Qualle hing es plötzlich an ihr. Ein blutloser Leichnam trotz
diesem wüsten Drängen, eine lustlose Gehirnblase trotz diesen
wirren Trieben, so hing plötzlich die klügste Frau Berlins an
ihr. In vollem Entsetzen stieß sie die Qualle zurück und schlug
ihr mit ganzer Kraft zweimal mitten ins Gesicht.

Die Brille flog zu Boden und zerklirrte. Die klügste Frau von
Berlin hielt sich stöhnend die Hände vors Gesicht und schrie\dopp{}
\aa{}Hinaus, Du Hure\ausr{}\ae{} Lea wollte noch irgend etwas rufen, Du
Qualle, Du Bestie, Du Buden-Angst, Du Irgendetwas, aber
% Seite 67
sie preßte die Lippen aufeinander und lief stumm aus dem
Atelier. Das Dienstmädchen, das ihr die Gangtür öffnete,
grinste ihr verständnisvoll nach.

Vor dem Haus fand sie ein leeres Taxi. Ohne Besinnen gab
sie dem Chauffeur die Adresse ihres Vaters. Erst als sie schon
zehn Minuten gefahren war, fiel ihr ein, daß Samstag war,
Samstag und fünf Uhr nachmittags, sämtliche Bureaus der
Stadt waren längst geschlossen, ganz gewiß, auch das Ge"-%
schäftshaus ihres Vaters war geschlossen. Aber sie ließ den
Chauffeur die ganze Strecke zu Ende fahren, vom Westen der
Stadt zum Zentrum der Stadt, ohne sich zu rühren.
\abstand{}
Die Vorhalle des Hauses Pasternak war leer. Auch der Por"-%
tier \aanah{}Alle-Menschen-gehören-ins-Zuchthaus\ae{} war nirgends
zu sehn. Natürlich war längst Schluß.

Sinnlos, hier noch irgend etwas zu hoffen. Aber durch diesen
Raum war vor kurzer Zeit ihr leibhaftiger Vater geschriten,
velleicht schwebte noch ein Stückchen Unsichtbares von ihm
in der Luft, um ihre anwachsende Melchancholie zu umspielen
und zu lindern\frag{}

Sie lief vor der Marmortreppe ein bißchen auf und ab, als
erwarte sie noch einen säumigen Liebhaber aus einem der
Bureaus. Während zwei uniformierte Bengelchen die Treppe
herunterschlaksten, warf sie ungeduldige Blicke auf die Uhr
überm Portal, um zu zeigen, daß sie bestellt war. Die
Bengelchen liefen an ihr vorbei auf die Straße, ohne sie zu
beachten. Die Halle war wieder still und tot.

% Seite 68
Nichts. Sie mußte wieder gehn. Sie mußte wieder ins Hotel
gehn, zurück in die leere Bude. Sie mußte ihm einen neuen
Brief schrieben.

Als sie vom Portal aus einen letzten Blick in die Halle warf,
sah sie noch einen verspäteten Angestellten des Werkes Paster"-%
nak die Treppe herunterkommen. Ein subalterner älterer Herr,
ein abgekämpfter Beamter mit Spitzbauch und Aktenmappe,
müde und überarbeitet stapfte er die Marmortreppe herab
und auf sie zu. Da er zwischen den Türen des Portals einen
verwunderten Blick auf sie warf, was sie wohl hier zu suchen
hätte, wollte sie ihn wissen lassen, daß sie keine Einbrecherin
war, sondern eine Freundin seines großen Chefs, und sprach
ihn an. \aanah{}Verzeihung -- Professor Pasternak noch im Haus\frag{}\ae{}
Der Spitzbauch blieb vor ihr stehen. \ae{}Professor Pasternak\frag{}
Ich glaube, ja, er ist noch im Haus.\ae{}

Sie fühlte, daß sie knallrot wurde. \aa{}So\frag{}\ae{}, sagte sie kühl.
\aa{}Besten Dank.\ae{}

Er wollte sich offenbar in ein Gespräch mit ihr einlassen. Er
blieb stehn, zog eine hölzerne Dose aus der Tasche, rauchte
sich mit Pedanterie eine Zigarette an und musterte sie dabei
mit gutmütigen, dunklen Augen. \aa{}Sie wünschen Professor
Pasternak zu sprechen\frag{}\ae{}

\aa{}Jawohl -- er ist ganz bestimmt noch da\frag{}\ae{}

\aa{}Ich glaube ja -- fragen Sie doch bitte mal im ersten Stock bei der
Anmeldung nach -- wenn ich mich nicht irre, ist er noch oben --\ae{}

\aa{}Danke --\ae{}

Sie schritt in die Halle zurück und flog die Treppe empor.
An der Ecke vor dem ersten Stockwerk stoppte sie und wartete
fünf Minuten. Sie wollte nicht noch mal vor einem leeren
Zettel mit \haa{}Angelegenheit\hae{} stehn, für heute war ihr Mut
% Seite 69
dahin\dopp{} aber sie
%% Erster Teil.

% Seite 
\kapitel{Drittes Kapitel}



\addchap{Zweiter Teil}
\cleardoublepage
%% Zweiter Teil.

% Seite 107
\kapitel{Erstes Kapitel}


%% Zweiter Teil.

% Seite 131
\kapitel{Zweites Kapitel}


%% Zweiter Teil.

% Seite 
\kapitel{Drittes Kapitel}


\addchap{Dritter Teil}
\cleardoublepage
%% Dritter Teil.

% Seite 199
\kapitel{Erstes Kapitel}


%% Dritter Teil.

% Seite 229
\kapitel{Zweites Kapitel}


%% Dritter Teil.

% Seite 
\kapitel{Drittes Kapitel}



\endlinenumbers % Ab hier Zeilennummern wieder aus.




% ************************* Anhang *****************************
\addpart{Anhang\vspace{2mm}}
\cleardoublepage
\begin{small}

%% Anhang.

%%SuppressWarnings("InputStyle")
%%SuppressWarnings("ContentCheck")

% Hier folgendes machen:
% a) Ein Makro schreiben, dass das ordentlich formatiert, ala Endnoten.
% b) Prüfen ob man in Latex eine Datenstruktur anlegen kann, in
% der man alle Anwendungsreferenzen speichern kann.
% c) Damit kann dann überprüft werden, ob alle Referenzen behandelt wurden
% und
% d) ggf. kann man die Referenzen in den Text einfügen, was diesen aber u. U.
% aufblähen könnte.

\kapitel{Anmerkungen}

\anmerkung{lS63-1}{Meltau}{Meyers Großes Konversations-Lexikon 1905 ff.\ bezeichnet
das, was heute Mehltau geschrieben. Meltau heute bedeutet nach aktueller
Orthographie Honigtau.}

\anmerkung{lS88-1}{Bebe}{Keinerlei Einträge in diversen Wörterbüchern gefunden.}

%% Anhang.

\kapitel{Zu diesem Band}

Der edierte Text folgt grundsätzlich zeichengetreu der Buchausgabe:

xxxxxx xxxxx xxxxxxx xxxxxxx xxxx xxx xxxx xxx xxxxx xxxx xxx xxx xxxx
xxx xxxxx xxxxx

In Antiqua gesetzte Wörter in der
Buchausgabe werden hier \begin{it}kursiv\end{it} gesetzt wiedergegeben.

Eingriffe in den Text:

\theendnotes


\end{small}




% ******************** Inhaltsverzeichnis **********************
\tableofcontents




\end{document}
