%% Erster Teil.

% Seite 43
\kapitel{Zweites Kapitel}

Der liebe Gott, nähme er wieder einmal Menschengestalt an,
um unter seinen Erdenkindern zu wandeln, und versetzte er
sich zu diesem Zweck plötzlich in ein Hotelzimmer im Westen
irgend einer Großstadt, auch der liebe Gott hätte hier nur
zwischen zwei Dingen die Wahl\dopp{} entweder auf die Straße
rennen, um sich mit irgendeiner Predigt lächerlich zu machen,
oder telefonieren. Lea Herse telefonierte.

Nach dem Auspacken der Koffer war sie allmählich in den
kleinen Champagnerschwips der Ankunft hineingeschliddert.
Der Chauffeur hatte vor Freude über ihr Trinkgeld fast geheult
und der Portier hatte sie begrüßt wie ein Ehrenmitglied des
Geheimbundes \aa{}Ging-Gang-Genkerlein and Company\ae{}. Die
Liftjungen waren wie die Ziegenböckchen vom letzten April"-%
wurf um sie herumgesprungen und die Zimmerzofe, als sie
versicherte\dopp{} \aa{}Die Toilette ist im Gang schräg vis-a-vis ganz
hinten links um die Ecke\ae{}, hatte deutlich gelächelt\dopp{} \aa{}Das
kommt für Dich Reh aus der Glonn natürlich gar nicht in
Frage\ausr{}\ae{} Jetzt trug sie ihr Lichtgrünes und telefonierte. Des
Jahrhunderts dicke Bibel lag aufgeschlagn von Par bis Paw
vor ihr, die Nummer war schnell gefunden.

Zentrale Hotel, Amt, Zentrale Pasternak, Herr Professor
ist noch nicht im Haus, tüt -- tüt -- tüt.

Gottseidank. Das war ein Fingerzeig des Himmels. War hätte
sie sagen sollen, wenn sie jetzt plötzlich seine Stimme gehört
hätte\frag{} Erst in der Sekunde, da die kühle Muschel an ihrem
% Seite 44
Ohr gelegen war, hatte sie die Gefahr erkannt. Sie war eine
Träumerin, jawohl, noch immer viel zu tief im Traum der
Glonn versunken. Viele Träume der letzten zwei Wochen
waren ausgeträumt und abgestoßen, doch auch die kühlsten
Pläne der Glonn erwiesen sich jetzt plötzlich immer noch als
trübe Träumerei -- wo steckte die kalte Wirklichkeit, nach der
diese kalte Muschel verlangte\frag{}\label{lS44-1}

Ein gewaltiger Garten, Pflanzen und Tiere und ein paar
Menschen, so rollte die große Kugel um die Sonne. Aus dem
Wald trat ein Mann und schaute über die Ebene. Wer bist
Du, sprach er zu dem Kind am geschlängelten Bach, bist Du
nicht von meinem Stoff, damit wir Hand in Hand marschieren
können über dieses herrliche sonnenbestrahlte Reich\eingriff{S44-1}{Reich\punkte{}
] Reich~\punkte{}}\punkte{} Oder eine Landstraße, weithin zwischen den Dörfern und
Märkten, die Handwerker zu Fuß, auf Schimmeln und Rappen
die Grandseigneurs und ihre Damen. Wohin so allein, mein
Kind\frag{} Meinen Vater suchen, den Herrn vom Pommeranzen"-%
land\ausr{} Ich bin der Herr vom Pommeranzenland, mein Kind,
doch hast du auch ein kleines Muttermal unter der rechten
Brust\frag{} Hier ist es, mein Vater, in der Form einer kauernden
Taube\punkte{} Oder eine Weltstadt, jeder Tag eine dröhnende
Schlacht, jede Nacht ein Siegesfest mit Feuerwerk, gesegnet,
wer dem stolzen Wachstum dient. Dort sitzt der Mann, der
Brot aus Luft zu machen versteht, die Kellner am Bufett
flüstern sich den Namen zu und die Gäste ringsum starren
mit blanken Augen auf ihn, aber einsam sitzt der große Che"-%
miker vor seinem Schweinsbraten mit Kartoffelsalat, bis
endlich ein Ding in Lichtgrün zu ihm tritt. Erinnern Sie Sich%
\eingriff{S44-2}{Sich ] sich}
noch an Botzong-Kamin und Gruttenhütte und Daniela
Oldenkott, Professor Pasternak\frag{} Jeden Tag meines Lebens
% Seite 45
denke ich daran, mein Fräulein. So lassen Sie Sich%
\eingriff{S45-1}{Sich ] sich}%
, aber bitte
nicht sentimental zu werden, ein kleine Geschichte erzählen,
Professor. Kein Wort mehr, meine einzige kleine Tochter,
ich wartete auf diese Stunde\punkte{} Pfui über dies Geträum,
gut für die Glonn, gut für Dichter und Dienstmädchen, wo
steckte die kalte Wirklichkeit, nach der diese kalte Muschel ver"-%
langte\frag{} \aa{}Mein Name ist Lea Herse, ich muß Sie dringend
sprechen, Professor.\ae{} \aanah{}Worum handelt es sich\frag{}\ae{} \aa{}Nicht
um Geld und nicht um Chemie, sondern um eine seltsame familiäre
Sache, ich kann es am Telefon nicht sagen.\ae{} Dann kam er an
die Reihe und gab ihr eine Stunde. Dann traf sie ihn und
überließ sich dem Schicksal. Dann rollte das Schicksal im Zwie"-%
gespräch dahin%
\eingriff{eS45-2}{dahin\punkte{} ] dahin \punkte{}}\punkte{}
Und hier
steckte der Rechenfehler der Glonn\dopp{} im Zwiegespräch. In der Glonn vertraute man noch
auf Zwiegespräche, in der Großstadt fühlte man plötzlich,
daß es nur noch Geschäfte und Monologe gab. Würde sie
nicht einen höchst blamablen Monolog in diese kalte Muschel
sprechen, wenn der Mensch am andern Ende der Welt nicht
die richtige Antwort gab\frag{}

Ach was\ausr{} Selbstverständlich mußte sie ihren Vater noch heute
Aug in Aug sehn\ausr{} Selbstverständlich mußte sich alles andere
von selbst ergeben\ausr{} Wer seine Fahrt vor Beginn zu Ende
denken will, versagt schon beim ersten Schritt. Wie gehst Du
so fröhlich mit tausend Füßen dahin, frug die Heuschrecke den
Tausendfüßler, wie machst Du das\frag{} Und der Tausendfüßler
stoppte und überlegte und kam bis an sein Ende keinen
Schritt mehr voran.

Zentrale Hotel, Amt, Zentrale Pasternak, jawohl, Herr Pro"-%
fessor ist im Haus, er hat eine dringende Sitzung, wird aber
nicht mehr lange dauern, tüt -- tüt -- tüt.

% Seite 46
Eine Sitzung hatte Papi, ei ei\frag{} Vermutlich im Zimmer neben
dem freundlichen Telefonisten\frag{} War man sich also doch schon
bis auf zehn Schritt nahgekommen im unendlichen All\frag{}
Selbstverständlich war es die Einfalt der Glonn, diese plötz"-%
liche Angst, daß es in der Stadt nur noch Monologe und
Geschäfte gäbe. Selbstverständlich gab es auch noch Zwiege"-%
spräche. Reizende Zwiegespräche gab es noch, hin und her und
her und hin, der eine Mensch sagte Bims, da sagte der andere
Mensch Bams, mit Bims-Bams aus einem einzigen Munde
kam keine Menschenseele vom Fleck. Immer hübsch hin und
her im Zwiegespräch, so würde sich auch dieser große Fall
ganz von selbst entwickeln, was konnte viel passieren\ausr{}

Zentrale Hotel, Amt, Zentrale Pasternak, jawohl, eine Mi"-%
nute, sofort --

\aa{}Ja\frag{}\ae{}

Das war eine weibliche Stimme. Der böse Wolf hatte Kreide
gefressen, um die sieben Zicklein zu täuschen.

\aa{}Sie wünschen Herrn Professor selbst\frag{}\ae{}

\aanah{}Wer ist am Apparat\frag{}\ae{}

\aa{}Lea Herse\punkte{}\ae{}

\aa{}Einen Augenblick.\ae{}

Dies also war ein Augenblick. Wer war doch der Mann, vor
dessen Sinn tausend Jahre waren wie ein Augenblick\frag{}

\aa{}Sind Sie noch da\frag{}\ae{} Es war immer noch der Wolf aus dem
Vorzimmer. \aa{}Um was handelt es sich\frag{}\ae{}

\aa{}Ich will Herrn Professor Pasternak sprechen\punkte{}\ae{}

\aa{}Ja in welcher Sache denn\frag{}\ae{}

\aa{}Eine ganz persönliche Sache\punkte{}\ae{}

\aa{}Moment.\ae{}

% Seite 47
Unsinn. Persönliche Sache war Unsinn gewesen. Geschäftliche
Sache hätte es%
\eingriff{eS47-1}{es ] er}
heißen müssen. Es handelte sich um etwas
Chemisches, um irgendein Patent, um die Erfindung, Kaviar
aus Kuhmist zu fabrizieren --

\aa{}Pasternak.\ae{}

\aa{}Professor Pasternak selbst\frag{}\ae{}

\aa{}Ja, wer ist dort\frag{}\ae{}

Sie schwieg. Kein Wort fiel ihr durch den Sinn.

\aa{}Bist Du es, Zilly\frag{}\ae{}

Zilly\frag{} Wer war Zilly\frag{}

\aa{}Halloh\frag{} Was los\frag{}\ae{}

\aa{}Hier ist Miß Dsching-Dschang-Dschenkerlein aus New York\ae{},
sagte sie mit starkem amerikanischem Mauschel-Akzent. \aa{}Ich
bin von die New York-Times befohlen, Herrn Professor
Pasternak um eine kleine Interview zu bitten~--\ae{}

\aanah{}Wer ist das\frag{}\ae{}

\aa{}Dschenkerlein, New York, Times, \fremdsprachlich{photographer and inter"-%
viewer} --\ae{}

\aa{}Ja und\frag{}\ae{}

\aa{}Ich interessieren mich for Ihnen, Professor, weil Sie so
chemisch sind --\ae{}

\aa{}Das ist irgend eine Mystifikation, wer ist denn dort\frag{}\ae{}

\aa{}Dschenkerlein, New York, ich haben eine ähnliche Patent
wie der Herr Professor, ich machen \fremdsprachlich{diamants}\label{lS47-1} aus
Kuhmist --\ae{}

\aa{}Ich verbitte mir Ihre dummen Witze\ausr{}\ae{}

Tüt -- tüt -- tüt.
\abstand{}
% Seite 48
Der junge Mann hieß Bechterew, \fremdsprachlich{Dr.}\label{lS48-1} Fritz
Bechterew, wissenschaftlicher Mitarbeiter einer großen Tageszeitung.
Das Abendessen in dem kleinen russischen Restaurant be"-%
stand aus Borschtsch, Beefsteak, Obstsalat, Wodka. Die Ka"-%
pelle spielte argentinische Tangos, Wiener Walzer, Urwald-%
Krach, Beethoven, \fremdsprachlich{I am lonely without you,} Lieder vom
Rhein.

\aa{}Bitte, sehn Sie mal möglichst unauf"|fällig nach dem dritten
Tisch hinter Ihrem Rücken\ae{}, sagte Bechterew zu Lea. Er
sprach eine Spur Sächsisch, denn er stammte aus Chemnitz,
doch das war nur für geographisch geschulte Ohren zu hören.
Lea Herse nahm ihn für einen amüsanten und gerissenen
Berliner Juden, obwohl er als dumpfer Antisimist und als
englisches Halbblut gelten wollte. Seit einer halben Stunde
analysierte er ihr die Gäste ringsum, ihre Berufe und ihre
Laster, ihre Lächerlichkeit und ihre Tricks, die meisten waren
Freunde oder Bekannte von ihm.

\aa{}Der dritte Tisch hinter Ihnen, der fette kleine Kerl mit der
hübschen blonden Leiche in Schwarz. Was, glauben Sie, ist
der Mann\frag{}\ae{}

\aanah{}Ach, ich will es gar nicht wissen\ae{}, sagte Lea, ohne sich zu
wenden. \aa{}Ich hab schon genug. Halte ich ihn für einen Dichter,
dann ist er gewiß ein Lustmörder, und umgekehrt\ausr{}\ae{}

Bechterew lachte. \aa{}Erraten. Er ist nämlich beides, Dichter
und Lustmörder in einer Person\dopp{} berühmter Kritiker. Martin
Frogge, guter Freund von mir, an den gleichen Zeitungstrust
versklavt wie ich. Wollen Sie ihn kennenlernen\frag{}\ae{}

\aa{}Nein, danke.\ae{}

\aanah{}Verzeihn Sie, wenn ich ihn begrüße, eine Minute nur\frag{}\ae{}

\aa{}Bittebittebitte --\ae{}

% Seite 49
Sie war froh, eine kleine Weile allein zu sein. Der erste Tag
der Expedition ging zu Ende, der erste Ansturm war abge"-%
schlagen, doch der Tag war nicht verloren, morgen würde ihr
das Leben in der Großstadt sicherer von der Hand gehn als
heut. Sehr gut, daß sie unter Führung des Herrn Bechterew
ein paar Stunden auf dem Asphalt herumgerutscht war, ehe
sie den Kampf mit ihrem Vater antrat. Sehr gut, daß sie sich
von einem Fremden auf der Straße hatte ansprechen lassen,
zu einem kleinen Bummel durch die große Welt, die sie nur
aus Büchern und Zeitungen kannte, Straße, Kaufhaus
Kino, Restaurant. Bei dem Blinden mit dem Hund war sie
von Herrn Bechterew angesprochen worden. Er hatte be"-%
hauptet, schon eine ganze Stunde hinter ihr hergetrabt zu
sein, verliebt in ihren Gang, und verlegen, weil er sich noch
niemals einer Dame auf der Straße angehängt hätte, erst die
gemeinsame Ergriffenheit vor dem blinden Bettler mit dem
Hund hätte den Bann gebrochen: doch das war sicher Lüge
gewesen. Alles war Lüge ringsum. Aber man mußte diese
Lüge fest ins Auge fassen, um die Wahrheit, die dahinter steckte,
zu packen.

Die Musik spielte einen amerikanischen Marsch. Der Refrain
wurde von allen Nichtbläsern der Kapelle mitgesungen. Lea
geriet in eine heroische Stimmung bei diesen ungewohnten
schrillen Klängen. Wunderbar war das Leben.

