%% Erster Teil.

% Seite 43
\kapitel{Zweites Kapitel}

\noindent{}%
Der liebe Gott, nähme er wieder einmal Menschengestalt an,
um unter seinen Erdenkindern zu wandeln, und versetzte er
sich zu diesem Zweck plötzlich in ein Hotelzimmer im Westen
irgend einer Großstadt, auch der liebe Gott hätte hier nur
zwischen zwei Dingen die Wahl\dopp{} entweder auf die Straße
rennen, um sich mit irgendeiner Predigt lächerlich zu machen,
oder telefonieren. Lea Herse telefonierte.

Nach dem Auspacken der Koffer war sie allmählich in den
kleinen Champagnerschwips der Ankunft hineingeschliddert.
Der Chauffeur hatte vor Freude über ihr Trinkgeld fast geheult
und der Portier hatte sie begrüßt wie ein Ehrenmitglied des
Geheimbundes \aa{}Ging-Gang-Genkerlein and Company\ae{}. Die
Liftjungen waren wie die Ziegenböckchen vom letzten April\-%
wurf um sie herumgesprungen und die Zimmerzofe, als sie
versicherte\dopp{} \aa{}Die Toilette ist im Gang schräg vis-a-vis ganz
hinten links um die Ecke\ae{}, hatte deutlich gelächelt\dopp{} \aa{}Das
kommt für Dich Reh aus der Glonn natürlich gar nicht in
Frage\ausr{}\ae{} Jetzt trug sie ihr Lichtgrünes und telefonierte. Des
Jahrhunderts dicke Bibel lag aufgeschlagn von Par bis Paw
vor ihr, die Nummer war schnell gefunden.

Zentrale Hotel, Amt, Zentrale Pasternak, Herr Professor
ist noch nicht im Haus, tüt -- tüt -- tüt.

Gottseidank. Das war ein Fingerzeig des Himmels. War hätte
sie sagen sollen, wenn sie jetzt plötzlich seine Stimme gehört
hätte\frag{} Erst in der Sekunde, da die kühle Muschel an ihrem
% Seite 44
Ohr gelegen war, hatte sie die Gefahr erkannt. Sie war eine
Träumerin, jawohl, noch immer viel zu tief im Traum der
Glonn versunken. Viele Träume der letzten zwei Wochen
waren ausgeträumt und abgestoßen, doch auch die kühlsten
Pläne der Glonn erwiesen sich jetzt plötzlich immer noch als
trübe Träumerei -- wo steckte die kalte Wirklichkeit, nach der
diese kalte Muschel verlangte\frag{}

Ein gewaltiger Garten, Pflanzen und Tiere und ein paar
Menschen, so rollte die große Kugel um die Sonne. Aus dem
Wald trat ein Mann und schaute über die Ebene. Wer bist
Du, sprach er zu dem Kind am geschlängelten Bach, bist Du
nicht von meinem Stoff, damit wir Hand in Hand marschieren
können über dieses herrliche sonnenbestrahlte Reich~\punkte{}
Oder eine Landstraße, weithin zwischen den Dörfern und
Märkten, die Handwerker zu Fuß, auf Schimmeln und Rappen
die Grandseigneurs und ihre Damen. Wohin so allein, mein
Kind\frag{} Meinen Vater suchen, den Herrn vom Pommeranzen\-%
land\ausr{} Ich bin der Herr vom Pommeranzenland, mein Kind,
doch hast du auch ein kleines Muttermal unter der rechten
Brust\frag{} Hier ist es, mein Vater, in der Form einer kauernden
Taube~\eingriff{Taube~\punkte{} Oder ] Taube\punkte{} Oder}\punkte{}
Oder eine Weltstadt, jeder Tag eine dröhnende
Schlacht, jede Nacht ein Siegesfest mit Feuerwerk, gesegnet,
wer dem stolzen Wachstum dient. Dort sitzt der Mann, der
Brot aus Luft zu machen versteht, die Kellner am Bufett
flüstern sich den Namen zu und die Gäste ringsum starren
mit blanken Augen auf ihn, aber einsam sitzt der große Che\-%
miker vor seinem Schweinsbraten mit Kartoffelsalat, bis
endlich ein Ding in Lichtgrün zu ihm tritt. Erinnern Sie Sich%
\eingriff{Sie Sich ] Sie sich}
noch an Botzong-Kamin und Gruttenhütte und Daniela
Oldenkott, Professor Pasternak\frag{} Jeden Tag meines Lebens
% Seite 45
denke ich daran, mein Fräulein.