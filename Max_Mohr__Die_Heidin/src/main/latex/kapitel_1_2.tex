%% Erster Teil.

% Seite 43
\kapitel{Zweites Kapitel}

Der liebe Gott, nähme er wieder einmal Menschengestalt an,
um unter seinen Erdenkindern zu wandeln, und versetzte er
sich zu diesem Zweck plötzlich in ein Hotelzimmer im Westen
irgend einer Großstadt, auch der liebe Gott hätte hier nur
zwischen zwei Dingen die Wahl\dopp{} entweder auf die Straße
rennen, um sich mit irgendeiner Predigt lächerlich zu machen,
oder telefonieren. Lea Herse telefonierte.

Nach dem Auspacken der Koffer war sie allmählich in den
kleinen Champagnerschwips der Ankunft hineingeschliddert.
Der Chauffeur hatte vor Freude über ihr Trinkgeld fast geheult
und der Portier hatte sie begrüßt wie ein Ehrenmitglied des
Geheimbundes \aa{}Ging-Gang-Genkerlein and Company\ae{}. Die
Liftjungen waren wie die Ziegenböckchen vom letzten April"-%
wurf um sie herumgesprungen und die Zimmerzofe, als sie
versicherte\dopp{} \aa{}Die Toilette ist im Gang schräg vis-a-vis ganz
hinten links um die Ecke\ae{}, hatte deutlich gelächelt\dopp{} \aa{}Das
kommt für Dich Reh aus der Glonn natürlich gar nicht in
Frage\ausr{}\ae{} Jetzt trug sie ihr Lichtgrünes und telefonierte. Des
Jahrhunderts dicke Bibel lag aufgeschlagn von Par bis Paw
vor ihr, die Nummer war schnell gefunden.

Zentrale Hotel, Amt, Zentrale Pasternak, Herr Professor
ist noch nicht im Haus, tüt -- tüt -- tüt.

Gottseidank. Das war ein Fingerzeig des Himmels. Was hätte
sie sagen sollen, wenn sie jetzt plötzlich seine Stimme gehört
hätte\frag{} Erst in der Sekunde, da die kühle Muschel an ihrem
% Seite 44
Ohr gelegen war, hatte sie die Gefahr erkannt. Sie war eine
Träumerin, jawohl, noch immer viel zu tief im Traum der
Glonn versunken. Viele Träume der letzten zwei Wochen
waren ausgeträumt und abgestoßen, doch auch die kühlsten
Pläne der Glonn erwiesen sich jetzt plötzlich immer noch als
trübe Träumerei -- wo steckte die kalte Wirklichkeit, nach der
diese kalte Muschel verlangte\frag{}

Ein gewaltiger Garten, Pflanzen und Tiere und ein paar
Menschen, so rollte die große Kugel um die Sonne. Aus dem
Wald trat ein Mann und schaute über die Ebene. Wer bist
Du, sprach er zu dem Kind am geschlängelten Bach, bist Du
nicht von meinem Stoff, damit wir Hand in Hand marschieren
können über dieses herrliche sonnenbestrahlte Reich\punkte{}\eingriff{eS44-1}{Reich\punkte{} ] Reich~\punkte{}}
Oder eine Landstraße, weithin zwischen den Dörfern und
Märkten, die Handwerker zu Fuß, auf Schimmeln und Rappen
die Grandseigneurs und ihre Damen. Wohin so allein, mein
Kind\frag{} Meinen Vater suchen, den Herrn vom Pommeranzen"-%
land\ausr{} Ich bin der Herr vom Pommeranzenland, mein Kind,
doch hast Du auch ein kleines Muttermal unter der rechten
Brust\frag{} Hier ist es, mein Vater, in der Form einer kauernden
Taube\punkte{} Oder eine Weltstadt, jeder Tag eine dröhnende
Schlacht, jede Nacht ein Siegesfest mit Feuerwerk, gesegnet,
wer dem stolzen Wachstum dient. Dort sitzt der Mann, der
Brot aus Luft zu machen versteht, die Kellner am Bufett
flüstern sich den Namen zu und die Gäste ringsum starren
mit blanken Augen auf ihn, aber einsam sitzt der große Che"-%
miker vor seinem Schweinsbraten mit Kartoffelsalat, bis
endlich ein Ding in Lichtgrün zu ihm tritt. Erinnern Sie Sich\eingriff{eS44-2}{Sich ] sich}
noch an Botzong-Kamin und Gruttenhütte und Daniela
Oldenkott, Professor Pasternak\frag{} Jeden Tag meines Lebens
% Seite 45
denke ich daran, mein Fräulein. So lassen Sie Sich\eingriff{eS45-1}{Sich ] sich}, aber bitte
nicht sentimental zu werden, eine kleine Geschichte erzählen,
Professor. Kein Wort mehr, meine einzige kleine Tochter,
ich wartete auf diese Stunde\punkte{} Pfui über dies Geträum,
gut für die Glonn, gut für Dichter und Dienstmädchen, wo
steckte die kalte Wirklichkeit, nach der diese kalte Muschel ver"-%
langte\frag{} \aa{}Mein Name ist Lea Herse, ich muß Sie dringend
sprechen, Professor.\ae{} \aanah{}Worum handelt es sich\frag{}\ae{} \aa{}Nicht um
Geld und nicht um Chemie, sondern um eine seltsame familiäre
Sache, ich kann es am Telefon nicht sagen.\ae{} Dann kam er an
die Reihe und gab ihr eine Stunde. Dann traf sie ihn und
überließ sich dem Schicksal. Dann rollte das Schicksal im Zwie"-%
gespräch dahin\punkte{}\eingriff{eS45-2}{dahin\punkte{} ] dahin \punkte{}} Und hier steckte der Rechenfehler der
Glonn\dopp{} im Zwiegespräch. In der Glonn vertraute man noch
auf Zwiegespräche, in der Großstadt fühlte man plötzlich,
daß es nur noch Geschäfte und Monologe gab. Würde sie
nicht einen höchst blamablen Monolog in diese kalte Muschel
sprechen, wenn der Mensch am andern Ende der Welt nicht
die richtige Antwort gab\frag{}

Ach was\ausr{} Selbstverständlich mußte sie ihren Vater noch heute
Aug in Aug sehn\ausr{} Selbstverständlich mußte sich alles andere
von selbst ergeben\ausr{} Wer seine Fahrt vor Beginn zu Ende
denken will, versagt schon beim ersten Schritt. Wie gehst Du
so fröhlich mit tausend Füßen dahin, frug die Heuschrecke den
Tausendfüßler, wie machst Du das\frag{} Und der Tausendfüßler
stoppte und überlegte und kam bis an sein Ende keinen
Schritt mehr voran.

Zentrale Hotel, Amt, Zentrale Pasternak, jawohl, Herr Pro"-%
fessor ist im Haus, er hat eine dringende Sitzung, wird aber
nicht mehr lange dauern, tüt -- tüt -- tüt.

% Seite 46
Eine Sitzung hatte Papi, ei ei\frag{} Vermutlich im Zimmer neben
dem freundlichen Telefonisten\frag{} War man sich also doch schon
bis auf zehn Schritt nahgekommen im unendlichen All\frag{}
Selbstverständlich war es die Einfalt der Glonn, diese plötz"-%
liche Angst, daß es in der Stadt nur noch Monologe und
Geschäfte gäbe. Selbstverständlich gab es auch noch Zwiege"-%
spräche. Reizende Zwiegespräche gab es noch, hin und her und
her und hin, der eine Mensch sagte Bims, da sagte der andere
Mensch Bams, mit Bims-Bams aus einem einzigen Munde
kam keine Menschenseele vom Fleck. Immer hübsch hin und
her im Zwiegespräch, so würde sich auch dieser große Fall
ganz von selbst entwickeln, was konnte viel passieren\ausr{}

Zentrale Hotel, Amt, Zentrale Pasternak, jawohl, eine Mi"-%
nute, sofort --

\aa{}Ja\frag{}\ae{}

Das war eine weibliche Stimme. Der böse Wolf hatte Kreide
gefressen, um die sieben Zicklein zu täuschen.

\aa{}Sie wünschen Herrn Professor selbst\frag{}\ae{}

\aa{}Ja natürlich\punkte{}\ae{}

\aanah{}Wer ist am Apparat\frag{}\ae{}

\aa{}Lea Herse\punkte{}\ae{}

\aa{}Einen Augenblick.\ae{}

Dies also war ein Augenblick. Wer war doch der Mann, vor
dessen Sinn tausend Jahre waren wie ein Augenblick\frag{}

\aa{}Sind Sie noch da\frag{}\ae{} Es war immer noch der Wolf aus dem
Vorzimmer. \aa{}Um was handelt es sich\frag{}\ae{}

\aa{}Ich will Herrn Professor Pasternak sprechen\punkte{}\ae{}

\aa{}Ja in welcher Sache denn\frag{}\ae{}

\aa{}Eine ganz persönliche Sache\punkte{}\ae{}

\aa{}Moment.\ae{}

% Seite 47
Unsinn. Persönliche Sache war Unsinn gewesen. Geschäftliche
Sache hätte es\eingriff{eS47-1}{es ] er} heißen müssen. Es handelte sich um etwas
Chemisches, um irgendein Patent, um die Erfindung, Kaviar
aus Kuhmist zu fabrizieren --

\aa{}Pasternak.\ae{}

\aa{}Professor Pasternak selbst\frag{}\ae{}

\aa{}Ja, wer ist dort\frag{}\ae{}

Sie schwieg. Kein Wort fiel ihr durch den Sinn.

\aa{}Bist Du es, Zilly\frag{}\ae{}

Zilly\frag{} Wer war Zilly\frag{}

\aa{}Halloh\frag{} Was los\frag{}\ae{}

\aa{}Hier ist Miß Dsching-Dschang-Dschenkerlein aus New York\ae{},
sagte sie mit starkem amerikanischem Mauschel-Akzent. \aa{}Ich
bin von die New York-Times befohlen, Herrn Professor
Pasternak um eine kleine Interview zu bitten~--\ae{}

\aanah{}Wer ist das\frag{}\ae{}

\aa{}Dschenkerlein, New York, Times, \fremdsprachlich{photographer and inter"-%
viewer} --\ae{}

\aa{}Ja und\frag{}\ae{}

\aa{}Ich interessieren mich for Ihnen, Professor, weil Sie so
chemisch sind --\ae{}

\aa{}Das ist irgend eine Mystifikation, wer ist denn dort\frag{}\ae{}

\aa{}Dschenkerlein, New York, ich haben eine ähnliche Patent
wie der Herr Professor, ich machen \fremdsprachlich{diamants}\label{lS47-1} aus Kuhmist --\ae{}

\aa{}Ich verbitte mir Ihre dummen Witze\ausr{}\ae{}

Tüt -- tüt -- tüt.
\abstand{}
% Seite 48
Der junge Mann hieß Bechterew, \fremdsprachlich{Dr.}\label{lS48-1} Fritz Bechterew,
wissenschaftlicher Mitarbeiter einer großen Tageszeitung.
Das Abendessen in dem kleinen russischen Restaurant be"-%
stand aus Borschtsch, Beefsteak, Obstsalat, Wodka. Die Ka"-%
pelle spielte argentinische Tangos, Wiener Walzer, Urwald-%
Krach, Beethoven, \fremdsprachlich{I am lonely without you,} Lieder vom
Rhein.

\aa{}Bitte, sehn Sie mal möglichst unauf"|fällig nach dem dritten
Tisch hinter Ihrem Rücken\ae{}, sagte Bechterew zu Lea. Er
sprach eine Spur Sächsisch, denn er stammte aus Chemnitz,
doch das war nur für geographisch geschulte Ohren zu hören.
Lea Herse nahm ihn für einen amüsanten und gerissenen
Berliner Juden, obwohl er als dumpfer Antisimit und als
englisches Halbblut gelten wollte. Seit einer halben Stunde
analysierte er ihr die Gäste ringsum, ihre Berufe und ihre
Laster, ihre Lächerlichkeit und ihre Tricks, die meisten waren
Freunde oder Bekannte von ihm.

\aa{}Der dritte Tisch hinter Ihnen, der fette kleine Kerl mit der
hübschen blonden Leiche in Schwarz. Was, glauben Sie, ist
der Mann\frag{}\ae{}

\aanah{}Ach, ich will es gar nicht wissen\ae{}, sagte Lea, ohne sich zu
wenden. \aa{}Ich hab schon genug. Halte ich ihn für einen Dichter,
dann ist er gewiß ein Lustmörder, und umgekehrt\ausr{}\ae{}

Bechterew lachte. \aa{}Erraten. Er ist nämlich beides, Dichter
und Lustmörder in einer Person\dopp{} berühmter Kritiker. Martin
Frogge, guter Freund von mir, an den gleichen Zeitungstrust
versklavt wie ich. Wollen Sie ihn kennenlernen\frag{}\ae{}

\aa{}Nein, danke.\ae{}

\aanah{}Verzeihn Sie, wenn ich ihn begrüße, eine Minute nur\frag{}\ae{}

\aa{}Bittebittebitte --\ae{}

% Seite 49
Sie war froh, eine kleine Weile allein zu sein. Der erste Tag
der Expedition ging zu Ende, der erste Ansturm war abge"-%
schlagen, doch der Tag war nicht verloren, morgen würde ihr
das Leben in der Großstadt sicherer von der Hand gehn als
heut. Sehr gut, daß sie unter Führung des Herrn Bechterew
ein paar Stunden auf dem Asphalt herumgerutscht war, ehe
sie den Kampf mit ihrem Vater antrat. Sehr gut, daß sie sich
von einem Fremden auf der Straße hatte ansprechen lassen,
zu einem kleinen Bummel durch die große Welt, die sie nur
aus Büchern und Zeitungen kannte, Straße, Kaufhaus,
Kino, Restaurant. Bei dem Blinden mit dem Hund war sie
von Herrn Bechterew angesprochen worden. Er hatte be"-%
hauptet, schon eine ganze Stunde hinter ihr hergetrabt zu
sein, verliebt in ihren Gang, und verlegen, weil er sich noch
niemals einer Dame auf der Straße angehängt hätte, erst die
gemeinsame Ergriffenheit vor dem blinden\eingriff{eS49-1}{blinden ] blindem} Bettler mit dem
Hund hätte den Bann gebrochen\dopp{} doch das war sicher Lüge
gewesen. Alles war Lüge ringsum. Aber man mußte diese
Lüge fest ins Auge fassen, um die Wahrheit, die dahinter steckte,
zu packen.

Die Musik spielte einen amerikanischen Marsch. Der Refrain
wurde von allen Nichtbläsern der Kapelle mitgesungen. Lea
geriet in eine heroische Stimmung bei diesen ungewohnten
schrillen Klängen. Wunderbar war das Leben.

Wunderbar war das Leben, wenn man Augen hatte zu
schauen und Ohren zu hören und ein Herz sich zu entscheiden.
Am Rand der steilsten Glonner Wiese lag Frau Herse be"-%
graben, nach einem blutigen Kampf mit den Behörden war
es gelungen, ein Grab außerhalb des Kirchhofs graben zu
dürfen, ein Grab auf eigenem Boden. Fast hätte Lea in letzter
% Seite 50
Stunde den Kampf aufgegeben und den Leichnam an Staat
und Kirche ausgeliefert, schließlich war dieser heidnische
Wunsch ihrer Mutter nur eine Sentimentalität gewesen und
im Tod war alles so gleichgültig. Jetzt kam eine tiefe Freude
über sie, daß sie den Kampf gegen die Beamten siegreich
durchgefochten und die Mutter in der Erde der Glonn einge"-%
bettet hatte.

Ihr verdankte sie es, daß sie ein Herz hatte, sich zu entscheiden.
Nicht aus Weltangst hatte sich Frau Daniela in die Glonn
zurückgezogen, sondern in blanker Entschiedenheit. Sie hatte
sich entschieden, entschieden, entschieden, das war es. Leicht
möglich, daß sie sich für die falsche Seite des Lebens entschie"-%
den hatte, vielleicht mußte man sich in das Chaos stürzen,
mittenhinein, um eingetragen zu werden in das Gästebuch
des Weltgeists\dopp{} aber sie hatte sich wenigstens entschieden, sie
hatte nicht geflunkert wie diese Zeitgenossen ringsum, wie
diese Hin-und-her-Pendler zwischen Gut und Bös. Frau
Daniela hatte sich entschieden, das war die große Erinnerung,
die ihre Tochter ihr an diesem Abend weihte. So oder so,
Glonn oder Weltstadt, tierischer Dünger oder Chemie, es gab
keinen Mittelweg, das war die Erkenntnis ihres ersten Groß"-%
stadt-Tags, das war das heroische Gefühl bei diesen schrillen
amerikanischen Synkopen. Man mußte sich zwischen zwei
Welten entscheiden und es war heroisch, sich entscheiden zu
müssen. Wunderbar war das Leben, wie auch die Entscheidung
in den nächsten Tagen fiel, wenn sie nur fiel.

\aa{}Bitte, Frogge, gehn Sie doch mal möglichst unauf"|fällig an
meiner Lady vorbei\ae{}, sagte unterdessen Bechterew zu dem klei"-%
nen fetten Herrn am dritten Tisch hinter Leas Tisch. \aa{}Ein toller
Fall\ausr{} Möchte furchtbar gern wissen, was Sie von ihr halten.\ae{}

% Seite 51
\aa{}Mensch, löchern Sie mich nicht mit Ihren ewigen Weiber"-%
geschichten -- na, was ist denn, Ober, wo sind denn meine
Muscheln, eine halbe Stunde warte ich schon, bei Euch geht
das Geschäft wieder mal viel zu gut --\ae{}

\aa{}Nein wirklich, Frogge, ein toller Fall. Das Mädchen ist heute
zum ersten mal in ihrem Leben in der Großstadt, die richtige
Unschuld vom Lande, aber klüger als wir alle zusammen.\ae{}

\aa{}Das muß ich mir begucken\ae{}, sagte Martin Frogges Freundin.
Sie hieß Lizzy von Save und war aus Düsseldorf importiert,
um rhythmische Tanzkunst zu studieren.

\aa{}Guck sie Dir an, die Unschuld aus der Dragonerstraße\ae{},
sagte Frogge und ballte wütend die Faust gegen den Kellner,
der am Nachbartisch servierte.

\aa{}Kostet zehn Pfennig\ae{}, sagte die rhythmische Tänzerin und
hielt Bechterew die offene Hand hin.

Bechterew gab ihr ein Markstück. Sie schritt mit einem
scharfen Seitenblick auf Lea rhythmisch zur Toilette.

\aa{}Hören Sie mal, Bechterew\ae{}, sagte Martin Frogge, \aa{}Ihr
gestriger Artikel über die Lehre vom Yoga war aber richtiger
Schmus. Glauben Sie wirklich an dieses Zeug\frag{}\ae{}

\aa{}Ich schwöre Ihnen, Frogge, es ist die einzige Religion, an
die ich noch glauben kann.\ae{}

Martin Frogge bekam seine Muscheln und stürzte sich wie ein
trojanischer Held darüber her.

\aanah{}Tatsächlich, Frogge\ausr{} Der Mensch lebt zwischen dem Zentrum
der Erdkugel und dem Zentrum der Sonnenkugel. Wenn der
Mensch in dem Gefühl lebt, die Verbindung zwischen dem
Erdmittelpunkt und dem Sonnenball zu sein, dann hat er
sein wahres Lebensgefühl gefunden.\ae{}

% Seite 52
\aa{}Und wenn er es nicht hat\frag{}\ae{}

\aa{}Dann ist er tot, geistig und seelisch tot, der körperliche Tod
spielt dabei gar keine Rolle.\ae{}

\aa{}Und was bekommen Sie für so einen Aufsatz gezahlt\frag{}\ae{}

\aa{}Das ist Geschäftsgeheimnis.\ae{}

\aanah{}Auf jeden Fall viel zu viel.\ae{}

\aanah{}Auf jeden Fall viel zu wenig, Sie kennen ja unsere Blut"-%
sauger. Aber das Tolle ist, daß ich gerade heute dieses
Mädchen dort treffe, sie hat nämlich wirklich jenes uralte und
doch so moderne Lebensgefühl, von dem ich in meinem
Aufsatz schrieb. Ehrenwort, Frogge, ich spüre es bei ihr ganz
genau.\ae{}

\aanah{}Toller Fall\ae{}, sagte Martin Frogge und legte sich lachend eine
neue Portion Muscheln vor. \aa{}Na was ist, Lizzy\frag{} Ist sie Yoga
oder ist sie Yogurt\frag{}\ae{}

Die rhythmische Tänzerin zählte schweigend neun Zehn"-%
pfennigstücke auf den Tisch.

\aa{}Na, was ist\frag{}\ae{},\eingriff{eS52-1}{ist\frag{}\ae{}, ] ist\frag{}\ae{}} frug Bechterew gespannt und schob das
Kleingeld in die Hosentasche.

\aa{}Schlagen Sie Sich\eingriff{eS52-2}{Sich ] sich} diese Dame aus dem Kopf, Bechterew"-%
chen\ausr{}\ae{}

\aa{}Ist sie nicht hübsch\frag{}\ae{}

\aa{}Hübsch ist sie, sehr hübsch, aber --\ae{} Sie winkte Bechterews
gekräuseltes Haupt zu sich und flüsterte ihm mit gespitztem,
aristokratischem Mäulchen ihre Diagnose ins Ohr.

Martin Frogge brach in Gelächter aus, ohne hinzuhören.
\aa{}Na eben, na was denn, na selbstverständlich\ausr{} Das seh
ich doch von hier aus an ihrem knabenhaften Schulter"-%
geblätter\ausr{}\ae{}

% Seite 53
Er trank strahlend dem zusammengeknickten, wissenschaft"-%
lichen Mitarbeiter seiner Zeitung zu. \aa{}Hoch Yoga, Bechterew\ausr{}
Hoch Yoga, Berlin, Chemnitz und Lesbos\ausr{}\ae{}
\abstand{}
Die Uhr auf dem Nachttisch war stehn geblieben. In der
Glonn wußte man beim Erwachen, wie spät es war, vor
allem im Juni. Dort brauchte man nicht erst die Augen zu
öffnen, um nach dem Licht des Tags zu sehn. Drei Uhr
und es piepsten die ersten Vögel, vier Uhr und der Krallen"-%
peter begann die Sense zu dengeln, fünf Uhr und die alte
Nana wälzte sich mit Krach aus ihrer geblümten Kiste, sechs
Uhr und das Konzert des Tags stand auf fortissimo. Hier
wußte man nichts. Totenstille ringsum. Besaß man Geld,
so konnte man sich mitten in der Weltstadt eine Portion
Totenstille kaufen. Totenstille mit Luxussteuer. Und prunk"-%
voll sanken die samtenen Vorhänge des Fensters zum Par"-%
kett herab und garantierten das Dunkel, das Totendunkel
mit Luxussteuer.

Um elf Uhr wollte Herr Bechterew anrufen. Es handelte
sich um eine chinesische Ausstellung, die man unbedingt gesehn
haben mußte. Aber Herr Bechterew war zum Schluß etwas
abgekühlt gewesen, das liebe Halbblut. War er eifersüchtig
gewesen, weil sie sich von Herrn Martin Frogge Theater"-%
karten für heute Abend besorgen lassen wollte\frag{} Nun hatte
sie sich doch noch in Herrn Martin Frogges Gesellschaft ziehn
lassen\ausr{} Um zwei Uhr sollte sie Fräulein Lizzy von Save
treffen, der Name des Lunch-rooms stand auf dem kleinen
Notizblock, viele interessante und rhythmische Menschen und
% Seite 54
sehr billige Knödel mit Kraut gab es in jenem Lunch"-%
room. Dann mußte endlich das Telegramm mit der Adresse
und der guten Ankunft an Terese Nüll und Nana abgesandt
werden. Dann mußte anstandshalber eine alte Tante
Oldenkott besucht werden, die in Berlin wohnte. Bechterew
schätzte die Fahrt bis zur Wohnung dieser alten Tante auf
mindestens eine Stunde, mit der besten Verbindung eine
geschlagene Stunde, alle alten Tanten wohnten an der
Peripherie.

Sie sprang aus dem Bett und beschloß, nach allen Seiten
wortbrüchig zu werden. Nach dem Stand der Sonne war es
schon elf Uhr, sie telefonierte in die Zentrale des Hotels, daß
sie nicht zu sprechen wäre. Herr Bechterew sollte sich bei dem
Blinden mit dem Hund eine andere Ergriffene aufgabeln,
Fräulein von Save sollte ihre Lunch-room-Knödel allein
verzehren, Herr Frogge sollte die Theaterkarten für das Nackt"-%
ballett an Tante Oldenkott schicken, die Arme war siebzig und
hatte gewiß schon lange nichts Nacktes mehr gesehn. Neun
Telefon-Nummern standen schon auf dem kleinen Notizblock\dopp{}
morgen waren es neunzig, wenn sich pro Kopf zehn neue Be"-%
ziehungen auftaten. In einem Monat ging es ihr wie Mister
Dschenkerlein aus New York\dopp{} der hatte zehn Jahre lang
ununterbrochen telefoniert, und als man ihm dann den Kopf"-%
hörer mit Gewalt abriß, krochen ihm große bleiche Würmer
aus den Ohren, unzählige Würmer, immerzu Würmer, bis
er endlich den letzten Wurm von sich gab und starb.

Aber gesegnet sei das Klima dieser Stadt\ausr{} Eine herrliche Luft
drang durch das offene Fenster, als die Vorhänge zurück"-%
gewürgt waren. Wie ein Nichts wurde der schlechte Atem der
Millionen Lungen vom Himmel aufgeschluckt und in reine
% Seite 55
Ware umgetauscht, von Herzschlag zu Herzschlag ein ewiger
Vorrat an Reinheit und Macht.

Gesegnet diese Luft und gesegnet dieses Badezimmer\ausr{} Bei
Kalt allein konnte man die Brause nicht sehr lange ertragen,
aber bei einer winzigen Mischung mit Warm konnte man ver"-%
weilen. Fing man den Strahl im Nacken auf, so stob der
Wassergott in dunklem Zorn das harte Flußbett der Wirbel"-%
säule hinunter. Doch bog man sich zurück, so glitt er sanft und
hell über das kleine weiche Vordach der Brüste und umspielte
mit zarten Rinnsalen den Bauch und seine Säulen.

Schluß, Frottage, Wäsche, Kleid, Schuhlitzen-Abreißen, Hut,
Frühstück, Taxi, helle Straßen, schnelle Menschen, Portier,
Vorhalle, Lift, zweiter Stock, Anmeldung.

Auf dem Zettel, den der kleine uniformierte Bengel ihr
vorlegte, stand gedruckt\dopp{} \aa{}Frau oder Herr~\punkte{}\ldots{}\eingriff{eS55-1}{Herr~\punkte{}\ldots{} ] Herr\punkte{}\ldots{}} wünscht
Frau oder Herrn~\punkte{}\ldots{}\eingriff{eS55-2}{Herrn~\punkte{}\ldots{} ] Herrn\punkte{}\ldots{}} in Angelegenheit~\punkte{}\ldots{}\eingriff{eS55-3}{Angelegenheit~\punkte{}\ldots{} ] Angelegenheit\punkte{}\ldots{}} zu sprechen.\ae{}
Natürlich wünschte Frau Herse Herrn Pasternak zu sprechen,
doch in welcher Angelegenheit\frag{} Angelegenheit Gruttenhütte\frag{}
Angelegenheit 29.~September 1901\frag{} Angelegenheit \aa{}Eine-%
Waise-im-Sturm-der-Zeit-sucht-ihren-väterlichen-Felsen\frag{}\ae{}

Warum hatte sie den schönen Brief zerrissen, den sie ihm
gestern geschrieben hatte\frag{} Warum wollte sie ihn durchaus
Aug in Aug sehn, bevor sie ihr Geheimnis preisgab\frag{} Wozu
all diese Schwierigkeiten\frag{}

Neben ihr stand ein blutarmer Familienvater und schrieb
ohne Besinnen seine Angelegenheit auf seinen Zettel\dopp{} \aa{}Herr
Meier~1 wünscht Herrn Maier~2 in Angelegenheit Habe-%
nichts-zu-fressen zu sprechen.\ae{} Der uniformierte Bengel be"-%
gann sie bereits mißtrauisch abzuschätzen, weil sie ihre An"-%
gelegenheit nicht wußte. Ja, mein Junge, Deine Angelegen"-%
% Seite 56
heit ist klar, Du bist ein kleiner Maulaufreißer und wirst
mal ein großer Maulaufreißer werden, bei mir liegt die
Angelegenheit etwas schwieriger, mein süßer Gockel mit den
engen Hosen.

Sie zerriß ihren Zettel und gab dem Anmelde-Gockel einen
Fünfziger. Sie ging wieder. Hinab das große Treppenhaus
\aanah{}Wir-sind-wir\ae{}, hindurch durch die große Vorhalle \aa{}Eine-%
feste-Burg-ist-unser-Geld\ae{}, vorbei an dem Portier \aanah{}Alle-%
Menschen-gehören-ins-Zuchthaus\ae{}, auf die Straße.
\abstand{}
Tags zuvor, beim Bummel über den Bummel, hatte Bech"-%
terew von einer schrecklichen Großstadt-Krankheit erzählt, die
er Buden-Angst nannte. Nach Bechterew war die Buden-%
Angst eine chronische seelische Erkrankung, epidemisch auf"-%
tretend wie die Beulenpest und wie die schwarzen Pocken,
aber weit gefährlicher als diese Kinderkrankheiten der Mensch"-%
heit, weil der Bazillus noch nicht entdeckt war.

Im Gegensatz zur Beulenpest, die mit hohem Fieber und
geschwollenen Lymphdrüsen einsetzte, begann die Buden-%
Angst schleichend und unbemerkbar. Bei männlichen Patien"-%
ten konnte man am Anfang der Erkrankung leichte Zer"-%
streutheit und leichte Verstopfung feststellen, bei weiblichen
Patienten ziehende Schmerzen in den Hüften, verstärkte
Schönheitspflege, verstärktes Wippen beim Gang\dopp{} doch dieses
erste Stadium der Krankheit wurde meistens übersehn, weil
es als normal galt. Erst wenn die Buden-Angst in voller
Blüte stand, merkten die Patienten, daß sie vor ihrem eigenen
Selbst und vor ihrer leeren Bude entsetzliche Angst empfanden.
% Seite 57
Dann griffen sie ohne Wahl nach den zahlreichen Medika"-%
menten, die von der Großstadt zur Linderung der Buden-%
Angst geboten wurden. Gesteigertes Arbeitstempo war bei
allen Buden-Ängstlern als Linderung ihrer Seuche sehr
beliebt. Für Kassenpatienten wurde außerdem empfohlen
Theater-Kino-Radio-Sport. Bessere Patienten nahmen Reli"-%
gion-Literatur-Psychoanalyse-Auto-Verjüngung. Allen sozi"-%
alen Klassen gemeinsam war der große Verbrauch an Zeitung,
der billigsten Droge gegen Buden-Angst. Niemals jedoch hatte
man gehört, daß ein richtiger Buden-Ängstler durch alle diese
Drogen der großstädtischen Buden-Angst-Industrie geheilt
worden war.

Schon am zweiten Tag ihres neuen Lebens lernte Lea Herse
diese Krankheit kennen. Nach einem sinnlosen Trottoirmarsch
vom Zentrum der Stadt zum Westen der Stadt, kehrte sie
mit müden Füßen und müden Augen in ihr Hotelzimmer
zurück. Aber fünf Minuten später trieb sie die Angst vor der
leeren Bude wieder auf den Asphalt. Nun mußte sie doch in den
neuen Lunch-room wallen, zu dem preiswerten Sauerkraut
der rhythmischen Tänzerin, sie wußte nichts anderes zu tun.

Zum zweitenmal in ihrem Leben ein Bummel über den Bum"-%
mel. Beim Passieren eines Blumenladens fiel ihr ein, daß
jetzt in der Glonn ihr \aa{}Delphinium Berghimmel\ae{} seine erste
Blüte trieb. Zwei Meter hoch, der Stolz ihres Stauden"-%
gartens, der hellblaue Rittersporn \aa{}Delphinium Berg"-%
himmel\ae{}, daneben die lichte \aa{}\fremdsprachlich{Sancy de Parabère}\ae{}, die
Kletterrose des Hersehofs, jetzt blühten sie. Hier blühte die
Buden-Angst.

\aa{}Ich dachte schon, Sie versetzen mich\ae{}, rief Fräulein Lizzy
von Save durch den überfüllten Kohl-room. \aa{}Da wären Sie
% Seite 58
aber schwer reingefallen, mein Kind. Erstens gibt es hier eine
Gemüseplatte, wie sie in der ganzen Stadt nicht mehr zu
finden ist -- Sie sind doch als Landmensch gewiß ein richtiger
Gemüse-Mensch\frag{} Und zweitens habe ich Ihnen hier die klügste
Frau von Berlin mitgebracht, sie ist schon ganz verzweifelt,
weil Sie nicht kommen. Fräulein Lea Herse aus den Alpen --
Frau Johanna Duske aus -- woher stammst Du eigentlich,
Johanna\frag{}\ae{}

\aanah{}Aus dem Nichts\ae{}, sagte die klügste Frau von Berlin und
reichte Lea die Hand. Sie trug eine Hornbrille und war stark
gepudert. Aber hinter den großen Gläsern steckten ein paar
wirklich kluge Augen, ein ähnliches Eisblau wie Leas Augen,
und unter dem Puder steckte eine frische junge Haut. Im
Gegensatz zu der Rhythmischen schien sie ruhig und selbstsicher
zu sein.

\aa{}Sagen Sie offen, Sie wollten mich versetzen\frag{}\ae{}, maulte
Fräulein von Save, nachdem sie für Lea eine Gemüseplatte
bestellt hatte. \aa{}Mein Kind, Sie wollten mich versetzen, ob"-%
wohl Sie mir gestern in die Hand versprochen haben, zu
kommen\frag{}\ae{}

\aa{}Offengestanden ja\ae{}, sagte Lea, \aa{}ich wollte nicht kommen.\ae{}

\aa{}Und dann hat eine geheimnisvolle innere Stimme Sie doch
noch hierhergetrieben\frag{}\ae{}

\aanah{}Ach nein\ae{}, sagte Lea, \aa{}es war nur Buden-Angst, gar nichts
anderes.\ae{}

\aa{}Na so was Entzückendes\ae{}, rief Fräulein von Save begeistert.
\aanah{}Was sagst Du zu dieser frechen Beleidigung, Johanna\frag{}\ae{}

\aa{}Ganz in Ordnung\ae{}, sagte Frau Duske mit ihrem wunder"-%
hübschen tiefen Alt und griff nach der Mittagszeitung. Sie
begann zu lesen und schien Leas Ankunft schon wieder ver"-%
% Seite 59
gessen zu haben. Das war ein wenig kränkend, dennoch fand
Lea dieses freiheitliche Gebaren sehr sympathisch.

Eine Viertelstunde lang war über dem Zeitungsrand nur eine
glatte Stirn und eine glatte helle Haartafel zu sehn. Frau
Duske schien sich nicht im geringsten für ihre Umgebung zu
interessieren, weder für den dicken Herrn, der seinen Hut
über ihren Kopf hinweg vom Kleiderständer schwang, noch
für Fräulein Herse aus den Alpen, am wenigsten für Fräulein
von Saves soziale Ideen, die Lea mit der Gemüseplatte
vorgesetzt bekam.

\aa{}Mensch, ich muß gehn\ae{}, sagte die Düsseldorferin endlich und
brach mit dem amerikanischen Armbanduhr-Blick ihr Ge"-%
schwätz ab. \aanah{}Theoretischer Unterricht, Muskellehre, mag ich
gar nicht --\ae{}

Lea geriet in Verlegenheit, während Fräulein von Save
zahlte. Sie wartete noch auf ihre Mehlspeise und Frau Duske
saß hinter ihrer Zeitung, als gäbe es nichts andres auf der
Welt wie Eisenbahnkatastrophen, möblierte Zimmer, Völker"-%
bundssitzungen, Theaterklatsch, fast fabrikneue Markenwagen,
Schwergewichtler.

Sollte sie auf den Apfelstrudel nach Wiener Art verzichten
und mit Fräulein von Save zahlen und gehn\frag{} Hatte sie auf
die klügste Frau Berlins einen so üblen Eindruck gemacht,
daß die sich ihretwegen hinter der Zeitung vergrub\frag{} Sie rief
den Kellner zu sich, zu zahlen und zu gehn.

\aanah{}Aber wieso denn\frag{}\ae{} Frau Duske lächelte freundlich über die
Zeitungswand hinüber. \aa{}Bleiben wir nicht noch ein wenig\frag{}\ae{}

\aa{}Ja\frag{}\ae{}, frug Lea verdutzt, \aa{}meinen Sie\frag{}\ae{}

\aa{}Natürlich bleiben Sie\ae{}, sagte Fräulein von Save schnell
und verabschiedete sich. \aa{}Sie sind doch ein freier Mensch, Sie
% Seite 60
stecken doch nicht in der Tretmühle wie ich -- djüs djüs djüs --\ae{}
Sie rannte zur Muskellehre und ließ Lea mit der klügsten
Frau Berlins allein.

\aa{}Ich kann nämlich diesen sozialen Quatsch nicht mehr hören\ae{},
sagte Frau Duske und legte die Zeitung beiseite, während
ihre Freundin noch in der Drehtür steckte. \aa{}Lizzy ist bürgerlich
bis in die Knochen, sie soll die Hände von diesen Dingen lassen
-- aber sie ist reizend, nicht wahr\frag{}\ae{} Dabei blinzelte sie Lea
zu, als wollte sie sagen\dopp{} \aa{}Die blödeste Kuh der Welt, nicht
wahr\frag{}\ae{}

Lea lachte.

\aanah{}Wir verstehn uns\ae{}, sagte Frau Duske mit bezauberndem
Lächeln. \aa{}Sind Sie Kommunistin oder sind Sie Nihilistin\frag{}
Eins von beiden muß man ja schließlich sein\frag{}\ae{}

\aa{}Ich bin gar nix\ae{}, sagte Lea trocken.

\aanah{}Wunderbar\ausr{} Ich bin auch nur beruflich an diesen Dingen
interessiert, nicht substantiell. Ich bin Photographin und
entwerfe nebenbei Dekorationen für die kommunistischen
Theater, ganz interessante Arbeit, jedenfalls besser als bürger"-%
liche Kunst, aber im Grunde der gleiche Schwindel. Inter"-%
essieren Sie Sich\eingriff{eS60-1}{Sich ] sich} für photographische Kunst\frag{}\ae{}

\aa{}O ja\ae{}, sagte Lea, \aa{}was photographieren Sie denn\frag{}\ae{}

\aanah{}Was mir über den Weg läuft\ausr{} Katzen, Menschen, Blumen,
alles. Darf ich Sie auch photographieren\frag{}\ae{}

Lea empfand diese Frage unangenehm und sagte schnell\dopp{}
\aa{}Haben Sie schon mal ein \haa{}Delphinium Berghimmel\hae{}\eingriff{eS60-2}{\haa{}Delphinium Berghimmel\hae{} ] \aa{}Delphinium Berghimmel\ae{}}
photographiert\frag{}\ae{}

\aanah{}Was ist das\frag{}\ae{}

\aa{}Eine hellblaue Sorte Rittersporn.\ae{}

\aa{}Möglich, ich kann mich nicht erinnern.\ae{}

% Seite 61
\aa{}Haben Sie schon mal einen Herrn Pasternak photographiert\frag{}\ae{}

\aanah{}Was ist das\frag{}\ae{}

\aa{}Professor Pasternak, ich kenn ihn nicht, aber ich sah mal ein
glänzendes Photo von ihm.\ae{}

\aa{}Pasternak\frag{} Cellist\frag{}\ae{}

\aa{}Nein, berühmter Chemiker.\ae{}

\aa{}Chemiker\frag{} Nein. Ich glaube, ich hab in meinem ganzen
Leben überhaupt noch keinen Chemiker gesehn. Aber trinken
Sie doch eine Tasse Kaffee bei mir, wenn Sie sich\eingriff{eS61-1}{Sich ] sich} für schöne
Aufnahmen interessieren, ich zeige Ihnen ein paar gute Stu"-%
dien. Oder sind Sie heute nachmittag besetzt\frag{}\ae{}

\aa{}Nein, ich bin frei\ae{}, sagte Lea.

Sie fuhren in Frau Duskes Atelier für moderne Bildkunst
und tranken in einer Ecke mit vielen bunten Kissen bittern
Mokka und süßen Schnaps.

\aa{}Später mache ich ein paar Bilder von Ihnen\ae{}, sagte Frau
Duske, \aa{}aber zuvor will ich Ihnen ein paar interessante
Dinge aus meinem Archiv zeigen. Was wollen Sie sehn\frag{}
Kinder, Hunde, Männer, Weiber\frag{}\ae{} Sie nahm Leas Arm
und schlenkerte mit ihr in das kleine Archiv-Zimmer. \aa{}Männer
interessieren Sie wohl nicht\frag{}\ae{}

\aa{}Nicht im geringsten\ae{}, plapperte Lea ohne Nachdenken. Doch
das schien ein glänzender Spaß gewesen zu sein, denn Frau
Duske gluckste zur Antwort wie eine Lachtaube. \aanah{}Aber wirklich
nicht im geringsten\ae{}, wiederholte Lea, um den verborgenen
Witz dieses Ausspruchs für ihre liebenswürdige Wirtin noch
ein wenig auszutreten. \aa{}Ich will gar nichts von Männern
wissen, Katzen sind mir lieber.\ae{}

\aa{}Wir verstehn uns, mein süßes kleines Alpenveilchen\ae{}, sagte
Frau Duske und legte zuerst ein paar Katzenporträts vor, die
% Seite 62
Lea entzückend fand. Dann kamen ein paar farbige Blumen"-%
studien, Astern und Phlox und Dahlien, aber \aa{}Delphinium
Berghimmel\ae{} war nicht dabei. Ein paar Damen und Hunde
aus der Gesellschaft fand Lea langweilig, doch sie blieb bei
ihrem freundlichen \aa{}Entzückend\ae{}. Es kam eine Mappe mit
Dekorations-Entwürfen für ein modernes Theaterstück\semi{} zu
jedem Blatt gab es eine langatmige Erklärung über die beab"-%
sichtigte Wirkung\semi{} sie verspürte keinen Hauch von dieser Wir"-%
kung und dachte nur noch krampfhaft daran, nicht gähnen zu
müssen. Es kamen ein paar Aktstudien, Freilicht-Aufnahmen\semi{}
sie kannte dieses Zeug bereits aus den illustrierten Zeitungen
und Magazinen, die sich hie und da in die Glonn verirrt hatten\semi{}
sie kannte diese verzückten Stellungen ihrer berühmten Ge"-%
schlechtsgenossinnen und diese unappetitliche Manier, dem
begeisterten Beschauer den Steiß entgegenzustrecken\dopp{} ent"-%
zückend. Dann kamen doch noch ein paar Männerköpfe, be"-%
rühmte Künstler und berühmte Kaufleute\semi{} die Kaufleute
versuchten künstlerisch zu glotzen und die Künstler glotzten
möglichst kaufmännisch\semi{} eitle Tröpfe die und die\dopp{} entzückend.
Sie war froh, als bei ihrem hundertsten \aa{}Entzückend\ae{} eine
Assistentin im weißen Operationsmantel in das Archiv trat
und Besuch meldete.

\aa{}Na so was\ae{}, rief Frau Duske und schlug sich mit der Hand an
die Stirn, \aa{}wie konnte ich das vergessen\ausr{} Ists schon vier Uhr\frag{}
Für vier Uhr ist nämlich Major Ellen Wladden aus New
York, eine Heilige, eine führende Persönlichkeit in der Heils"-%
armee, setzen Sie Sich\eingriff{eS62-1}{Sich ] sich} während der Aufnahme ganz still in die
Ecke, ich werde Sie als meine Schülerin vorstellen, sonst
% Seite 63
wird mir dieser Engel scheu und verpatzt mir die Aufnahme,
kommen Sie --\ae{}

Lea wurde von trüben Gefühlen beschlichen, als sie wieder
in der Ecke mit den vielen bunten Kissen hockte und guckte,
wie der Engel von der Heilsarmee für ein literarisches Magazin
photographiert wurde. Zuerst war sie von der klügsten Frau
Berlins bezaubert gewesen. Aber die Atmosphäre im Archiv
war ihr aufs Herz gefallen wie Meltau\label{lS63-1} auf eine junge Staude.
Sie wußte selbst nicht, wieso es kam. Vielleicht war sie nur
matt von dem ewigen Bewundern-Müssen\frag{} War Frau Duske
nicht reizend\frag{} Hatte sie sich nicht wie eine Schwester an sie
gepreßt, als sie bei einem roten Phlox-Photo\label{lS63-2} in wirkliche
Begeisterung ausgebrochen war\frag{}

Major Ellen Wladden trug den langen Rock der Heilsarmee, am
Kittel das bescheidene Abzeichen ihres Ranges, dazu die große
Schute mit den zwei Bändern. Sie sprach wenig und ließ sich
geduldig von Frau Duske hin und her schieben, eine halbe
Stunde lang, bis Stellung und Licht paßten. Sie sah tatsächlich
wie eine Heilige aus, ein süßes rundes Gesichtchen, sanfte dunk"-%
le Augen, frische rote Bäckchen, ein wenig säuerlich der Mund
vom vielen Hallelujah und Gebettel um die Gnade des Herrn.

Aber je länger Lea hinsah, umso nervöser wurde sie. Die Hei"-%
lige aus New York wurde ihr von Minute zu Minute peinlicher.
Warum stand sie vor dem Apparat und vor den grellen
Lampen mit dem gleichen Sanftmut wie in den Kaschemmen
vor den Sündern der Welt, ihren frommen Song zu plärren
und ihre Cents zu sammeln\frag{} Warum streckte sie nicht, während
ihr Frau Duske die blöden Schutenbänder wirkungsvoll
drapierte, ihre Zunge aus dem Engelsmäulchen raus\frag{} Wa"-%
rum sprach sie kein Wort\frag{}

% Seite 64
Immer kribbeliger wurde Lea, als die Prozedur nicht zu Ende
kam. Die nervösen Ameisen liefen ihr über die Arme und über
die Oberschenkel. Die Wut auf die Heilige wuchs von Minute
zu Minute in ganz sinnloser Weise. Ein richtiger Krach mit
dem sanften Geschöpf wäre eine wahre Erlösung gewesen.
Nur aus Rücksicht auf die liebe Frau Duske hielt sie an sich.
Und es war ja auch wahrhaftig kein Wort gegen Major Ellen
Wladden vorzubringen\frag{} Ein leibhaftiger Engel, was sollte
dieser schwere heidnische Zorn\frag{}

Wenigstens beantwortete sie den schwesterlichen Abschieds"-%
gruß, als dann der Engel endlich ging, mit einem starren und
gehässigen Blick und übersah die zarte Patschhand, die sich ihr
entgegenstreckte. Sollte der Engel doch fühlen, daß hier ein
Feind gesessen war\ausr{} Vermutlich aber war der Engel viel zu
dumm, um überhaupt etwas zu fühlen, was nicht nach Elend
oder Hallelujah roch\frag{} Hauptsache, es war vorüber und die
neurasthenischen Kribbel-Ameisen verzogen sich.

Frau Duske kam strahlend aus dem Entree zurück. \aa{}Ist sie
nicht himmlisch\frag{} Einer unserer up-to-datesten jungen Lyriker
hat eine brilliante Schauerballade auf sie geschrieben, guter
Freund von mir, wissen Sie, so eine Ballade in primitivem
Stil, dazu brauchen wir ihr Bild -- ist sie nicht wunder"-%
bar\frag{}\ae{}

\aa{}Die\frag{}\ae{}, sagte Lea. \aa{}In meinem Leben noch kein so blödes
Weib gesehn.\ae{}

\aanah{}Was sagen Sie da\frag{}\ae{}

\aa{}Na, fühlen Sie denn nicht, daß das alles Schwindel ist\frag{}\ae{}

\aanah{}Aber hören Sie, Kind\ausr{} Sie täuschen Sich\eingriff{eS64-1}{Sich ] sich} wirklich\ausr{} Ellen
Wladden opfert sich für ihre Idee.\ae{}

\aanah{}Was geht das mich an\ausr{}\ae{}

% Seite 65
\aa{}Man kann sich ja zu jeder Idee stellen, wie man will, aber man
muß doch den heiligen Willen achten\ausr{}\ae{}

\aa{}Das wäre ja noch schöner\ae{},\eingriff{eS65-1}{schöner\ae{}, ] schöner,\ae{}} rief Lea, \aa{}ich achte gar nichts.\ae{}

\aa{}Sie wissen nichts von dieser Frau, mein Kindchen. Die geht
in die finstersten Spelunken, um den Menschen Hilfe zu
bringen, aber wirkliche Hilfe. Die reist in der ganzen Welt
herum und organisiert ihre einzelnen Stationen. Die ist über
jede menschliche Eitelkeit erhaben. Wirklich ein Engel, glauben
Sie mir.\ae{}

\aa{}Glaube ich gern, daß sie ein Engel ist, glaube ich, glaube ich\ausr{}\ae{}

\aanah{}Aber gegen einen Engel läßt sich doch nichts Böses sagen\frag{}\ae{},
rief Frau Duske lachend.

\aa{}Nein\frag{} Läßt sich nichts Böses sagen\frag{}\ae{}, schrie Lea in heller
Wut. \aa{}Dieser Engel ist ein Schwein\ausr{} Jawohl\ausr{} Ein Engel und
ein Schwein in einer Person, jawohl\ausr{} Daß sie in die Hölle
kommt, wenn sie gestorben ist, will ich gar nicht behaupten --
wir andern kommen natürlich in die Hölle und diese Dame
schwebt zum Himmel auf, das glaube ich gern -- als Schwein
mit Flügeln nämlich, als Schwein mit Flügeln wird sie in den
Himmel aufsteigen.\ae{}

Frau Duske brach in ein Gelächter aus und war hingerissen
von diesem unbegründeten und ungerechten Ausbruch.

\aa{}So ist die Sache\ausr{}\ae{}, schrie Lea. \aa{}Jawohl\ausr{} Ganz einsam wird
dieser Engel als Flügelschwein auf seiner nassen Wolke hocken
und Halleluja plärren\ausr{} Ich möchte nicht mir ihr tauschen, tat"-%
sächlich nicht\ausr{} Lieber als das, was ich bin, in der tiefsten Hölle
rösten\ausr{} Immer noch besser als solch Flügelschwein im
Himmel\ausr{}\ae{}

Hingerissen war Frau Duske. Sie umarmte Lea. Die war
froh, daß ihr Groll entleert war und daß die klügste Frau
% Seite 66
Berlins sie verstand. Es schien nicht schwer zu sein, gute
Freunde in der Großstadt zu finden und verstanden zu
werden. Zwar verstand sie selbst nicht recht, warum diese
schlimme Wut auf die sanfte Heilige über sie gekommen war,
aber ihre neue Freundin schien es zu verstehn. Immer wieder
wurde sie von ihr umschlungen. Mit geheimnisvoller Be"-%
geisterung wurde jedes böse Wort aufgenommen, das sie der
Amerikanerin nachrief. Als sie merkte, welches Vergnügen
ihr Zorn auslöste, geriet sie ins Kälbern und kälberte immer
weiter hinter dem Engel her. \aa{}Überhaupt hasse ich alle
Amerikanerinnen\ausr{} Ob sie mit dem lieben Gott flirten oder
mit einem Filmstar, ist ganz egal\ausr{} Das ist zwar die erste
Amerikanerin, die ich in meinem Leben sah, aber das ge"-%
nügt für alle.\ae{} Und das schien nun wieder ein viel größerer
Spaß zu sein, als sie selber ahnte. Frau Duske wälzte sich vor
Lachen, riß sie keuchend an sich, gab ihr einen Kuß auf den
Mund.

Zuerst war es ein fader Lippenstift-Geschmack. Sie ließ sich
gutmütig noch ein paarmal küssen. Plötzlich fühlte sie, daß ihr
Mund nicht mehr freigegeben wurde. Eine feuchte fremde
Zunge versuchte zwischen ihren Lippen vorzudringen. Wie eine
Qualle hing es plötzlich an ihr. Ein blutloser Leichnam trotz
diesem wüsten Drängen, eine lustlose Gehirnblase trotz diesen
wirren Trieben, so hing plötzlich die klügste Frau Berlins an
ihr. In vollem Entsetzen stieß sie die Qualle zurück und schlug
ihr mit ganzer Kraft zweimal mitten ins Gesicht.

Die Brille flog zu Boden und zerklirrte. Die klügste Frau von
Berlin hielt sich stöhnend die Hände vors Gesicht und schrie\dopp{}
\aa{}Hinaus, Du Hure\ausr{}\ae{} Lea wollte noch irgend etwas rufen, Du
Qualle, Du Bestie, Du Buden-Angst, Du Irgendetwas, aber
% Seite 67
sie preßte die Lippen aufeinander und lief stumm aus dem
Atelier. Das Dienstmädchen, das ihr die Gangtür öffnete,
grinste ihr verständnisvoll nach.

Vor dem Haus fand sie ein leeres Taxi. Ohne Besinnen gab
sie dem Chauffeur die Adresse ihres Vaters. Erst als sie schon
zehn Minuten gefahren war, fiel ihr ein, daß Samstag war,
Samstag und fünf Uhr nachmittags, sämtliche Bureaus der
Stadt waren längst geschlossen, ganz gewiß, auch das Ge"-%
schäftshaus ihres Vaters war geschlossen. Aber sie ließ den
Chauffeur die ganze Strecke zu Ende fahren, vom Westen der
Stadt zum Zentrum der Stadt, ohne sich zu rühren.
\abstand{}
Die Vorhalle des Hauses Pasternak war leer. Auch der Por"-%
tier \aanah{}Alle-Menschen-gehören-ins-Zuchthaus\ae{} war nirgends
zu sehn. Natürlich war längst Schluß.

Sinnlos, hier noch irgend etwas zu hoffen. Aber durch diesen
Raum war vor kurzer Zeit ihr leibhaftiger Vater geschriten,
velleicht schwebte noch ein Stückchen Unsichtbares von ihm
in der Luft, um ihre anwachsende Melchancholie zu umspielen
und zu lindern\frag{}

Sie lief vor der Marmortreppe ein bißchen auf und ab, als
erwarte sie noch einen säumigen Liebhaber aus einem der
Bureaus. Während zwei uniformierte Bengelchen die Treppe
herunterschlaksten, warf sie ungeduldige Blicke auf die Uhr
überm Portal, um zu zeigen, daß sie bestellt war. Die
Bengelchen liefen an ihr vorbei auf die Straße, ohne sie zu
beachten. Die Halle war wieder still und tot.

% Seite 68
Nichts. Sie mußte wieder gehn. Sie mußte wieder ins Hotel
gehn, zurück in die leere Bude. Sie mußte ihm einen neuen
Brief schrieben.

Als sie vom Portal aus einen letzten Blick in die Halle warf,
sah sie noch einen verspäteten Angestellten des Werkes Paster"-%
nak die Treppe herunterkommen. Ein subalterner älterer Herr,
ein abgekämpfter Beamter mit Spitzbauch und Aktenmappe,
müde und überarbeitet stapfte er die Marmortreppe herab
und auf sie zu. Da er zwischen den Türen des Portals einen
verwunderten Blick auf sie warf, was sie wohl hier zu suchen
hätte, wollte sie ihn wissen lassen, daß sie keine Einbrecherin
war, sondern eine Freundin seines großen Chefs, und sprach
ihn an. \aanah{}Verzeihung -- Professor Pasternak noch im Haus\frag{}\ae{}
Der Spitzbauch blieb vor ihr stehen. \ae{}Professor Pasternak\frag{}
Ich glaube, ja, er ist noch im Haus.\ae{}

Sie fühlte, daß sie knallrot wurde. \aa{}So\frag{}\ae{}, sagte sie kühl.
\aa{}Besten Dank.\ae{}

Er wollte sich offenbar in ein Gespräch mit ihr einlassen. Er
blieb stehn, zog eine hölzerne Dose aus der Tasche, rauchte
sich mit Pedanterie eine Zigarette an und musterte sie dabei
mit gutmütigen, dunklen Augen. \aa{}Sie wünschen Professor
Pasternak zu sprechen\frag{}\ae{}

\aa{}Jawohl -- er ist ganz bestimmt noch da\frag{}\ae{}

\aa{}Ich glaube ja -- fragen Sie doch bitte mal im ersten Stock bei der
Anmeldung nach -- wenn ich mich nicht irre, ist er noch oben --\ae{}

\aa{}Danke --\ae{}

Sie schritt in die Halle zurück und flog die Treppe empor.
An der Ecke vor dem ersten Stockwerk stoppte sie und wartete
fünf Minuten. Sie wollte nicht noch mal vor einem leeren
Zettel mit \haa{}Angelegenheit\hae{} stehn, für heute war ihr Mut
% Seite 69
dahin\dopp{} aber sie wollte sich auch nicht vor dem Herrn im Portal
lächerlich machen. Als er nach ihrer Schätzung verschwunden
sein mußte, schlenkerte sie wieder die Treppe hinab.

Er stand noch zwischen den Flügeltüren des Portals. Er hatte
auf sie gewartet. \aa{}Na, haben Sie Professor Pasternak ge"-%
troffen\frag{}\ae{}

\aa{}Nein, erist schon weg.\ae{}

\aa{}Er ist schon weg\frag{} So, so, er ist schon weg\frag{} Wer hat Ihnen
das gesagt\frag{}\ae{}

\aa{}Der Boy in der Anmeldung.\ae{}

\aa{}Der Boy in der Anmeldung\frag{} So, so\frag{} Ich glaube, der Boy
hat Sie angelogen. Soll ich mal selber nachsehn\frag{}\ae{}

\aa{}Sehr liebenswürdig, besten Dank, es ist nicht nötig, nicht so
wichtig.\ae{}

\aa{}Darf ich fragen, um was es sich handelt\frag{}\ae{}

Sie ärgerte sich über die zudringliche Art, mit der er sie an"-%
quatschte und ihr in die Augen stierte. \aa{}Nichts Geschäftliches\ae{},\eingriff{eS69-1}{Geschäftliches\ae{}, ] Geschäftliches,\ae{}}
sagte sie hochmütig, \aa{}besten Dank, guten Tag.\ae{} Sie lief
schnell auf das Trottoir hinaus, das von den Menschen des
Geschäftsschlusses überfüllt war.

\aa{}Eine Sekunde bitte\ae{}, sagte der Spitzbauch und schritt hastig
an ihre Seite.

\aa{}Ja\frag{}\ae{} Sie blieb mitten im Gedränge stehn, um zu zeigen,
daß sie keine Begleitung wünschte.

\aa{}Sagen Sie mir doch, um was es sich handelt, ich bin näm"-%
lich Professor Pasternak.\ae{}

\aanah{}Was\frag{}\ae{}

Fast hätte sie gesagt\dopp{} \aa{}Machen Sie keinen Blödsinn\ae{}, da
erkannte sie plötzlich, daß wirklich ihr Vater vor ihr stand.
Auf jenem Bild in der illustrierten Zeitung hatte sie ihn im
% Seite 70
Profil gesehn, ohne Hut, es war ein schmeichelhaftes Photo
gewesen und ihre Phantasie hatte die fehlenden Lichter ein"-%
gesetzt. Jetzt erkannte sie ihn tatsächlich. \aanah{}Ach, Sie sind es
selbst\frag{}\ae{}, sagte sie kühl. Die große Sekunde war vorüber, sie
spürte keinerlei Erschütterung. Statt der unsterblichen Um-%
den-Hals-Fallerei nur eine kleine Schulmädchen-Neugier, wie
der Streich zu Ende gehn würde.

\aa{}Ja, ich bin es\ae{}, hörte sie ihn. \aa{}Ich bin sehr gespannt, was
Sie mir zu sagen haben, gnädiges Fräulein.\ae{}

Das sollte freundlich klingen und kroch ihr kalt und fremd
übers Herz. \aa{}Nicht so einfach\ae{}, sagte sie, um Zeit zu gewinnen.
\aa{}Schwierige Sache.\ae{}

\aa{}Schwierige Sache\frag{}\ae{} Er lächelte ihr ein wenig zweideutig zu.
\aanah{}Vielleicht begleiten Sie mich ein Stück\frag{} Ich habe meinen Wa"-%
gen weggeschickt, weil ich noch ein bißchen Bewegung haben
muß. Nach welcher Richtung führt Ihr Weg\frag{}\ae{}

\aa{}Ganz egal.\ae{}

\aa{}Dann erstmal raus aus diesem Irrsinn -- halt, so geht das
nicht --\ae{} Er riß sie am Arm auf den Fußsteig zurück und hielt
sie fest, bis der Fahrdamm überschritten werden durfte.
Erst auf dem nächsten Fußsteig ließ er ihren Arm wieder los.
\aa{}Noch keine Woche in Berlin, schätze ich\frag{}\ae{}

\aa{}Stimmt\ae{}, sagte sie und lachte.

\aa{}Und direkt aus den Alpen importiert\frag{}\ae{}

\aanah{}Woher wissen Sie --\ae{}

\aa{}Das ist kein Kunststück bei Ihrem Tonfall, gehn wir hier
rechts --\ae{}

Sie lief eine Weile schweigend neben ihm her, ohne zu ahnen
wohin der Weg führte. Endlich kamen sie in eine menschenleere
Straße.

% Seite 71
\aa{}Na, legen Sie los, gnädiges Fräulein, mit unserer schwieri"-%
gen Sache --\ae{}

Sie hatte sich schon in der Glonn einen kleinen Trick zurecht"-%
gelegt, um sich aus der Klemme zu ziehn, wenn die große
Um-den-Hals-Fallerei nicht klappte. \aa{}Sie sind doch der erste
Besteiger des Botzong-Kamins\ae{}, frug sie in trockenem Ton.

\aanah{}Was bin ich\frag{}\ae{}

\aa{}Kennen Sie nicht den Botzong\frag{}\ae{}

\aa{}Nein.\ae{}

\aanah{}Was\frag{}\ae{}, rief sie entsetzt.

\aanah{}Wer soll das sein\frag{}\ae{}

\aa{}Botzong, Botzong-Kamin, Kaisergebirg, Predigtstuhl, Bot"-%
zong-Kamin, erinnern Sie Sich\eingriff{eS71-1}{Sie Sich ] Sie sich} nicht\frag{}\ae{}

Er blieb stehn und starrte sie an. Wie ein Auferstandener,
der auf sein eigenes Grab starrt, starrte er sie an. \aa{}Natürlich,
selbstverständlich erinnere ich mich, Predigtstuhl, Botzong-%
Kamin -- ja um Gottes Willen\eingriff{eS71-2}{Willen ] willen}, wie kommen Sie denn darauf,
gnädige Frau, daran hab ich mindestens zwanzig Jahre
nicht mehr gedacht -- Botzong, selbstverständlich, Botzong-%
Kamin, was soll denn das --\ae{}

\aa{}Das ist sehr einfach.\ae{}\eingriff{eS71-3}{einfach.\ae{} ] einfach\ae{}.} Sie versuchte sich wieder in Marsch zu
setzen. Aber er ging keinen Schritt vorwärts. Sie mußte vor
ihm stehn bleiben und ihm ins Gesicht sehn. \aa{}Sehr einfach,
ich bin Alpinistin, ziemlich bekannt als Alpinistin, ich schreibe
ein Buch über das Kaisergebirg, eine Geschichte sämtlicher
Erstbesteigungen im Kaisergebirg, so wollte ich doch den ersten
Bezwinger des Predigtstuhls sehn, da ich zufällig in Berlin
bin, das ist die ganze Geschichte.\ae{}

\aa{}Das ist wirklich wunderbar\ae{}, rief Professor Pasternak
begeistert und man konnte deutlich hören, daß ihr Trick keine
% Seite 72
Sekunde lang angezweifelt wurde. \aa{}Das ist wunderbar,
Sie ahnen nicht, wie wunderbar das ist\ausr{}\ae{} Er setzte sich endlich
wieder in Bewegung. \aa{}Können Sie mich nicht noch ein Stück
begleiten, gnädiges Fräulein\frag{} Bitte sehr\ausr{} Das ist nämlich
wirklich eine ganz wunderbare Erinnerung für mich.\ae{}

\aa{}Das freut mich.\ae{}

\aa{}Darf ich Sie nicht um Ihren Namen bitten\frag{}\ae{}

\aa{}Lea Herse.\ae{}

\aa{}Frau oder Fräulein\frag{}\ae{}

\aa{}Fräulein.\ae{}

\aa{}Fräulein Lea Herse\ausr{} Und schon so eine große Alpinistin\frag{}
Verzeihn Sie, wenn ich Ihren Namen nicht kenne, ich
habe mich seit Ewigkeit nicht mehr um alpine Literatur ge"-%
kümmert.\ae{}

\aanah{}Ach es ist nicht so schlimm mit meinem alpinen Ruhm.\ae{}

\aa{}Doch, ganz gewiß -- übrigens täuschen Sie Sich\eingriff{eS72-1}{Sich ] sich}, fällt mir
soeben ein\dopp{} nicht ich habe die Erstbesteigung des Botzong-%
Kamins gemacht, sondern Botzong selbst, das ist der Mann,
nach dem der Kamin benannt ist, Botzong hat ihn zuerst
durchstiegen --\ae{}

\aa{}Natürlich\ae{}, sagte sie schnell, \aa{}das weiß ich.\ae{}

\aa{}Ich habe nur ein paar Jahre später die erste völlige Durch"-%
steigung bis zum Hauptgipfel gemacht --\ae{}

\aa{}Ich weiß, ich weiß, genau so wird es in meinem Buch
stehn.\ae{}

\aa{}Und zwar allein, ich war Alleingänger.\ae{} Er zog seinen Bauch
ein und marschierte straff neben ihr her. \aa{}Bitte schreiben Sie
das doch auch in Ihr Buch, daß ich allein war, als ich diese
Sache schmiß.\ae{}

\aa{}Selbstverständlich.\ae{}

% Seite 73
Er war hingerissen von dieser plötzlichen Erinnerung an die
Kletterfahrten seines früheren Lebens. Ach das Leben in"-%
mitten jener heidnischen Felswände\ausr{} Beim nächsten Straßen"-%
übergang packte er wieder ihren Arm und führte sie mit einem
festen, freundschaftlichen Griff. \aanah{}Wollen Sie einem alten
alpinen Kameraden, der mit Haut und Haar von der Groß"-%
stadt aufgefressen worden ist, eine große Freude bereiten,
gändiges Fräulein\frag{}\ae{}

Sie schwieg. Er hielt noch immer ihren Oberarm umspannt,
obwohl der Fahrdamm längst passiert war.

\aanah{}Wollen Sie mir nicht diesen kleinen Samstagabend schen"-%
ken\frag{} Oder sind Sie besetzt\frag{}\ae{}

\aa{}Besetzt bin ich nicht --\ae{}

\aanah{}Aber\frag{}\ae{}

Sie verschluckte ihre Antwort.

\aa{}Fahren Sie doch zum Abendessen mit mir in mein Landhaus,
ja\frag{} Auch meine Frau wird sich sehr freun, ja\frag{}\ae{}

\aa{}Ja\frag{}\ae{}

\aa{}Es ist zwar ziemlich weit außerhalb der Stadt, aber ich bringe
Sie selbst in meinem Wagen zurück. Ja\frag{}\ae{}

\aa{}Ja\frag{}\ae{}

\aa{}Bitte sehr\ausr{}\ae{}

\aa{}Gut.\ae{}

\aa{}Besten Dank.\ae{}

Endlich gab er ihren Arm frei.

\aa{}Im nächsten Zigarrenladen telephoniere ich meinem Chauf"-%
feur.\ae{}

Sie mußten noch durch eine endlose, öde Straße wandern,
bevor er seinen Wagen bestellen konnte. Dann standen sie
vor dem Zigarrenladen und warteten, schweigend. Der Feier"-%
% Seite 74
abend lastete schwer auf den Häuserfronten und Menschen"-%
gesichtern ringsum. Die Arbeitswoche war zu Ende. Die
Großstadt rüstete sich, den toten Sabbat und den toten Sonn"-%
tag zu begehen und ihre große Buden-Angst zu übertosen.
\abstand{}
\aa{}Fahren Sie langsam, Jünemann\ae{},\eingriff{eS74-1}{Jünemann\ae{}, ] Jünemann,\ae{}} sagte Professor Paster"-%
nak zu dem Chauffeur des offenen Sechssitzers, \aa{}nur immer
langsam voran, Jünemann, ich bin ein alter Mann und im
neunten Monat.\ae{} Das war ein Witz, den er von einem be"-%
freundeten Schauspieler hatte, aber der Witz tat, was solche
Großstadtwitze schon am zweiten Tag nach ihrer Geburt zu
tun pflegen\dopp{} er fiel tot zu Boden. Der Schauspieler hatte ihn
schon lange im Gebrauch gehabt, ehe er in an Professor
Pasternak weitergegeen hatte, und der hatte ihn auch schon
ziemlich oft benutzt, das fühlte Lea und das verstimmte sie.

Mit leeren Augen schaute sie auf die vorübergleitende Mond"-%
kraterlandschaft aus Beton-Asphalt-Blech-Gummi-Menschen"-%
fleisch. Einige Plätze und Straßen, durch die sie vorhin gefah"-%
ren war, auf der Fahrt von der klügsten Frau Berlins zu
ihrem Vater, erkannte sie wieder. Es schien wieder nach
Westen zu gehen.

Professor Pasternak sah sie von der Seite an und schien ihre
Trauer zu spüren, denn plötzlich streckte er ihr die Hand hin
und sagte\dopp{} \aanah{}Wir haben uns noch nicht mal die Hand gegeben,
Fräulein Herse.\ae{}

Sie nahm die Hand und drückte sie scheu. Aber er hielt sie fest,
sie mußte eine Zeitlang Hand in Hand mit ihm fahren.
Erst beim Stop an der nächsten Kreuzung ließ er los.

% Seite 75
\aa{}Ja, ja, wenn ich an den Botzong-Kamin denke, kommt mir
dies alles hier wie Irrsinn vor.\ae{} Er deutete auf die überfüllten
Kanäle der Mondkraterlandschaft. \aa{}Sie sind zum erstenmal
in Berlin\frag{}\ae{}

\aa{}Zum ersten Mal in meinem Leben in der Stadt\ausr{}\ae{}

\aanah{}Was\frag{}\ae{}

\aanah{}Wirklich.\ae{}

\aa{}Na so was\ausr{} Immer auf dem Land\frag{} Immer in den Alpen\frag{}
Dann drückt Ihnen dieser Betrieb natürlich aufs Herz. Aber
Sie\eingriff{eS75-1}{Sie ] sie} dürfen die Stadt nicht nach ihrem äußeren Antlitz be"-%
urteilen.\ae{}

\aa{}Nein\frag{}\ae{}

\aa{}Nein, gewiß nicht\ausr{} Obwohl auch die Fassade schön ist. Ist
das zum Beispiel nicht toll\frag{}\ae{} Der Stop war zu Ende, von
beiden Seiten schossen die Wagen wieder los, weiter, weiter,
immer weiter, wie freigelassene Tiger aus dem Käfig stürzte
es ringsum aus dem Stop. \aa{}Finden Sie das nicht wunder"-%
bar\frag{}\ae{}

\aa{}O ja, ganz schön.\ae{} Es klang nicht sehr überzeugt.

\aa{}Man muß das Leben in dieser Stadt lieben, sonst geht man
zu Grund. Dies alles muß genau so sein, wie es ist.\ae{}

\aa{}Nein, es muß nicht so sein\ausr{} Man muß sich entscheiden\ausr{}\ae{}

\aanah{}Was heißt das\frag{} Entscheiden wozu\frag{} Raus aus der Stadt
oder rin in die Stadt\frag{} Raus aus dem Betrieb oder rin in
den Betrieb\frag{} Aber wer hat denn hier noch die Freiheit zum
freien Entscheid\frag{}\ae{}

\aanah{}Alle Menschen.\ae{}

\aa{}Irrtum.\ae{}

\aa{}Ich jedenfalls.\ae{}

\aa{}Möglich.\ae{}

% Seite 76
Sie schwieg hochmütig.

\aa{}Und wozu haben Sie Sich\eingriff{eS76-1}{Sich ] sich} entschieden\frag{} Natürlich bald wieder
in Ihre Alpen zurück\frag{}\ae{}

\aa{}Das weiß ich nicht. Ich habe meine Entscheidung noch nicht
getroffen. Aber ich werde es tun. Wenn ich in der Stadt bleibe,
werde ich jedenfalls komplett hierbleiben.\ae{}

\aanah{}Was heißt das\frag{} Komplett\frag{}\ae{}

\aa{}Ganz und gar\ausr{} Nicht mit dem einen Auge in die Stadt
schielen und mit dem andern Auge in die Natur schielen\ausr{}
Nicht halb und halb wie Ihr hier\ausr{}\ae{}

\aanah{}Wer Ihr\frag{} Er schien sich über ihren zornigen Ton zu mo"-%
kieren.

\aa{}Keinesfalls werde ich es so machen wie Sie, Herr Professor
Pasternak.\ae{}

\aanah{}Aber was mache ich denn\frag{}\ae{}, frug er bestürzt. \aanah{}Was wissen
Sie denn von mir\frag{}\ae{}

\aa{}Sie haben selbst gesagt, daß Sie Sich\eingriff{eS76-2}{Sich ] sich} mit Haut und Haar
auf"|fressen lassen von dieser Stadt. Das genügt.\ae{}

Er lachte. \aa{}Das dürfen Sie doch nicht wörtlich nehmen,
gändiges Fräulein. Sagen wir nur \haa{}mit Haut\hae{}. Die Haare
sind noch da, wie Sie sehn.\ae{} Er nahm den Hut ab und strich
durch sein dichtes kanstanienrotes Haar. Ohne Hut sah er frischer
und männlicher aus. Die Haare waren nur wenig angegraut.

\aa{}Nein mit Haut und Haar\ae{},\eingriff{eS76-3}{Haar\ae{}, ] Haar,\ae{}} sagte sie mit bösem Lachen, \aa{}die
Haare sind auch weg.\ae{}

\aa{}Na hören Sie\ausr{}\ae{}

\aanah{}Weg\ausr{}\ae{} Sie war froh, das Gespräch in Spaß überleiten zu
können. \aa{}Das ist nur Schein, die Haare sind weg, aufge"-%
fressen von der Stadt, weg, ganz weg\ausr{}\ae{}

% Seite 77
\aa{}Bitte, überzeugen Sie Sich\eingriff{eS77-1}{Sich ] sich}\ausr{}\ae{} Er hielt ihr den Kopf hin.

Sie tippte vorsichtig mit der Fingerkuppe an sein Haar.
\aanah{}Weg, ganz weg, die Haare sind weg, ich spüre nichts mehr
davon, es hat schon gestimmt, aufgefressen mit Haut und
Haar.\ae{}

\aa{}Sehr traurig\ae{}, sagte er und drückte sich den Hut schief ins
Gesicht.

Sie fuhren durch die letzten Straßenzeilen und gelangten
ins Freie. Lea streckte sich begeistert de offenen Landschaft
entgegen. Die Luftmassage und das schöne kapitalistische
Polstergefühl am Hinterteil hoben ihre Stimmung. Es war
ja albern, nach zwei Tagen Stadt und nach zwei Stunden
Papa irgendwelche großen Entscheidungen treffen zu wollen.
Bei dem großen Stop im Zentrum der Stadt hatte sie be"-%
schlossen, mit einem der nächsten Züge in die Glonn zurück"-%
zufliehen und ihre Expedition als gescheitert zu betrachten.
Aber das war irgendeine Überreizung gewesen, ganz gewiß.
So einfach lagen die Dinge nicht, die Entscheidung zwischen
Stadt und Land war schwierig, ganz gewiß.

Ganz gewiß fühlten alle diese Menschen ringsum, arm und
reich, die gleiche Sehnsucht, aus der Stadt zu fliehn, zurück
zur Natur, wie sie. Aber die Armen verhungerten auf dem
flachen Land, die waren mit den Ketten des Hungers an die
Stadt gekettet\semi{} und die Reichen wollten ihre städterne Macht
nicht fahren lassen, die waren an den Ketten des Geizes fest"-%
gelegt. So sanken sie immer tiefer in den Sumpf, arm und
reich, wie ein Denker versumpft, der einen halben Gedanken
gedacht hat, sich darin verbohrt, ihn nicht mehr lassen kann.

In eine halbe Idee hatten sie sich verbohrt, die Städter,
und waren zu geizig, diese halbe Idee über Bord zu werfen
% Seite 78
und von vorn zu beginnen\ausr{} In eine halbe Idee verbohrt, die
ganze menschliche Rasse\ausr{} Die eine Hälfte ihres Lebens war
verkörpert in den Städten, in der Technik, in der Wissenschaft,
entsprossen aus der christlichen Gemeinschaft, vernietet und
gehalten durch deren kümmerlichen Rest -- die andre Hälfte
lag weit weg davon, im heidnischen Wurzelsaft der Pflanze
Mensch, die war auf diesem Wege niemals zu verkörpern\dopp{}
was flohn sie nicht heraus aus diesen christlich-technischen
Städten\frag{} Was flohn sie nicht in heidnischen Massen über den
geräumigen Ball, Ausschau nach einem neuen Leben zu
halten\frag{} Und hast du dich in einen wirren Wald verirrt, mein
Mensch, so lauf doch schnell den falschen Weg zurück und hoffe
nicht, daß dich ein Zufall oder irgendeine höhere Macht zum
Ziel geleitet, sonst tappst du in den Sumpf, Sumpf, Sumpf,
Sumpf, Sumpf.

Zu schnell, zu schnell\ausr{} Entscheidungen rasten nicht so schnell
wie dieser hübsche blecherne Kasten\ausr{} Weil ihr Papa sie in der
ersten Stunde schwer enttäuschte, vielleicht durch ihre eigene
Blödigkeit, deswegen mußte nicht die ganze Menschheit
blöde und sie allein bei wachen Sinnen sein\ausr{}

\aa{}Nein, die Haare sind noch da\ae{}, sagte sie plötzlich zu ihrem
Vater, der vor sich hingebrütet hatte. \aanah{}Alles noch da, Herr
Kamerad vom Botzong-Kamin, zeigen Sie mal --\ae{} Sie nahm
ihm den Hut ab und beguckte mit einem eisblauen Blick sein
Haupthaar, sein Gesicht. Sie tippte an seine Wange, sie
tippte an seinen Hinterkopf. \aanah{}Tadellos, Haut und Haar
noch da, alles noch da, nichts aufgefressen.\ae{} Sie drückte ihm
den Hut wieder fest in die Stirn.

Er strahlte und ließ den Hut sitzen, wie er saß, äußerst schief.
Plötzlich sah er wie ein angetrunkener alter Stromer aus.
Und da begann er auch schon zu dudeln, ein kleines Stromer"-%
lied dudelte er leis vor sich hin, während der Wagen kurz
nach dem Wannsee in eine Seitenstraße einbog. Zum ersten"-%
mal seit Jahren hörte der alte Jünemann seinen Chef vor
sich hin dudeln.
\abstand{}
Das Gedudel des Herrn Professor Pasternak entstammte nicht
etwa irgendeinem unbewußten Vatergefühl. Es kam nicht aus
einem dunklen Instinkt, daß sein Fleisch und Blut neben ihm
saß. Es kam auch nicht von der freudigen Erinnerung an den
Botzong-Kamin. Was er in Leas Nähe spürte, war jene leben"-%
dige Urhülle der Natur, darin der Mensch noch atmen kann
ohne das Asthma der Gedanken, atmen kann ohne das Asthma
von Gut und Bös, atmen kann als frommer Heide und dudeln.
Es war ein Rest jener lebenspendenden Urhülle der Natur,
die einst den ganzen Ball umspannte und ihn bewahrte vor
der Nacktheit im All.

\fremdsprachlich{Pauvre reste\ausr{}} Schon im Garten Eden begann diese körperliche
Urhülle der Natur vom Menschen zurückzuweichen. Adam
spürte es mit Schrecken und griff zum Apfel der Erkenntnis,
den Eva ihm bot, das findige Weib -- aber was nutzte es,
daß er mit diesem jämmerlichen Ersatz-Organ die Stimme
eines Gottes hörte\frag{} Er schämte sich seiner neuen Nacktheit im
Raum, immer weiter wich die Urhülle der Natür vom Men"-%
schen zurück. Moses fühlte sich noch von ihr umhüllt auf Sinau,
doch er mußte schon schlimme Märchen lügen, der findige
Dichter, um sich mit seinen Goldenen-Kalb-Tänzern zu ver"-%
ständigen. Bis endlich Christus kam, der findige Nihilist, und
% Seite 80
ihnen als Ersatz für ihr erkranktes Diesseits das tote Jenseits
anbot. Und sie griffen zu sie klammerten sich an diesen letzten
Diesseits-Ersatz, sie klammerten sich in ihrer Todesangst
zwei Jahrtasende lang daran. Bis ihnen auch Christus und
sein heiliger Geist entschwunden war, nachdem sie sein ge"-%
waltiges Wort in kleine Münze umgemünzt hatten\dopp{} in das fal"-%
sche Gold der Kirche und in das schäbige Kupfer des Staates,
in ungedeckte russische Schecks und in gefälschte amerikanische
Wechsel. Dann war es zu Ende. Dann sind sie verzweifelt,
verzweifelt und verkommen, verkommen und verkrüppelt,
verkrüppelt und verstorben, tot, wahrhaftig tot ohne es zu
wissen, denn nur innerhalb jener Urhülle der Natur ist
Leben.

Um die Tiere, in einzelnen Seitentälern, um einzelne Men"-%
schen lagert sie noch. Dort lagert sie noch, geheimnisvoll-%
heidnischer Rest, hoffnungsvoll dahinkeimend, lebendiger
Raum um einzelne Menschen, dort ist es noch zu spüren.
Und spürt es dann ein andrer Mensch, ein alter abgekämpfter
Mensch der Stadt, dann lebt er auf, dann keimt es auch in
ihm, dann fängt er an und dudelt für ein paar Minuten
selig vor sich hin.