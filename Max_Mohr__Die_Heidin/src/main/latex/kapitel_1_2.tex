%% Erster Teil.

% Seite 43
\kapitel{Zweites Kapitel}

Der liebe Gott, nähme er wieder einmal Menschengestalt an,
um unter seinen Erdenkindern zu wandeln, und versetzte er
sich zu diesem Zweck plötzlich in ein Hotelzimmer im Westen
irgend einer Großstadt, auch der liebe Gott hätte hier nur
zwischen zwei Dingen die Wahl\dopp{} entweder auf die Straße
rennen, um sich mit irgendeiner Predigt lächerlich zu machen,
oder telefonieren. Lea Herse telefonierte.

Nach dem Auspacken der Koffer war sie allmählich in den
kleinen Champagnerschwips der Ankunft hineingeschliddert.
Der Chauffeur hatte vor Freude über ihr Trinkgeld fast geheult
und der Portier hatte sie begrüßt wie ein Ehrenmitglied des
Geheimbundes \aa{}Ging-Gang-Genkerlein and Company\ae{}. Die
Liftjungen waren wie die Ziegenböckchen vom letzten April"-%
wurf um sie herumgesprungen und die Zimmerzofe, als sie
versicherte\dopp{} \aa{}Die Toilette ist im Gang schräg vis-a-vis ganz
hinten links um die Ecke\ae{}, hatte deutlich gelächelt\dopp{} \aa{}Das
kommt für Dich Reh aus der Glonn natürlich gar nicht in
Frage\ausr{}\ae{} Jetzt trug sie ihr Lichtgrünes und telefonierte. Des
Jahrhunderts dicke Bibel lag aufgeschlagn von Par bis Paw
vor ihr, die Nummer war schnell gefunden.

Zentrale Hotel, Amt, Zentrale Pasternak, Herr Professor
ist noch nicht im Haus, tüt -- tüt -- tüt.

Gottseidank. Das war ein Fingerzeig des Himmels. War hätte
sie sagen sollen, wenn sie jetzt plötzlich seine Stimme gehört
hätte\frag{} Erst in der Sekunde, da die kühle Muschel an ihrem
% Seite 44
Ohr gelegen war, hatte sie die Gefahr erkannt. Sie war eine
Träumerin, jawohl, noch immer viel zu tief im Traum der
Glonn versunken. Viele Träume der letzten zwei Wochen
waren ausgeträumt und abgestoßen, doch auch die kühlsten
Pläne der Glonn erwiesen sich jetzt plötzlich immer noch als
trübe Träumerei -- wo steckte die kalte Wirklichkeit, nach der
diese kalte Muschel verlangte\frag{}\label{lS44-1}

Ein gewaltiger Garten, Pflanzen und Tiere und ein paar
Menschen, so rollte die große Kugel um die Sonne. Aus dem
Wald trat ein Mann und schaute über die Ebene. Wer bist
Du, sprach er zu dem Kind am geschlängelten Bach, bist Du
nicht von meinem Stoff, damit wir Hand in Hand marschieren
können über dieses herrliche sonnenbestrahlte Reich\punkte{}%
\eingriff{eS44-1}{Reich\punkte{} ] Reich~\punkte{}}
Oder eine Landstraße, weithin zwischen den Dörfern und
Märkten, die Handwerker zu Fuß, auf Schimmeln und Rappen
die Grandseigneurs und ihre Damen. Wohin so allein, mein
Kind\frag{} Meinen Vater suchen, den Herrn vom Pommeranzen"-%
land\ausr{} Ich bin der Herr vom Pommeranzenland, mein Kind,
doch hast du auch ein kleines Muttermal unter der rechten
Brust\frag{} Hier ist es, mein Vater, in der Form einer kauernden
Taube\punkte{} Oder eine Weltstadt, jeder Tag eine dröhnende
Schlacht, jede Nacht ein Siegesfest mit Feuerwerk, gesegnet,
wer dem stolzen Wachstum dient. Dort sitzt der Mann, der
Brot aus Luft zu machen versteht, die Kellner am Bufett
flüstern sich den Namen zu und die Gäste ringsum starren
mit blanken Augen auf ihn, aber einsam sitzt der große Che"-%
miker vor seinem Schweinsbraten mit Kartoffelsalat, bis
endlich ein Ding in Lichtgrün zu ihm tritt. Erinnern Sie Sich%
\eingriff{eS44-2}{Sich ] sich}
noch an Botzong-Kamin und Gruttenhütte und Daniela
Oldenkott, Professor Pasternak\frag{} Jeden Tag meines Lebens
% Seite 45
denke ich daran, mein Fräulein. So lassen Sie Sich%
\eingriff{eS45-1}{Sich ] sich}%
, aber bitte
nicht sentimental zu werden, ein kleine Geschichte erzählen,
Professor. Kein Wort mehr, meine einzige kleine Tochter,
ich wartete auf diese Stunde\punkte{} Pfui über dies Geträum,
gut für die Glonn, gut für Dichter und Dienstmädchen, wo
steckte die kalte Wirklichkeit, nach der diese kalte Muschel ver"-%
langte\frag{} \aa{}Mein Name ist Lea Herse, ich muß Sie dringend
sprechen, Professor.\ae{} \aanah{}Worum handelt es sich\frag{}\ae{} \aa{}Nicht
um Geld und nicht um Chemie, sondern um eine seltsame familiäre
Sache, ich kann es am Telefon nicht sagen.\ae{} Dann kam er an
die Reihe und gab ihr eine Stunde. Dann traf sie ihn und
überließ sich dem Schicksal. Dann rollte das Schicksal im Zwie"-%
gespräch dahin\punkte{}%
\eingriff{eS45-2}{dahin\punkte{} ] dahin \punkte{}}
Und hier
steckte der Rechenfehler der Glonn\dopp{} im Zwiegespräch. In der Glonn vertraute man noch
auf Zwiegespräche, in der Großstadt fühlte man plötzlich,
daß es nur noch Geschäfte und Monologe gab. Würde sie
nicht einen höchst blamablen Monolog in diese kalte Muschel
sprechen, wenn der Mensch am andern Ende der Welt nicht
die richtige Antwort gab\frag{}

Ach was\ausr{} Selbstverständlich mußte sie ihren Vater noch heute
Aug in Aug sehn\ausr{} Selbstverständlich mußte sich alles andere
von selbst ergeben\ausr{} Wer seine Fahrt vor Beginn zu Ende
denken will, versagt schon beim ersten Schritt. Wie gehst Du
so fröhlich mit tausend Füßen dahin, frug die Heuschrecke den
Tausendfüßler, wie machst Du das\frag{} Und der Tausendfüßler
stoppte und überlegte und kam bis an sein Ende keinen
Schritt mehr voran.

Zentrale Hotel, Amt, Zentrale Pasternak, jawohl, Herr Pro"-%
fessor ist im Haus, er hat eine dringende Sitzung, wird aber
nicht mehr lange dauern, tüt -- tüt -- tüt.

% Seite 46
Eine Sitzung hatte Papi, ei ei\frag{} Vermutlich im Zimmer neben
dem freundlichen Telefonisten\frag{} War man sich also doch schon
bis auf zehn Schritt nahgekommen im unendlichen All\frag{}
Selbstverständlich war es die Einfalt der Glonn, diese plötz"-%
liche Angst, daß es in der Stadt nur noch Monologe und
Geschäfte gäbe. Selbstverständlich gab es auch noch Zwiege"-%
spräche. Reizende Zwiegespräche gab es noch, hin und her und
her und hin, der eine Mensch sagte Bims, da sagte der andere
Mensch Bams, mit Bims-Bams aus einem einzigen Munde
kam keine Menschenseele vom Fleck. Immer hübsch hin und
her im Zwiegespräch, so würde sich auch dieser große Fall
ganz von selbst entwickeln, was konnte viel passieren\ausr{}

Zentrale Hotel, Amt, Zentrale Pasternak, jawohl, eine Mi"-%
nute, sofort --

\aa{}Ja\frag{}\ae{}

Das war eine weibliche Stimme. Der böse Wolf hatte Kreide
gefressen, um die sieben Zicklein zu täuschen.

\aa{}Sie wünschen Herrn Professor selbst\frag{}\ae{}

\aanah{}Wer ist am Apparat\frag{}\ae{}

\aa{}Lea Herse\punkte{}\ae{}

\aa{}Einen Augenblick.\ae{}

Dies also war ein Augenblick. Wer war doch der Mann, vor
dessen Sinn tausend Jahre waren wie ein Augenblick\frag{}

\aa{}Sind Sie noch da\frag{}\ae{} Es war immer noch der Wolf aus dem
Vorzimmer. \aa{}Um was handelt es sich\frag{}\ae{}

\aa{}Ich will Herrn Professor Pasternak sprechen\punkte{}\ae{}

\aa{}Ja in welcher Sache denn\frag{}\ae{}

\aa{}Eine ganz persönliche Sache\punkte{}\ae{}

\aa{}Moment.\ae{}

% Seite 47
Unsinn. Persönliche Sache war Unsinn gewesen. Geschäftliche
Sache hätte es%
\eingriff{eS47-1}{es ] er}
heißen müssen. Es handelte sich um etwas
Chemisches, um irgendein Patent, um die Erfindung, Kaviar
aus Kuhmist zu fabrizieren --

\aa{}Pasternak.\ae{}

\aa{}Professor Pasternak selbst\frag{}\ae{}

\aa{}Ja, wer ist dort\frag{}\ae{}

Sie schwieg. Kein Wort fiel ihr durch den Sinn.

\aa{}Bist Du es, Zilly\frag{}\ae{}

Zilly\frag{} Wer war Zilly\frag{}

\aa{}Halloh\frag{} Was los\frag{}\ae{}

\aa{}Hier ist Miß Dsching-Dschang-Dschenkerlein aus New York\ae{},
sagte sie mit starkem amerikanischem Mauschel-Akzent. \aa{}Ich
bin von die New York-Times befohlen, Herrn Professor
Pasternak um eine kleine Interview zu bitten~--\ae{}

\aanah{}Wer ist das\frag{}\ae{}

\aa{}Dschenkerlein, New York, Times, \fremdsprachlich{photographer and inter"-%
viewer} --\ae{}

\aa{}Ja und\frag{}\ae{}

\aa{}Ich interessieren mich for Ihnen, Professor, weil Sie so
chemisch sind --\ae{}

\aa{}Das ist irgend eine Mystifikation, wer ist denn dort\frag{}\ae{}

\aa{}Dschenkerlein, New York, ich haben eine ähnliche Patent
wie der Herr Professor, ich machen \fremdsprachlich{diamants}\label{lS47-1} aus
Kuhmist --\ae{}

\aa{}Ich verbitte mir Ihre dummen Witze\ausr{}\ae{}

Tüt -- tüt -- tüt.
\abstand{}
% Seite 48
Der junge Mann hieß Bechterew, \fremdsprachlich{Dr.}\label{lS48-1} Fritz
Bechterew, wissenschaftlicher Mitarbeiter einer großen Tageszeitung.
Das Abendessen in dem kleinen russischen Restaurant be"-%
stand aus Borschtsch, Beefsteak, Obstsalat, Wodka. Die Ka"-%
pelle spielte argentinische Tangos, Wiener Walzer, Urwald-%
Krach, Beethoven, \fremdsprachlich{I am lonely without you,} Lieder vom
Rhein.

\aa{}Bitte, sehn Sie mal möglichst unauf"|fällig nach dem dritten
Tisch hinter Ihrem Rücken\ae{}, sagte Bechterew zu Lea. Er
sprach eine Spur Sächsisch, denn er stammte aus Chemnitz,
doch das war nur für geographisch geschulte Ohren zu hören.
Lea Herse nahm ihn für einen amüsanten und gerissenen
Berliner Juden, obwohl er als dumpfer Antisimist und als
englisches Halbblut gelten wollte. Seit einer halben Stunde
analysierte er ihr die Gäste ringsum, ihre Berufe und ihre
Laster, ihre Lächerlichkeit und ihre Tricks, die meisten waren
Freunde oder Bekannte von ihm.

\aa{}Der dritte Tisch hinter Ihnen, der fette kleine Kerl mit der
hübschen blonden Leiche in Schwarz. Was, glauben Sie, ist
der Mann\frag{}\ae{}

\aanah{}Ach, ich will es gar nicht wissen\ae{}, sagte Lea, ohne sich zu
wenden. \aa{}Ich hab schon genug. Halte ich ihn für einen Dichter,
dann ist er gewiß ein Lustmörder, und umgekehrt\ausr{}\ae{}

Bechterew lachte. \aa{}Erraten. Er ist nämlich beides, Dichter
und Lustmörder in einer Person\dopp{} berühmter Kritiker. Martin
Frogge, guter Freund von mir, an den gleichen Zeitungstrust
versklavt wie ich. Wollen Sie ihn kennenlernen\frag{}\ae{}

\aa{}Nein, danke.\ae{}

\aanah{}Verzeihn Sie, wenn ich ihn begrüße, eine Minute nur\frag{}\ae{}

\aa{}Bittebittebitte --\ae{}

% Seite 49
Sie war froh, eine kleine Weile allein zu sein. Der erste Tag
der Expedition ging zu Ende, der erste Ansturm war abge"-%
schlagen, doch der Tag war nicht verloren, morgen würde ihr
das Leben in der Großstadt sicherer von der Hand gehn als
heut. Sehr gut, daß sie unter Führung des Herrn Bechterew
ein paar Stunden auf dem Asphalt herumgerutscht war, ehe
sie den Kampf mit ihrem Vater antrat. Sehr gut, daß sie sich
von einem Fremden auf der Straße hatte ansprechen lassen,
zu einem kleinen Bummel durch die große Welt, die sie nur
aus Büchern und Zeitungen kannte, Straße, Kaufhaus
Kino, Restaurant. Bei dem Blinden mit dem Hund war sie
von Herrn Bechterew angesprochen worden. Er hatte be"-%
hauptet, schon eine ganze Stunde hinter ihr hergetrabt zu
sein, verliebt in ihren Gang, und verlegen, weil er sich noch
niemals einer Dame auf der Straße angehängt hätte, erst die
gemeinsame Ergriffenheit vor dem blinden Bettler mit dem
Hund hätte den Bann gebrochen\dopp{} doch das war sicher Lüge
gewesen. Alles war Lüge ringsum. Aber man mußte diese
Lüge fest ins Auge fassen, um die Wahrheit, die dahinter steckte,
zu packen.

Die Musik spielte einen amerikanischen Marsch. Der Refrain
wurde von allen Nichtbläsern der Kapelle mitgesungen. Lea
geriet in eine heroische Stimmung bei diesen ungewohnten
schrillen Klängen. Wunderbar war das Leben.

Wunderbar war das Leben, wenn man Augen hatte zu
schauen und Ohren zu hören und ein Herz sich zu entscheiden.
Am Rand der steilsten Glonner Wiese lag Frau Herse be"-%
graben, nach einem blutigen Kampf mit den Behörden war
es gelungen, ein Grab außerhalb des Kirchhofs graben zu
dürfen, ein Grab auf eigenem Boden. Fast hätte Lea in letzter
% Seite 50
Stunde den Kampf aufgegeben und den Leichnam an Staat
und Kirche ausgeliefert, schließlich war dieser heidnische
Wunsch ihrer Mutter nur eine Sentimentalität gewesen und
im Tod war alles so gleichgültig. Jetzt kam eine tiefe Freude
über sie, daß sie den Kampf gegen die Beamten siegreich
durchgefochten und die Mutter in der Erde der Glonn einge"-%
bettet hatte.

Ihr verdankte sie es, daß sie ein Herz hatte, sich zu entscheiden.
Nicht aus Weltangst hatte sich Frau Daniela in die Glonn
zurückgezogen, sondern in blanker Entschiedenheit. Sie hatte
sich entschieden, entschieden, entschieden, das war es. Leicht
möglich, daß sie sich für die falsche Seite des Lebens entschie"-%
den hatte, vielleicht mußte man sich in das Chaos stürzen,
mittenhinein, um eingetragen zu werden in das Gästebuch
des Weltgeists\dopp{} aber sie hatte sich wenigstens entschieden, sie
hatte nicht geflunkert wie diese Zeitgenossen ringsum, wie
diese Hin-und-her-Pendler zwischen Gut und Bös. Frau
Daniela hatte sich entschieden, das war die große Erinnerung,
die ihre Tochter ihr an diesem Abend weihte. So oder so,
Glonn oder Weltstadt, tierischer Dünger oder Chemie, es gab
keinen Mittelweg, das war die Erkenntnis ihre rsten Groß"-%
stadt-Tags, das war das heroische Gefühl bei diesen schrillen
amerikanischen Synkopen. Man mußte sich zwischen zwei
Welten entscheiden und es war heroisch, sich entscheiden zu
müssen. Wunderbar war das Leben, wie auch die Entscheidung
in den nächsten Tagen fiel, wenn sie nur fiel.

\aa{}Bitte, Frogge, gehn Sie doch mal möglichst unauf"|fällig an
meiner Lady vorbei\ae{}, sagte unterdessen Bechterew zu dem klei"-%
nen fetten Herrn am dritten Tisch hinter Leas Tisch. \aa{}Ein toller
Fall\ausr{} Möchte furchtbar gern wissen, was Sie von ihr halten.\ae{}

% Seite 51
\aa{}Mensch, löchern Sie mich nicht mit Ihren ewigen Weiber"-%
geschichten -- na, was ist denn, Ober, wo sind denn meine
Muscheln, eine halbe Stunde warte ich schon, bei Euch geht
das Geschäft wieder mal viel zu gut --\ae{}

\aa{}Nein wirklich, Frogge, ein toller Fall. Das Mädchen ist heute
zum ersten mal in ihrem Leben in der Großstadt, die richtige
Unschuld vom Lande, aber klüger als wir alle zusammen.\ae{}

\aa{}Das muß ich mir begucken\ae{}, sagte Martin Frogges Freundin.
Sie hieß Lizzy von Save und war aus Düsseldorf importiert,
um rhythmische Tanzkunst zu studieren.

\aa{}Guck sie Dir an, die Unschuld aus der Dragonerstraße\ae{},
sagte Frogge und ballte wütend die Faust gegen den Kellner,
der am Nachbartisch servierte.

\aa{}Kostet zehn Pfennig\ae{}, sagte die rhythmische Tänzerin und
hielt Bechterew die offene Hand hin.

Bechterew gab ihr ein Markstück. Sie schritt mit einem
scharfen Seitenblick auf Lea rhythmisch zur Toilette.

\aa{}Hören Sie mal, Bechterew\ae{}, sagte Martin Frogge, \aa{}Ihr
gestriger Artikel über die Lehre vom Yoga war aber richtiger
Schmus. Glauben Sie wirklich an dieses Zeug\frag{}\ae{}

\aa{}Ich schwöre Ihnen, Frogge, es ist die einzige Religion, an
die ich noch glauben kann.\ae{}

Martin Frogge bekam seine Muscheln und stürzte sich wie ein
trojanischer Held darüber her.

\aa{}Tatsächlich, Frogge\ausr{} Der Mensch lebt zwischen dem Zentrum
der Erdkugel und dem Zentrum der Sonnenkugel. Wenn der
Mensch in dem Gefühl lebt, die Verbindung zwischen dem
Erdmittelpunkt und dem Sonnenball zu sein, dann hat er
sein wahres Lebensgefühl gefunden.\ae{}

% Seite 52
\aa{}Und wenn er es nicht hat\frag{}\ae{}

\aa{}Dann ist er tot, geistig und seelisch tot, der körperliche Tod
spielt dabei gar keine Rolle.\ae{}

\aa{}Und was bekommen Sie für so einen Aufsatz gezahlt\frag{}\ae{}

\aa{}Das ist Geschäftsgeheimnis.\ae{}

\aanah{}Auf jeden Fall viel zu viel.\ae{}

\aanah{}Auf jeden Fall viel zu wenig, Sie kennen ja unsere Blut"-%
sauger. Aber das Tolle ist, daß ich gerade heute dieses
Mädchen dort treffe, sie hat nämlich wirklich jenes uralte und
doch so moderne Lebensgefühl, von dem ich in meinem
Aufsatz schrieb. Ehrenwort, Frogge, ich spüre es bei ihr ganz
genau.\ae{}

\aanah{}Toller Fall\ae{}, sagte Martin Frogge und legte sich lachend eine
neue Portion Muscheln vor. \aa{}Na was ist, Lizzy\frag{} Ist sie Yoga
oder ist sie Yogurt\frag{}\ae{}

Die rhythmische Tänzerin zählte schweigend neun Zehn"-%
pfennigstücke auf den Tisch.

\aa{}Na, was ist\frag{}\ae{},\eingriff{eS52-1}{ist\frag{}\ae{}, ] ist\frag{}\ae{}} frug Bechterew gespannt und schob das
Kleingeld in die Hosentasche.

\aa{}Schlagen Sie Sich%
\eingriff{eS52-1}{Sich ] sich}
diese Dame aus dem Kopf, Bechterew"-%
chen\ausr{}\ae{}

\aa{}Ist sie nicht hübsch\frag{}\ae{}

\aa{}Hübsch ist sie, sehr hübsch, aber --\ae{} Sie winkte Bechterews
gekräuseltes Haupt zu sich und flüsterte ihm mit gespitztem,
aristokratischem Mäulchen ihre Diagnose ins Ohr.

Martin Frogge brach in Gelächter aus, ohne hinzuhören.
\aa{}Na eben, na was denn, na selbstverständlich\ausr{} Das seh
ich doch von hier aus an ihrem knabenhaften Schulter"-%
geblätter\ausr{}\ae{}

% Seite 53
Er trank strahlend dem zusammengeknickten, wissenschaft"-%
lichen Mitarbeiter seiner Zeitung zu. \aa{}Hoch Yoga, Bechterew\ausr{}
Hoch Yoga, Berlin, Chemnitz und Lesbos\ausr{}\ae{}
\abstand{}
Die Uhr auf dem Nachttisch war stehn geblieben. In der
Glonn wußte man beim Erwachen, wie spät es war vor
allem im Juni. Dort brauchte man nicht erst die Augen zu
öffnen, um nach dem Licht des Tags zu sehn. Drei Uhr
und es piepsten die ersten Vögel, vier Uhr und der Krallen"-%
peter begann die Sense zu dengeln, fünf Uhr und die alte
Nana wälzte sich mir Krach aus ihrer geblümten Kiste, sechs
Uhr und das Konzert des Tags stand auf fortissimo. Hier
wußte man nichts. Totenstille ringsum. Besaß man Geld,
so konnte man sich mitten in der Weltstadt eine Portion
Totenstille kaufen. Totenstille mit Luxussteuer. Und prunk"-%
voll sanken die samtenen Vorhänge des Fensters zum Par"-%
kett herab und garantierten das Dunkel, das Totendunkel
mit Luxussteuer.

Um elf Uhr wollte Herr Bechterew anrufen. Es handelte
sich um eine chinesische Ausstellung, die man unbedingt gesehn
haben mußte. Aber Herr Bechterew war zum Schluß etwas
abgekühlt gewesen, das liebe Halbblut. War er eifersüchtig
gewesen, weil sie sich von Herrn Martin Frogge Theater"-%
karten für heute Abend besorgen lassen wollte\frag{} Nun hatte
sie sich doch noch in Herrn Martin Frogges Gesellschaft ziehn
lassen\ausr{} Um zwei Uhr sollte sie Fräulein Lizzy von Save
treffen, der Name des Lunch-rooms stand auf dem kleinen
Notizblock, viele interessante und rhythmische Menschen und
% Seite 54
sehr billige Knödel mit Kraut gab es in jenem Lunch"-%
room. Dann mußte endlich das Telegramm mit der Adresse
und der guten Ankunft an Terese Nüll und Nana abgesandt
werden. Dann mußte anstandshalber eine alte Tante
Oldenkott besucht werden, die in Berlin wohnte. Bechterew
schätzte die Fahrt bis zur Wohnung dieser alten Tante auf
mindestens eine Stunde, mit der besten Verbindung eine
geschlagene Stunde, alle alten Tanten wohnten an der
Peripherie.

Sie sprang aus dem Bett und beschloß, nach allen Seiten
wortbrüchig zu werden. Nach dem Stand der Sonne war es
schon elf Uhr, sie telefonierte in die Zentrale des Hotels, daß
sie nicht zu sprechen wäre. Herr Bechterew sollte sich bei dem
Blinden mit dem Hund eine andere Ergriffene aufgabeln,
Fräulein von Save sollte ihre Lunch-room-Knödel allein
verzehren, Herr Frogge sollte die Theaterkarten für das Nackt"-%
ballett an Tante Oldenkott schicken, die Arme war siebzig und
hatte gewiß schon lnge nichts Nacktes mehr gesehn. Neun
Telefon-Nummern standen schon auf dem kleinen Notizblock\dopp{}
morgen waren es neunzig, wenn sich pro Kopf zehn neue Be"-%
ziehungen auftaten. In einem Monat ging es ihr wie Mister
Dschenkerlein aus New York\dopp{} der hatte zehn Jahre lang
ununterbrochen telefoniert, und als man ihm dann den Kopf"-%
hörer mit Gewalt abriß, krochen ihm große bleiche Würmer
aus den Ohren, unzählige Würmer, immerzu Würmer, bis
er endlich den letzten Wurm von sich gab und starb.

Aber gesegnet sei das Klima dieser Stadt\ausr{} Eine herrliche Luft
drang durch das offene Fenster, als die Vorhänge zurück"-%
gewürgt waren. Wie ein Nichts wurde der schlechte Atem der
Millionen Lungen vom Himmel aufgeschluckt und in reine
% Seite 55
Ware umgetauscht, von Herzschlag zu Herzschlag ein ewiger
Vorrat an Reinheit und Macht.

Gesegnet diese Luft und gesegnet dieses Badezimmer\ausr{} Bei
Kalt allein konnte man die Brause nicht sehr lange ertragen,
aber bei einer winzigen Mischung mit Warm konnte man ver"-%
weilen. Fing man den Strahl im Nacken auf, so stob der
Wassergott in dunklem Zorn das harte Flußbett der Wirbel"-%
säule hinunter. Doch bog man sich zurück, so glitt er sanft und
hell über das kleine weiche Vordach der Brüste und umspielte
mit zarten Rinnsalen den Bauch und seine Säulen.

Schluß, Frottage, Wäsche, Kleid, Schuhlitzen-Abreißen, Hut,
Frühstück, Taxi, helle Straßen, schnelle Menschen, Portier,
Vorhalle, Lift, zweiter Stock, Anmeldung.

Auf dem Zettel, den der kleine uniformierte Bengel ihr
vorlegte, stand gedruckt\dopp{} \aa{}Frau oder Herr~\ldots{}\ldots{}%
\eingriff{eS55-1}{Herr~\ldots{}\ldots{} ] Herr\ldots{}\ldots{}}
wünscht
Frau oder Herrn~\ldots{}\ldots{}%
\eingriff{eS55-2}{Herrn~\ldots{}\ldots{} ] Herrn\ldots{}\ldots{}}
in Angelegenheit~\ldots{}\ldots{}%
\eingriff{eS55-3}{Angelegenheit~\ldots{}\ldots{} ] Angelegenheit\ldots{}\ldots{}}
zu sprechen.\ae{}
Natürlich wünschte Frau Herse Herrn Pasternak zu sprechen,
doch in welcher Angelegenheit\frag{} Angelegenheit Gruttenhütte\frag{}
Angelegenheit 29.~September 1901\frag{} Angelegenheit \aa{}Eine-%
Waise-im-Sturm-der-Zeit-sucht-ihren-väterlichen-Felsen\frag{}\ae{}

Warum hatte sie den schönen Brief zerrissen, den sie ihm
gestern geschrieben hatte\frag{} Warum wollte sie ihn durchaus
Aug in Aug sehn, bevor sie ihr Geheimnis preisgab\frag{} Wozu
all diese Schwierigkeiten\frag{}

Neben ihr stand ein blutarmer Familienvater und schrieb
ohne Besinnen seine Angelegenheit auf seinen Zettel\dopp{} \aa{}Herr
Meier~1 wünscht Herrn Maier~2 in Angelegenheit Habe-%
nichts-zu-fressen zu sprechen.\ae{} Der uniformierte Bengel be"-%
gann sie bereits mißtrauisch abzuschätzen, weil sie ihre An"-%
gelegenheit nicht wußte. Ja, mein Junge, Deine Angelegen"-%
% Seite 56
heit ist kar, Du bist ein kleiner Maulaufreißer und wirst
mal ein großer Maulaufreißer werden, bei mir liegt die
Angelegenheit etwas schwieriger, mein süßer Gockel mit den
engen Hosen.

Sie zerriß ihren Zettel und gab dem Anmelde-Gockel einen
Fünfziger. Sie ging wieder. Hinab das große Treppenhaus
\aanah{}Wir-sind-wir\ae{}, hindurch durch die große Vorhalle \aa{}Eine-%
feste-Burg-ist-unser-Geld\ae{}, vorbei an dem Portier \aanah{}Alle-%
Menschen-gehören-ins-Zuchthaus\ae{}, auf die Straße.
\abstand{}
Tags zuvor, beim Bummel über den Bummel, hatte Bech"-%
terew von einer schrecklichen Großstadt-Krankheit erzählt, die
er Buden-Angst nannte. Nach Bechterew war die Buden-%
Angst schleichend und unbemerkbar. Bei männlichen Patien"-%
ten konnte man am Anfang der Erkrankung leichte Zer"-%
streutheit und leichte Verstopfung feststellen, bei weiblichen
Patienten ziehende Schmerzen in den Hüften, verstärkte
Schönheitspflege, verstärktes Wippen beim Gang\dopp{} doch dieses
erste Stadium der Krankheit wurde meistens übersehn, weil
es als normal galt. Erst wenn die Buden-Angst in voller
Blüte stand, merkten die Patienten, daß sie vor ihrem eigenen
Selbst und vor ihrer leeren Bude entsetzliche Angst empfanden.
Dann griffen sie ohne Wahl nach den zahlreichen Medika"-%
menten, die von der Großstadt zur Linderung der Buden-%
Angst geboten wurden. Gesteigertes Arbeitstempo war bei
allen Buden-Ängstlern als Linderung ihrer Seuche sehr
beliebt. Für Kassenpatienten wurde außerdem empfohlen
Theater-Kino-Radio-Sport. Bessere Patienten nahmen Reli"-%
gion-Literatur-Psychoanalyse-Auto-Verjüngung. Allen sozi"-%
alen Klassen gemeinsam war der große Verbrauch an Zeitung,
der billigsten Droge gegen Buden-Angst. Niemals jedoch hatte
man gehört, daß ein richtiger Buden-Ängstler durch alle diese
Drogen der großstädtischen Buden-Angst-Industrie geheilt
worden war.

Schon am zweiten Tag ihres neuen Lebens lernte Lea Herse
diese Krankheit kennen. Nach einem sinnlosen Trottoirmarsch
vom Zentrum der Stadt zum Westen der Stadt, kehrte sie
mit müden Füßen und müden Augen in ihr Hotelzimmer
zurück. Aber fünf Minuten später trieb sie die Angst vor der
leeren Bude wieder auf den Asphalt. Nun mußte sie doch in den
neuen Lunch-room wallen, zu dem preiswerten Sauerkraut
der rhythmischen Tänzerin, sie wußte nichts anderes zu tun.

Zum zweitenmal in ihrem Leben ein Bummel über den Bum"-%
mel. Beim Passieren eines Blumenladens fiel ihr ein, daß
jetzt in der Glonn ihr \aa{}Delphinium Berghimmel\ae{} seine erste
Blüte trieb. Zwei Meter hoch, der Stolz ihres Stauden"-%
gartens, der hellblaue Rittersporn \aa{}Delphinium Berg"-%
himmel\ae{}, daneben die lichte \aa{}\fremdsprachlich{Sancy de Parabère}\ae{}, die
Kletterrose des Hersehofs, jetzt blühten sie. Hier blühte die
Buden-Angst.

\aa{}Ich dachte schon, Sie versetzen mich\ae{}, rief Fräulein Lizzy
von Save durch den überfüllten Kohl-room. \aa{}Da wären Sie
% Seite 58
aber schwer reingefallen, mein Kind. Erstens gibt es hier eine
Gemüseplatte, wie sie in der ganzen Stadt nicht mehr zu
finden ist -- Sie sind doch als Landmensch gewiß ein richtiger
Gemüse-Mensch\frag{} Und zweitens habe ich Ihnen hier die klügste
Frau von Berlin mitgebracht, sie ist schon ganz verzweifelt,
weil Sie nicht kommen. Fräulein Lea Herse aus den Alpen --
Frau Johanna Duske aus -- woher stammst Du eigentlich,
Johanna\frag{}\ae{}

\aanah{}Aus dem Nichts\ae{}, sagte die klügste Frau von Berlin und
reichte Lea die Hand. Sie trug eine Hornbrille und war stark
gepudert. Aber hinter den großen Gläsern steckten ein paar
wirklich kluge Augen, ein ähnliches Eisblau wie Leas Augen,
und unter dem Puder steckte eine frische junge Haut. Im
Gegensatz zu der Rhythmischen schien sie ruhig und selbstsicher
zu sein.

\aa{}Sagen Sie offen, Sie wollten mich versetzen\frag{}\ae{}, maulte
Fräulein von Save, nachdem sie für Lea eine Gemüseplatte
bestellt hatte. \aa{}Mein Kind, Sie wollten mich versetzen, ob"-%
wohl Sie mir gestern in die Hand versprochen haben, zu
kommen\frag{}\ae{}

\aa{}Offengestanden ja\ae{}, sagte Lea, \aa{}ich wollte nicht kommen.\ae{}

\aa{}Und dann hat eine geheimnisvolle innere Stimme Sie doch
noch hierhergetrieben\frag{}\ae{}

\aanah{}Ach nein\ae{}, sagte Lea, \aa{}es war nur Buden-Angst, gar nichts
anderes.\ae{}

\aa{}Na so was Entzückendes\ae{}, rief Fräulein von Save begeistert.

\aanah{}Was sagst Du zu dieser frechen Beleidigung, Johanna\frag{}\ae{}

\aa{}Ganz in Ordnung\ae{}, sagte Frau Duske mit ihrem wunder"-%
hübschen tiefen Alt und griff nach der Mittagszeitung. Sie
begann zu lesen und schien Leas Ankunft schon wieder ver"-%
% Seite 59
gessen zu haben. Das war ein wenig kränkend, dennoch fand
Lea dieses freiheitliche Gebaren sehr sympathisch.

Eine Viertelstunde lang war über dem Zeitungsrand nur eine
glatte Stirn und eine glate helle Hartafel zu sehn. Frau
Duske schie sich nicht im geringsten für ihre Umgebung zu
interessieren, weder für den dicken Herrn, der seinen Hut
über ihren Kopf hinweg vom Kleiderständer schwang, noch
für Fräulein Herse aus den Alpen, am wenigsten für Fräulein
von Saves soziale Ideen, die Lea mit der Gemüseplatte
vorgesetzt bekam.

\aa{}Mensch, ich muß gehn\ae{}, sagte die Düsseldorferin endlich und
brach mit dem amerikanischen Armbanduhr-Blick ihr Ge"-%
schwätz ab. \aanah{}Theoretischer Unterricht, Muskellehre, mag ich
gar nicht --\ae{}

Lea geriet in Verlegenheit, während Fräulein von Save
zahlte. Sie wartete noch auf ihre Mehlspeise und Frau Duske
saß hinter ihrer Zeitung, als gäbe es nichts andres auf der
Welt wie Eisenbahnkatastrophen, möblierte Zimmer, Völker"-%
bundssitzungen, Theaterklatsch, fast fabrikneue Markenwagen,
Schwergewichtler.

Sollte sie auf den Apfelstrudel nach Wiener Art verzichten
und mit Fräulein von Save zahlen und geht\frag{} Hatte sie auf
die klügste Frau Berlins einen so üblen Eindruck gemacht,
daß die sich ihretwegen hinter der Zeitung vergrub\frag{} Sie rief
den Kellner zu sich, zu zahlen und zu gehn.

\aanah{}Aber wieso denn\frag{}\ae{} Frau Duske lächelte freundlich über die
Zeitungswand hinüber. \aa{}Bleiben wir nicht noch ein wenig\frag{}\ae{}

\aa{}Ja\frag{}\ae{}, frug Lea verdutzt, \aa{}meinen Sie\frag{}\ae{}

\aa{}Natürlich bleiben Sie\ae{}, sagte Fräulein von Save schnell
und verabschiedete sich. \aa{}Sie sind doch ein freier Mensch, Sie
% Seite 60
stecken doch nicht in der Tretmühle wie ich -- djüs djüs djüs --\ae{}

Sie rannte zur Muskellehre und ließ Lea mit der klügsten
Frau Berlins allein.

\aa{}Ich kann nämlich diesen sozialen Quatsch nicht mehr hören\ae{},
sagte Frau Duske und legte die Zeitung beiseite, während
ihre Freundin noch in der Drehtür steckte. \aa{}Lizzy ist bürgerlich
bis in die Knochen, sie soll die Hände von diesen Dingen lassen
-- aber sie ist reizend, nicht wahr\frag{}\ae{} Dabei blinzelte sie Lea
zu, als wollte sie sagen\dopp{} \aa{}Die blödeste Kuh der Welt, nicht
wahr\frag{}\ae{}

Lea lachte.

\aanah{}Wir verstehn uns\ae{}, sagte Frau Duske mit bezauberndem
Lächeln. \aa{}Sind Sie Kommunistin oder sind Sie Nihilistin\frag{}
Eins von beiden muß man ja schließlich sein\frag{}\ae{}

\aa{}Ich bin gar nix\ae{}, sagte Lea trocken.

\aanah{}Wunderbar\ausr{} Ich bin auch nur beruflich an diesen Dingen
interessiert, nicht substantiell. Ich bin Photographin und
entwerfe nebenbei Dekorationen für die kommunistischen
Theater, ganz interessante Arbeit, jedenfalls besser als bürger"-%
liche Kunst, aber im Grunde der gleiche Schwindel. Inter"-%
essieren Sie Sich\eingriff{eS60-1}{Sich ] sich} für photographische Kunst\frag{}\ae{}

\aa{}O ja\ae{}, sagte Lea, \aa{}was photographieren Sie denn\frag{}\ae{}

