%% Erster Teil.

% Seite 81
\kapitel{Drittes Kapitel}

Professor Pasternak war schon seit vielen Jahren nicht mehr
Chemiker, wie man in der Glonn glaubte. Er war Wirtschafts"-%
führer, Organisator, Industrieller. Mit Fünfungdreißig hatte
er sich einen Namen in der Stickstoff-Chemie gemacht, da
hatte das Kapital den Mann gekauft, um ihn der Wissenschaft
zu entziehen und dem wirtschaftlich-technischen Großbetrieb
zuzuführen. Alle Menschen, denen der Erfolg treu ist, geraten
in den Großbetrieb, in den politischen oder künstlerischen
oder wirtschaftlich-technischen Großbetrieb. Und stecken sie erst
im Großbetrieb, so bleibt ihnen der Erfolg meist bis an ihr
Lebensende treu. Aber diese Treue des Erfolgs gleicht der
Treue einer läufigen Hündin, die nicht auf die Straße gelassen
wird. Draußen bellen die herrlichsten Straßenköter, heroisch
klingt und berauschend der Ruf der Natur durch die Nacht,
doch die Rassehündin liegt treu und erfolgreich auf ihrem
Seidenkissen im Salon, wohin ihr großer Aufstieg sie ver"-%
pflanzt hat. Sie versucht nicht einmal mehr an die Tür zu
kratzen oder ans Fensterkreuz. Sie weiß, es hat keinen Sinn,
alle Notausgänge sind verrammelt an diesem hervorragenden
Platz. Verächtlich horcht sie auf die Straße hinaus, auf das
Geheul der schmutzigen fremden Rüden, dann schläft sie gott"-%
ergeben ein. Nur im Traum näßt sie ein wenig unter sich,
auf das lila Seidenkissen des Erfolgs. So auch die Menschen"-%
wesen im Großbetrieb.

% Seite 82
Frau Zilly Pasternak nahm seit zwanzig Jahren an diesen
Erfolgen ihres Gatten teil. Dazu kamen noch ihre weiblichen
Spezialerfolge. Sie hatte zwei gesunde Kinder geboren,
und sie hatte ihr Hauswesen von Jahr zu Jahr modernisiert.
Außerdem hatte sie erreicht, daß die Pasternaksche Ehe nicht
ein einziges Mal gebrochen worden war. Von ihrer Seite
aus war das kein Kunststück gewesen\dopp{} sie war eine fette
Schwarze mit Feueraugen, und das Feuer ihrer Augen
war das einzige Feuer, das ihr Leib besaß. Aber auch ihren
Gatten hatte sie rein gehalten, tadellos stubenrein. Selbst
während der Epoche, da die Röcke immer kürzer wurden und
das Straßenbild für einen alten Ehemann immer verwirren"-%
dere Formen annahm, hatte sie über alle Konkurrenz ge"-%
siegt.

Fräulein Julia Pasternak, neunzehn, hatte die Feueraugen
der Mama geerbt und malte. Schon seit ihrem zehnten Le"-%
bensjahr war sie stolz auf das Entsetzen, das ihre freiheitliche
Manier im Elternhaus auslöste. Aber auch die Eltern waren
stolz auf ihr Entsetzen über diese neue Jugend. Ein hübsches
verzwicktes Geschöpf, und man erwartete große Dinge von
ihrer Malerei. Sie kam gerade von einer Studienreise zurück,
vierzehn Tage Paris, sie hatte das Café du Dôme gesehn,
zwei Revuen, fünf betrunkene russische Großfürsten, dazu
in einigen geheimen Nepplokalen jene graziösen tuberkolösen
Dirnen, davon die Dichter dichten und auch die Maler gut
verkaufen. Sie sprach von der Liebe wie vom Wetter und von
Petersilie, als hätte sie schon mindestens sieben interessante
Liebhaber gehabt. Aber das war Lüge, sie war eine amerika"-%
nische Jungfrau\dopp{} unberührt und rein und immerzu-darum"-%
herum

% Seite 83
Robert Pasternak, siebzehn, kastanienbraun wie sein Vater
und wie die fremde Dame aus den Alpen. Ein lustiger kleiner
Sportsmann, der sämtliche Rekordzahlen der letzten Saison
auswenig wußte. Nachts träumte er nicht von blauen Wasser"-%
fällen, grünen Weingewächsen, rotem Blut, sondern von
den Schwarz-Weiß-Annonchen berühmter Rennwagen. Er
verliebte sich sofort in die fremde Dame aus den Alpen,
weil sie noch vor dem Abendessen ein paar Bälle mit ihm
schlug, auf dem Tennisplatz hinter dem Ziergarten, und dabei
einen großartig freien Schlag zeigte\dopp{} Vorhand und Rückhand
ausgezeichnet, Netzspiel prima, aber ihren \fremdsprachlich{volley} fand er be"-%
sonders gute Klasse.

Frank Medes war ein Mitschüler Roberts, übern Sonntag
in der Villa Pasternak zu Gast. Ein eleganter schlanker Jude,
auf der Suche nach einem Gott, an dem man mit dem Ver"-%
stand glauben konnte, auf der Suche nach einem Weib, dem
man als Knecht hörig sein durfte, auf der Suche nach einem
Beruf, der garantiert ideal war und dennoch gute Zinsen
warf. Trotzdem er Leas starke Atmosphäre vor allen anderen
Anwesenden spürte, verknallte er sich nicht in sie, denn er war
zur Zeit bis über die Ohren, die etwas abstanden, in Julia
Pasternak verknallt. Er wußte, daß die amerikanische Jung"-%
frau ihn nur zum Narren hielt, doch da er schon sechzehn
Jahre alt war, verlangte er vom Leben gar nichts Besseres
mehr, als zum Narren gehalten zu werden.

Zwischen diesen fünf Menschen saß Lea auf der schneeweißen
Terrasse der Villa Pasternak beim Abendessen. Sie aß kleine
panierte Kotelettes mit Rollerbsen und Tomatensalat. Sie
aß gekühlten Reis mit eingemachten Früchten. Sie unterhielt
sich nach allen Seiten hin wie eine alte Eingeborene des
% Seite 84
Hauses. Und sie kam sich dabei vor wie Mevrouw Genkerlein
aus dem hohen Norden\dopp{} die hatte sechsundzwanzig Jahre
lang geschlafen und als sie erwachte, war es eine fremde
Welt.
\abstand{}
Um zehn Uhr wollte sie sich verabschieden, obwohl sie bereits
in den sanften grauen Familien-Nebel hineingeschwatzt
worden war. Es gab allgemeinen Protest. Man wollte sie
überreden, in dem freien Fremdenzimmer zu übernachten.
Sie sollte den ganzen Sonntag in der Villa Pasternak ver"-%
bringen. Papa Pasternak wollte ihr unbedingt noch einige
wichtige Daten für ihre Erstbesteigungs-Geschichte des Kaiser"-%
gebirgs geben, aus seinen alten Tagebüchern, die konnte er
aber erst morgen finden, morgen früh wollte er das Material
zusammensuchen und eine alpine Konferenz mit ihr abhalten.
Robert Pasternak wollte mit ihr Tennis spielen, ein richtiges
Turnier mit Schiedsrichtern und Linienrichtern, sie war die
einzige würdige Partnerin für ihn, seine Schwester gab
ihm nur zwanzig Meter hohe Ballonbälle, sein Freund Frank
hatte es noch nicht ein einziges Mal auf Einstand mit ihm
gebracht. Julia Pasternak erklärte, daß der Sonntag in der
Großstadt melancholischer wäre, als Lea ahnen könnte\semi{}
glatt zum Verzweifeln ohne Anschluß an irgendeine Sippe\semi{}
der einzige Tag, wo selbst ein Genie im Familien-Klimbim
untertauchen müßte\semi{} selbst in Paris könnte man an einem
Sonntag nichts anderes tun wie im Bett liegen zu bleiben,
allein oder zu zweit. \aa{}Schweig still\ae{}, fuhr ihre Mutter an dieser
Stelle dazwischen, aber Lea hatte die amerikanische Jungfrau
% Seite 85
bereits durchschaut und murmelte ihr zu\dopp{} \aa{}Dann schon wenig"-%
stens zu dritt oder viert, Sie Anfängerin.\ae{} Frank Medes sagte
nichts weiter wie\dopp{} \aa{}Bitte ja, bleiben Sie\ausr{}\ae{}, das allerdings
mit einem Blick, der sich gewaschen hatte, gewaschen in dem
uralten schwarzen Gewässer von Babylon. Nur Frau Zilly
Pasternaks Einladung klang ziemlich kühl -- und das gab, da
Lea ein Weib war, den Ausschlag\dopp{} sie blieb.

Um elf Uhr ging man zu Bett, um für den Festtag mit der
fremden Königstochter aus den Alpen frisch zu sein. Julia
Pasternak führte Lea in das Fremdenzimmer, das ursprüng"-%
lich für Frank Medes bestimmt gewesen war. Frank Medes
bekam ein Reservebett in Roberts Zimmer aufgeschlagen\semi{}
das war sehr gemütlich für die beiden Jungen, Lea brauchte
sich keine Gedanken zu machen. Die Toilette und das Bad
waren \fremdsprachlich{vis-à-vis}, das Nachthemd war von \fremdsprachlich{Printemps}, frisch
gewaschen und noch kein einziges Mal getragen, da war ein
Schreibtisch, wenn sie noch schnell ein Gedicht fabrizieren
wollte, da war eine Abendzeitung, wenn sie sich gern von den
Politikern mit den fetten Glatzköpfen und von den Boxern
mit den Zwei-Zentimeter-Stirnen in den Schlaf wiegen
lassen wollte, und \fremdsprachlich{bonne nuit, ma chérie}.
\abstand{}
Mechanisch entkleidete sie sich. Mit dem gleichen Knick und
in der gleichen Reihenfolge wie in der der Glonn legte sie ihr Zeug
über den Stuhl\dopp{} Kleid, Wäsche, Strümpfe, Strapsen. Dann
beguckte sie das feine Nachtgewand aus Paris. Es war tief
ausgeschnitten und mit imitierten gelben Kirchenspitzen
% Seite 86
besetzt. Obgleich es garantiert jungfräulich war, sah es aus, als
hätten schon Generationen von eleganten Charkutiersgattinnen
darin geschlummert und schlimme Träume hineingeträumt.
Besser nackt zwischen den frischen Leintüchern ruhn\ausr{} Sie nahm
das kostbare Stück zwischen zwei Fingerspitzen und ließ es
hinter dem Kopfend des Betts zu Boden fallen. Dort klappte
es zusammen und blieb liegen und sann auf Rache.

Sie trat vor den Wandspiegel, um sich Gutnacht zu sagen.
Grüß Gott, wie gehts, na, was sagst du zu dieser Geschichte,
Mevrouw Genkerlein\frag{} Mevrouw Genkerlein sagte\dopp{} \aa{}Die
Füchse haben Gruben und die Vögel im Himmel haben Nester,
aber des Menschen Sohn hat nicht, da er sein Haupt hinlege.\ae{}\eingriff{eS86-1}{hinlege.\ae{} ] hinlege\ae{}.}
Sie streckte ihr die Zunge raus und schlüpfte ins Bett.

Schlafen. Sie wie Maffa in sein eigenes Selbst einrollen und
schlafen. Es war gar nichts Schlimmes passiert. Woher kam
plötzlich dieser Todesengel über sie\frag{} Nachdem der sanfte
graue Familien-Nebel zerronnen war, rauschte plötzlich ein
schwarzer Todesengel durch die Welt, warum, warum\frag{}
Nichts wie Nervosität in der fremden Umgebung\ausr{} Erstmal
schlafen und morgen war ein andrer Tag. Laß mich die Arme
um dich schlingen, mein Gemahl, laß mich das Knie noch ein
wenig an mich ziehn, so ist es gut, so ist die Lage wunder"-%
voll, nun rühr dich nicht mehr, mein geliebtes Wesen.

Eine Affenhitze\ausr{} Die Fenster geschlossen, drückende Schwüle
im Zimmer, daher der Todesengel\ausr{} Sie sprang aus dem Bett
und öffnete das Fenster und kühlte ihren Leib im Wind der
Nacht.

Iris und Rosen im Garten, leuchtend durch die Nacht, der
ganze Garten voll Iris und Rosen. Keine andere Pflanze
wie Iris und Rosen in dem großen Garten, eine blödsinnige
% Seite 87
Anlage, sie hatte sich schon bei der Einfahrt darüber geärgert.
Um alle Beete ein schnurgerader Iris-Saum, dazwischen
wie eine Kompagnie Soldaten in Reih und Glied die Rosen"-%
stöcke. Das war sachlich, hatte Robert Pasternak behauptet.
Ach, mein kleiner Halbbruder, das war nicht sachlich, das war
tot, dieser Garten war tot.

Und was hätte man nicht alles aus diesem herrlichen Humus
machen können, man roch bis zum zweiten Stock herauf,
wie gut der Boden war. In der Glonn stieß man noch einem
halben Spatenstich auf Fels, für jede einzelne Pflanze gabs
dort jahrzehntelangen Kampf mit dem Gestein, hier aber
konnten tausend Blütenstauden wachsen, die weichen und die
harten Sorten, chinesisches Gelb, sibirisches Grün, dazwischen
rote Tränende Herzen, die alte deutsche Staude aus der
Minnezeit. Und wo war der hellblaue Rittersporn, und wo war
der Mohn mit allem Zinnober und Purpur der Welt, wirr
durcheinanderblühend wie die Schöpfung selbst\frag{} Sachlich\ausr{}
Sachlich und leer, leer und tot.

Das ganze Haus leer und tot\ausr{} Aus Angst vor dem üppigen
Diesseits ein sachliches Beinhaus\ausr{} Und mach dir nur nichts
vor, Königstochter, dein Vater ist gestorben, dein Reich ist
dahin, es stinkt gen Hmmel hier, bewahre dein Geheimnis,
hüte dich, bewahre dich, lauf auf und lauf davon\ausr{}

Sie öffnete leis die Tür und lauschte auf den dunklen Gang
hinaus. Alles still. Alles zu Bett. Man konnte nackt über den
Gang huschen, ins Bad, Gesicht und Brust und Beine waschen,
vielleicht gab das ein bißchen Schlaf.

Als sie schon die Klinke zum Badezimmer in der Hand hielt,
hörte sie durch das offene Treppenhaus Stimmen aus dem
ersten Stock heraufklingen. Sie tastete sich zum Geländer, es
% Seite 88
waren nur drei Schritte nach rechts, und horchte in den ersten
Stock hinab. Das eintönige Hin und Her einer Bettunter"-%
haltung vor dem Einschlafen. Man konnte kein Wort verstehn.
Man hörte nur, daß es eine männliche und eine weibliche
Stimme war. Ihr Vater und seine Madame\frag{} War es nicht
unanständig, in einem fremden Haus nackt auf dem Flur zu
stehn und zu lauschen\frag{} Äußerst unanständig, ganz gewiß.

Und so mochte es doch unanständig sein\ausr{} Die ganze Mensch"-%
heit war so anständig und brav geworden, je weiter der Fort"-%
schritt fortschritt, desto anständiger und braver wurden sie,
bald war die ganze Menschenwelt eine einzige brave Muster"-%
klasse, es war ganz gut, wenn kurz vor diesem Idealzustand
noch schnell etwas Unanständiges und Böses vor sich ging.
Ach, diese braven Schweinehunde\ausr{} Die einen waren brav,
um nach dem Tod in den Himmel zu kommen, und die andern
ware brav, um schon auf Erden belobt zu werden, von ihrem
Büro-Vorstand, von ihrem Partei-Vorstand, und viele andere
waren brav vor ihrem eigenen Wauwau-Gewissen und Bebe\label{lS88-1}-%
Ideal, ach diese braven Schweinehunde\ausr{} Selbst die Tiere
hatten sie schon brav gemacht, die Wölfe zu braven Hunde-%
Polizisten, die Büffel zu braven Milch-Maschinen, und was
nicht brav war, wurde abgeknallt und ausgetilgt, ach diese
fortschrittlichen braven Schweinehunde\ausr{} Nur die Katzen waren
noch ein wenig bös geblieben, Gottseidank, wenigstens ein"-%
zelne Katzen, wenigstens die Katzen aus der Glonn.

Im Hersehof war eine alte Katze, eine üble Vagabundin.
Oft ließ sie sich wochenlang nicht sehen, da war sie auf Jagd
im Berg. Bei der Heimkehr sprang sie durchs Fenster in die
Stube, als wäre nichts gewesen. Zerschunden und zerfetzt
kam sie von ihren Streunereien zurück und trank fünf Schalen
% Seite 89
Milch auf einen Sitz. Die wollte nicht gestreichelt werden.
Die pfiff auf das Lob des Himmels und pfiff auf das Lob der
Erde, danach die Menschen und die Hunde lechzen. Die war
nicht brav, die war bös geblieben, Gottseidank. Wie diese Berg"-%
katze mußte man sein, wollte man sich nicht in die Musterklasse
der städtischen Menschheit fügen. Wie die Minni-Minni von
der Glonn mußte man sein, ohne Lob und sich selber genug,
böse und einsam, weich im Rückgrat und voller Spaß an
allen Dingen.

Sie tastete sich am Geländer entlang zur Treppe. Sie schlich
auf den Zehenspitzen ein paar Stufen hinab. Nichts zu be"-%
fürchten, alle Leute lagen in der Klappe, keinerlei Gefahr.
War es nicht lustig, den leibhaftigen Vater zu belauschen,
wenn er den Stehkragen und die Weste und die Königs"-%
krone abgelegt hatte\frag{} Sie stieg die ganze Treppe bis zum
ersten Stock hinunter.

Wenn eine Stufe knarrte, stoppte sie ein paar Sekunden.
Den Atem angehalten, eingezogen den gewölbten Leib.
Dann ging es wieder weiter. Bis vor die Tür des Zimmers,
aus welchem das verschlafene Duett erklang. Keine Paster"-%
naksche, wie sie so dahinschlich\ausr{} Eine Minni-Minni aus der
Glonn\ausr{} Einsam und böse, leise, weich im Rückgrat, ein rich"-%
tiger Katzenspaß.
\abstand{}
Tausendmal hatte Professor Pasternak einen Menschen
welcher Ruschkewitz hieß, gewarnt --

Frau Zilly Pasternak hielt diesen Menschen, welcher Rusch"-%
kewitz\eingriff{eS89-1}{Ruschkewitz ] Rusch kewitz} hieß, für einen ausgemachten Trottel --

% Seite 90
Ach, tausendmal reichte gar nicht. Noch vor einer Woche hatte
er dem Menschen, welcher Ruschkewitz hieß, mit krassen
Worten die Wahrheit gesagt --

Unverständlich, wie solche Kerle auf solche Posten kamen\ausr{}
Durch dicke Aktienpakete natürlich --

Tatsächlich war diesmal der Mensch Ruschkewitz richtig hinein"-%
getapft --

Aber richtig --

Richtig hineingetapft --

In den Augen der Frau Zilly Pasternak war der Mensch
Ruschkewitz schon seit Jahren ein Hemmschuh für seine ganze
Umgebung --

Nicht das Kind mit dem Bad ausschütten, Zilly\ausr{} Trotz dieser
üblen Blamage hatte der Mensch Ruschkewitz auch seine
guten Seiten --

Möchte wissen, wo\ausr{} Der Mensch Ruschkewitz war ein ausge"-%
sprochener Hemmschuh für seine ganze Umgebung --

Nein, es hatte auch andere Zeitläufte gegeben --

Ein Hemmschuh in allen Zeitläuften, aber ja --

Nein, in manchen Zeitläuften war er sehr brauchbar gewesen,
das mußte man zugeben --

Nichts mußte man zugeben\ausr{} Ein richtiger Hemmschuh in
allen Zeitläuften --

Nein, das war ungerecht --

Ein Hemmschuh --

Und seine Stellung im Kohle-Öl-Krach --

Ein Hemmschuh --

Glänzend war er gewesen im Kohle-Öl-Krach --

Ein Hemmschuh, ein Hemmschuh, ein Hemmschuh, der Mensch
Ruschkewitz war ein Hemmschuh, in allen Zeitläuften, trotz
% Seite 91
allem Kohle-Öl -- Kohle-Öl -- Kohle-Öl -- bara~--
bara~-- bam -- bam -- bam~-- serifugi -- kech -- kich -- kuch
-- kech -- kech -- und -- immer -- auf -- der -- rechten~--
Bauchseite -- dieser -- dumpfe -- Schmerz -- vielleicht --
war -- es -- doch -- der -- Blinddarm -- hier -- immer --
hier -- weiter~-- unten -- da -- ja -- da -- sitzt -- da -- der --
Blinddarm -- Anton -- sag --

Lea hob vorsichtig den Arm, um sich ein wenig an der Tür"-%
fassung zu lehnen. Kühl die Wand am nackten Unterarm
und fest die blanke Stirn dagegen gepreßt. Leise hin und her
der Odem der Minni-Minni aus der Glonn. Dunkel das Haus,
der Gang, das ganze fremde Jagdgebiet. Der Mensch Rusch"-%
kewitz war ein Hemmschuh und bald auch würde bekannt sein,
obs der Blinddarm oder der Mastdarm war. Und so besah
sich der Mensch, welcher Anton Pasternak hieß, den Bauch
seiner rechtmäßigen Gattin und so hatte der Mensch, welcher
Anton Pasternak hieß, dereinst einen anderen Bauch besehn,
und das war vor vielen Jahren Daniela Oldenkotts lieblicher
junger Mädchenbauch gewesen im Damenraum der Grutten"-%
hütte im Kaisergebirg. Ach nein, dunkel wars in jener Nacht
gewesen im Damenraum der Gruttenhütte im Kaisergebirg,
nichts zu sehn, geschlossen die Augensterne, geschlossen alle
Sinne außer dem Sinn der blühenden Dolde. Aber wo war
er gewesen, der Mensch, welcher Anton Pasternak hieß, als
dann jener liebliche junge Mädchenbauch angeschwollen war
und sich aufgeworfen hatte zur Trommel des Lebens, um
zu zerbersten im Hersehof in der Glonn\frag{} Fort war der Jüng"-%
ling gewesen und allein war die Jungfrau gewesen. Fort
war der Kater gewesen und allein war die Kätzin gewesen
mit ihrer kleinen Minni-Minni aus jener Nacht. Ach du mein
% Seite 92
Vater und hast du meine beiden kleinen roten Fußballen
in deine harte breite Riesenhand genommen, als ich vor dir
in der Wiege lag, um mich daran emporzuzerren und mir mein
Leintuch unterm Rücken glatt zu ziehn\frag{} Ach du mein Vater
und hast du meine zehn ersten Worte vernommen\dopp{} Mama,
Nana, Happi-Happi, Bebe, Bischa-Blumi, Mondi, Balli,
Minni, Wauwau, Hotti\semi{} hast du's\eingriff{eS92-1}{du's ] Du's} mit Entzücken vernommen,
du, he, hast du danach gelauscht\frag{} Warst du's, was mich gelehrt
hat, stolz zu sein vor den Menschen der Straße und klein zu
sein vor dem Wind des Waldes\frag{} Nein, das warst du nicht,
Mensch Pasternak, du nicht, abgerückt bist du von deinem
lebendigen Samen, und es ist ganz gut, daß du abgerückt
bist davon. Wahrhaftig, gut so, besten Dank, meinen aller"-%
verbindlichsten Dank, daß du aus dem Spiel geblieben bist,
Mensch Pasternak\ausr{} Zu einer amerikanischen Jungfrau hast
du meine Halbschwester erzogen, dein anerkanntes geliebtes
Stadtkind Julia Pasternak, ach, die amerikanische Jungfrau,
ich möchte nicht mit ihr tauschen. Ach diese armen amerika"-%
nischen Jungfrauen, es sind ihrer soviel in diesen Zeitläuften\ausr{}
Ach, diese Zeitläufte der armen amerikanischen Jungfrauen,
es sind schlimme Zeitläufte. Ach, der heilige Schoß der Fami"-%
lie, er ist tot in diesen schlimmen Zeitläuften der armen
amerikanischen Jungfrauen\ausr{} Daher meinen verbindlichsten
Dank, Mensch Pasternak, daß du aus dem Spiel geblieben
bist, daß du nichts mehr weißt von jenem andern Bauch im
Damenraum der Gruttenhütte im Kaisergebirg, daß du
nichts ahnst von deiner kleinen Minni-Minni aus der Glonn\ausr{}
Schluß mit dem heiligen und toten Schoß der Familie,
du meine kleine Minni-Minni aus der Glonn, Schluß mit
der heiligen und toten Expedition, entscheide dich, lös dich
% Seite 93
los, lös ich los von diesem kalten, sachlichen Beinhaus deiner
Familie, lös dich los von dieser ganzen Musterklasse der städter"-%
nen Leichenwelt, lös dich los und sei dir selbst genug, du
meine kleine Minni-Minni, meine Minni-Minni aus der
Glonn.

Es war nicht der Blinddarm, Gott sei dank, es war nur eine
kleine Einbildung, Madame Pasternak --

Wahrhaftig, wirklich, ganz gewiß --

Von Blinddarm keine Rede, klatsch klapps klatsch --

Jetzt ist es weg, doch wenn es wiederkommt --

Es kommt nicht wieder, klatsch klapps klatsch, wir kennen ja
diese kleinen Samstag-Abend-Phantasien --

Laß doch --

Du weißt\eingriff{eS93-1}{weißt ] weiß} doch, daß heute Samstag ist --

Du hast dich wohl in die freche kleine Alpenkröte verliebt --

Mensch, die ist ja viel zu jung für mich, die könnte ja meine
Tochter sein, ich liebe nicht die schwierigen harten Knospen
vom April --

Laß --

Ich liebe die reifen und erschlossenen Blüten --

Du bist ja verrückt --

Samstag ist --

Lea floh, ohne Vorsicht, zur Treppe, die Stufen hinauf,
zum zweiten Stock zurück, ins Fremdenzimmer zurück, ohne
Vorsicht, einmal fluchte sie laut auf, die Zehe verprellt, an
einer messingnen Teppichstange, ohne alle Vorsicht, aber sie
wurde nicht gehört, und da stand sie wieder am offenen Fenster
ihres Zimmers, und da schaute sie wieder hinab auf den
Iris-Rosen-Garten, Montag-Dienstag-Mittwoch-Donnerstag-%
Freitag-Samstag, Samstag-Samstag-Samstag-Samstag,
% Seite 94
die große Expedition war nicht sehr ertragreich gewesen, ein
dürrer alter Samstag war die ganze Beute\punkte{} Sie drehte
das Licht an und kleidete sich wieder an. Mechanisch, wie
sie sich vor ein paar Stunden ausgekleidet hatte, kleidete sie
sich wieder an\punkte{} Wäsche, Strümpfe, Strapsen, Schuhe,
Kleid.
\abstand{}
Der Einzug in das Haus des Erfolgs war leichter als der
Auszug aus dem Haus des Erfolgs. Eine kleine Streunerei
durch die fremden Stockwerke war leichter als eine todernste
Flucht auf Nimmerwiedersehn. Die Schwierigkeiten begannen
schon mit dem Brief an die liebenswürdigen Wirte.

Der Brief an die liebenswürdigen Wirte sollte im leeren
Fremdenzimmer des Hauses Pasternak gefunden werden,
um zu erklären, warum der fremde Vogel aus den Alpen
so schnell wieder davongeflogen war. Und warum war er so
schnell wieder davongeflogen\frag{} Fünfmal wurde zu einer ge"-%
bührenden Erklärung angesetzt und fünfmal gerieten auf das
Briefpapier, das Julia Pasternak zur Fabrikation eines Nacht"-%
gedichts überreicht hatte, nur lauter kleine Männchen und
Mondgesichter, heulende Frätzchen und grinsende Frätz"-%
chen, aber keine gebührende Erklärung.

Und so mußten eben die liebenswürdigen Wirte das Nest
ohne gebührende Erklärung leer finden, Schluß\ausr{} Adieu Pa"-%
pa, adieu Mama, adieu geschwisterliche Brut, adieu ohne
gebührende Erklärung, Schluß\ausr{}

Eine Ewigkeit bis zum Parterre hinunter. Zuweilen mit der
tastenden Hand an den elektrischen Knipsern für Treppe und
% Seite 95
Flur vorbei, aber es wurde nicht geknipst. Dunkel der Weg,
und es ging auch so. Dies war die Diele, geradeaus mußte
es zur Haustür gehn, rechts von der Haustür war die Garde"-%
robe. In der Garderobe hing noch Hut und Mantel, dorthin
zuerst. Im Notfall gings auch ohne Hut und Mantel, jedoch
das gab den liebenswürdigen Wirten einen Anlaß zur Ver"-%
folgung -- besser war, es blieb nichts zurück im Haushalt
des Erfolgs. Mantel und Hut und dann ins Freie, und gings
nicht durch die Tür, so mußte es eben durch ein Fenster
gehn.

Doch vor dieser letzten Etappe der Flucht konnte man noch
ein wenig verschnaufen, wozu denn sonst standen diese dicken
Ledersessel in der Pasternakschen Diele herum, ganz gewiß
nicht nur, damit man sich ans Schienbein blaue Flecken stieß.
Jawohl, man konnte noch ein paar Minuten in den Sessel
sinken, den Kopf zur Seite fallen lassen wie ein Baby, die
Beine anziehn wie die Lieblingsfrau des Maharadschas, ein
wenig duseln, drei vier fünf Minuten nur.


Aber es wurde eine ganze Stunde draus und das erste Licht
des Tags drank schon in die Pasternaksche Diele, als sie mit
einem kleinen Ruck und Wehlaut wieder zu sich kam. Was los\frag{}
Schnell wieder ins weißlackierte Fremdenzimmer zurück\frag{}
Ins Bett, den Todesengel vergessen, schlafen, schlafen,
schlafen\frag{} Was los, warum dieser plötzliche Abbruch der großen
Wallfahrt, warum diese katzenmäßige Flucht aus dem freund"-%
lichen Haus des Erfolgs\frag{} Gabs im Menschenleben nur diese
heidnische Lösung der Dinge, gabs nicht auch noch andre
Lösungen der Dinge\frag{}

Gewiß gab es außer der heidnischen Lösung der Dinge noch
andere Lösungen der Dinge. Zwei andre Lösungen der Dinge
% Seite 96
gab es noch außer der heidnischen Lösung der Dinge nach Art
der Glonner Bergkatze. Es gab noch die amerikanische Lösung
der Dinge nach Art der amerikanischen Jungfrau und die
asiatische Lösung der Dinge nach Art des Rabbi von Naza"-%
reth, die gab es noch.

Wie einfach war die Lösung der Dinge nach Art der amerika"-%
nischen Jungfraun\ausr{} Keine Nachtgedanken\dopp{} Mit der letzten
Zigarette ins Bett, mit dem letzten Lächeln in den Schlaf\ausr{}
Keine Taggedanken\dopp{} Mit gutem Stuhlgang wieder ins
Leben zurück\semi{} im Sonnenschein an den Dingen des Lebens
vorüber\ausr{} Keine Berührung mit den Dingen selber\dopp{} nur immer
daran vorüber und dazwischen dahin und darum herum\ausr{}
Die einfachste Lösung der Dinge, ganz gewiß, aber man
mußte dazu geboren sein. Nicht in Amerika mußte man ge"-%
boren sein\dopp{} die Immer-nur-darum-herum-Gespenster wurden
in allen Erdteilen geboren. Nicht in einer bestimmten gesell"-%
schaftlichen Schicht mußte man geboren sein\dopp{} die Immer-nur-%
darum-herum-Gespenster wurden in allen Gesellschafts"-%
schichten geboren. Aber totgeboren mußte man sein, um diesen
gespenstigen Immer-nur-darum-herum-Tanz mitmachen zu
können, von sterbenden Vätern gezeugt und totgeboren von
abgestorbenen Müttern -- nein, die amerikanische Lösung der
Dinge kam nicht in Frage.

Und auch die asiatische Lösung der Dinge nach Art des Rabbi
von Nazareth kam nicht in Frage für die Heidin aus der Glonn.
Es war ja klar, was jene Lehre befahl, wenn die Wallfahrt nach
der Stadt des Lebens auf eine Stadt des Todes stieß, wenn
die Wallfahrt zum lebendigen Vater auf einen Leichnam stieß
im sachlichen Beinhaus des Erfolgs. \aa{}Es war aber eine Kluft
und ein Stein daraufgesetzt und Er sprach\dopp{} Hebt den Stein.
% Seite 97
Spricht zu ihm Martha, die Schwester des Verstorbenen\dopp{}
Herr, er stinket schon, denn er ist vier Tage gelegen. Rief er
mit lauter Stimme\dopp{} Lazarus, komm heraus\ausr{} Und der Ver"-%
storbene kam heraus, gebunden mit Grabtüchern an Füßen
und Händen und sein Gesicht verhüllet mit einem Schweiß"-%
tuch\ausr{}\ae{} Und so mußte die Pilgerin aus der Glonn verharren
und dienen, verharren bei ihrem verstorbnen Vater, dienen
in Glaube und Liebe und Hoffnung, verharren und dienen
wie Major Ellen Wladden von der Heilsarmee, krank mit den
Kranken und tot mit den Toten, und mit lauter Stimme rufen\dopp{}
Lazarus, komm heraus\ausr{}\,, und der Verstorbene kam heraus,
gebunden mit Grabtüchern an Händen und Füßen und sein
Gesicht verhüllet mit einem Schweißtuch\punkte{} Und warum
sollte sie nicht ihrem gesunden Gefühl folgen und davon"-%
laufen, wenn es hieß\dopp{} Er stinket schon und er hat schon lange
gelegen\frag{} Und warum sollte sie den Kadaver nicht liegen
lassen und einen weiten Bogen um seinen Gestank machen,
wie jedes Tier es tat außer dem Aasgeier und dem Leichen"-%
wurm\frag{} Lange genug ruhte der milde Blick des Rabbi von
Nazareth auf dieser Welt, abgewandt hatte er sich schon
davon, Er selbst, und wollte von den neuen Wandlungen
seiner Lehre nichts mehr wissen\dopp{} was also hatte eine
Heidin aus der Glonn noch damit zu tun\frag{} Warum sollte
sie ihr Gesicht verhüllen mit dem Schweißtuch Asiens
und sich binden lassen mit Grabtüchern an Händen und
Füßen\frag{}

Auf, los, raus aus dem Sessel, fort höchste Zeit, schon bald
heller Tag, keine amerikanischen und keine asiatischen Lösun"-%
gen der Dinge, dort war der Mantel, dort war der Hut,
dort war die Tür.

% Seite 98
Natürlich ein Patentschloß mit einem unerforschlichen Mecha"-%
nismus. Sicherheit gegen verbrecherisches Gesindel aus der
Glonn, das mußte sein. Rapp-Schnapp, aber nichts ging auf.
Eine patente Sicherheit gegen Heiden, die davonlaufen
wollten, wenn es stank.

Rapp-Schnapp, nichts. Die amerikanische Technik arbeitete
Hand in Hand mit der asiatischen Grabesmilde, ein mächtiger
Bund. Rapp-Schnapp, hoffnungslos.

Und so mußte es eben durchs Fenster gehen. Nach dem Garten
zu, nicht nach der Straßenfront zu, sonst knallte vielleicht von
der Straße her ein amerikanisch-asiatischer Nachtwächter die
Verbrecherin nieder, weil sie floh, wenn es stank.

Hier war die Küche, hier mußte es gehn. Noch einmal roch es
in der hygienischen gekachelten weißen sauberen Küche nach
Schweißtüchern, dann war das Fenster offen. Frische Luft, ein
Sprung, ein Bums auf die Hände, und wieder auf, sie war
im Freien, sie war im Garten, es war vollbracht.
\abstand{}
Der junge Tag war da, heller als im Haus zu ahnen ge"-%
wesen war. Doch die Sonne war noch drunten, es war die
kühle Stunde. Lea trat an die Hausmauer zurück und stand ein
paar Minuten still. Der Aufsprung auf dem Kies war laut
gewesen, war sie gehört worden\frag{}

Nichts. Niemand war wach. Die Route war frei.

Blödsinnig knirschte der Kies. Aber es schadete nichts, man
schließ ringsum. Die Villa Pasternak schlief, die Straße schlief,
die Villa gegenüber schlief, ganz Europa schließ, es war die
kühle Stunde. Madame Europa lag in der Muffkiste, am Nacht"-%
% Seite 99
gewand die imitierte Kirchenspitze, auf dem Nachtkasten die
Abendzeitung und das künstliche Gebiß, so schlief Madame
Europa und kümmerte sich nicht um Lea, die Krachmacherin
auf dem Kies zwischen Iris und Rosen.

Häßlich waren die Iris, gelb und lila, eine perverse Sorte.
Auch die hochstämmigen Rosen waren mit dem Gehirn ge"-%
züchtet statt mit den Augen. Jetzt wäre es gut gewesen, eine
vertraute Staude zu finden, einen kleinen Steinbrech oder eine
junge Jelängerjelieber, irgend etwas Vertrautes aus dem
millionengearteten Pflanzenreich, das wäre ein kleiner
Trost gewesen, da die Pilgerfahrt zu Ende ging und der junge
Tag noch in der kühlen Stunde steckte.

\begin{verse}Kaiserkron und Päonien rot,\\*
die müssen verzaubert sein,\\*
denn Vater und Mutter sind lange tot,\\*
was blühn sie hier so allein\frag{}
\end{verse}

\noindent{}Nein,