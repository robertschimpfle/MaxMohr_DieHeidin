%% Erster Teil.

% Seite 81
\kapitel{Drittes Kapitel}

Professor Pasternak war schon seit vielen Jahren nicht mehr
Chemiker, wie man in der Glonn glaubte. Er war Wirtschafts"-%
führer, Organisator, Industrieller. Mit Fünfungdreißig hatte
er sich einen Namen in der Stickstoff-Chemie gemacht, da
hatte das Kapital den Mann gekauft, um ihn der Wissenschaft
zu entziehen und dem wirtschaftlich-technischen Großbetrieb
zuzuführen. Alle Menschen, denen der Erfolg treu ist, geraten
in den Großbetrieb, in den politischen oder künstlerischen
oder wirtschaftlich-technischen Großbetrieb. Und stecken sie erst
im Großbetrieb, so bleibt ihnen der Erfolg meist bis an ihr
Lebensende treu. Aber diese Treue des Erfolgs gleicht der
Treue einer läufigen Hündin, die nicht auf die Straße gelassen
wird. Draußen bellen die herrlichsten Straßenköter, heroisch
klingt und berauschend der Ruf der Natur durch die Nacht,
doch die Rassehündin liegt treu und erfolgreich auf ihrem
Seidenkissen im Salon, wohin ihr großer Aufstieg sie ver"-%
pflanzt hat. Sie versucht nicht einmal mehr an die Tür zu
kratzen oder ans Fensterkreuz. Sie weiß, es hat keinen Sinn,
alle Notausgänge sind verrammelt an diesem hervorragenden
Platz. Verächtlich horcht sie auf die Straße hinaus, auf das
Geheul der schmutzigen fremden Rüden, dann schläft sie gott"-%
ergeben ein. Nur im Traum näßt sie ein wenig unter sich,
auf das lila Seidenkissen des Erfolgs. So auch die Menschen"-%
wesen im Großbetrieb.

% Seite 81
Frau Zilly Pasternak nahm seit zwanzig Jahren an diesen
Erfolgen ihres Gatten teil. Dazu kamen noch ihre weiblichen
Spezialerfolge. Sie hatte zwei gesunde Kinder geboren,
und sie hatte ihr Hauswesen von Jahr zu Jahr modernisiert.